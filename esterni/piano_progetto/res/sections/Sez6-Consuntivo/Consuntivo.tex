\section{Consuntivo di periodo}
		In questa sezione verranno riportati il consuntivo per le varie fasi di lavoro, considerando le ore sostenute da ogni componente per ruolo. Il bilancio potrà essere:
		\begin{itemize}
			\item \textbf{positivo:} se sono state necessarie meno ore rispetto a quelle preventivate;	 
			\item \textbf{paritario:} se le ore preventivate rispettano quelle effettive;	 
			\item \textbf{negativo:} se sono state necessarie più ore rispetto a quelle preventivate.
		\end{itemize}
	\subsection{Fase di analisi dei requisiti}
		Le ore di lavoro svolto in questo periodo sono di solo investimento e destinate all'apprendimento personale. Di conseguenza queste ore non sono rendicontate. 
		\subsubsection{Prospetto orario}
			Nella tabella in seguito viene illustrato il cambiamento nel numero d'ore di ogni persona, per ogni ruolo ricoperto:
			
			\rowcolors{2}{white}{lightest-grayest}
			\begin{longtable}{|l|c|c|c|c|c|c|c|}
				\hline
				\rowcolor{lighter-grayer}
				\textbf{Nome} & \textbf{Re} & \textbf{Am} & \textbf{An} & \textbf{Pg}  & \textbf{Pr}   & \textbf{Ve} & \textbf{Totale} \\
				\hline
				\endfirsthead
				
				\hline
				Giuseppe Vito Bitetti 		& 0 & 8(-1) & 10(+1) & 0 & 0 & 12 & 30\\
				\hline
				\hline
				Lorenzo Dei Negri			 & 6(-2) & 0 & 14(+1) & 0 & 0 & 10(+1) & 30\\
				\hline
				\hline
				Nicolò Frison 					& 0 & 8(-2) & 9(+1) & 0 & 0 & 13(+1) & 30\\
				\hline
				\hline
				Fouad Mouad 				& 0 & 7 & 11 & 0 & 0 & 12 & 30\\
				\hline
				\hline
				Mariano Sciacco 			& 8 & 0 & 12 & 0 & 0 & 10 & 30\\
				\hline
				\hline
				Alessandro Tommasin    & 9(-2) & 0 & 12(+2) & 0 & 0 & 9 & 30\\
				\hline
				\hline
				Giovanni Vidotto 			& 0 & 6(-1) & 9(+1) & 0 & 0 & 15 & 30\\
				\hline 
				\caption{Tabella contenente il prospetto orario preventivato per la fase di analisi dei requisiti}
			\end{longtable}
			\pagebreak	
			
			La tabella può essere riassunta nel seguente istogramma:
			
			\begin{figure}[H]
				\centering
				\includegraphics[width=0.8\linewidth]{images/consuntivo/analisiCons1.png}
				\caption{Grafico consuntivo ore/ruolo componenti nella fase di analisi dei requisiti}
				\label{fig:consuntivo grafico suddivisione ruoli fase analisi dei requisiti}
			\end{figure}
			
		\subsubsection{Prospetto economico}
			In base al prospetto orario, quello economico sarà il seguente: 
			
			\rowcolors{2}{white}{lightest-grayest}
			\begin{longtable}{|l|c|c|c|c|c|c|c}
				\hline
				\rowcolor{lighter-grayer}
				\textbf{Ruolo} & \textbf{Ore} & \textbf{Costo in €} \\
				\hline
				\endfirsthead
				
				\hline
				Responsabile & 23 (-4) & 690,00 (-120,00)\\
				\hline
				\hline
				Amministratore & 29 (-4) & 580,00 (-80,00)\\
				\hline
				\hline
				Analista & 77 (+6) & 1.925,00 (+150,00)\\
				\hline
				\hline
				Progettista & - & -\\
				\hline
				\hline
				Programmatore & - & -\\
				\hline
				\hline
				Verificatore & 81 (+2) & 1.215,00 (+30,00)\\
				\hline
				\textbf{Totale} & 210 & 4.410,00 (-20,00)\\
				\hline
				\caption{Tabella contenente il prospetto economico in riferimento al prospetto orario}
			\end{longtable}
			\pagebreak
			
			La tabella può essere riassunta nel seguente areogramma:
			\begin{figure}[H]
				\centering
				\includegraphics[width=0.8\linewidth]{images/consuntivo/analisiCons2.png}
				\caption{Grafico percentuale ore/ruolo nella fase di analisi dei requisiti}
				\label{fig:consuntivo2 grafico costi ruolo fase analisi dei requisiti}
			\end{figure}
		\subsubsection{Conclusioni}
			In questa fase il gruppo ha investito lo stesso numero di ore che erano state preventivate, applicando però delle correzioni per quanto riguarda l'assegnazione oraria per i diversi ruoli:
			\begin{itemize}
				\item \textbf{responsabile:} sono state impiegate meno ore rispetto a quelle preventivate in quanto nella pianificazione del lavoro e nella stesura del piano di progetto sono state riscontrate meno difficoltà del previsto;	 
				\item \textbf{amministratore:} dal momento che i software per la gestione del progetto sono stati individuati e configurati fin da subito e che i documenti sono stati ben strutturati in poco tempo, anche in questo caso sono state impiegate meno ore rispetto a quelle preventivate;
				\item \textbf{analista:} il ruolo di analista ha richiesto alcune ore in più del preventivato in quanto alcuni requisiti hanno richiesti più ore per essere compresi in maniera esaustiva;
				\item \textbf{verificatore:} le ore in surplus che sono state risparmiate dal responsabile e non ancora utilizzate dagli analisti sono state impiegate per effettuare una verifica più attenta dei documenti.  
			\end{itemize}
		\subsubsection{Preventivo a finire}
			Nonostante alcuni ruoli abbiano subito dei cambiamenti orari, il totale delle ore è rimasto invariato, di conseguenza il preventivo a finire resta in linea con quanto dichiarato in \S5.
			\newline
			Inoltre, gli scostamenti dal preventivo non richiedono interventi di mitigazione in quanto la fase di analisi non è rendicontata.

		\subsection{Fase di consolidamento dei requisiti}
		Le ore di lavoro svolte in questo periodo sono volte alla preparazione della presentazione e alla revisione dei requisiti individuati. 
		\subsubsection{Prospetto orario}
			Nella tabella in seguito viene illustrato il cambiamento nel numero d'ore di ogni persona, per ogni ruolo ricoperto:
			
			\rowcolors{2}{lightest-grayest}{white}
			\begin{longtable}{|l|c|c|c|c|c|c|c|}
				\hline
				\rowcolor{lighter-grayer}
				\textbf{Nome} & \textbf{Re} & \textbf{Am} & \textbf{An} & \textbf{Pg}  & \textbf{Pr}   & \textbf{Ve} & \textbf{Totale} \\
				\hline
				\endfirsthead
				
				\hline
				Giuseppe Vito Bitetti 	& 0 & 0 & 5 & 0 & 0 & 0 & 5\\
				\hline
				\hline
				Lorenzo Dei Negri	 	& 0 & 5 & 0 & 0 & 0 & 0 & 5\\
				\hline
				\hline
				Nicolò Frison			   & 0 & 0 & 0 & 0 & 0 & 5 & 5\\
				\hline
				\hline
				Fouad Mouad 			& 1 (-1) & 0 & 1 (+1) & 0 & 0 & 3 & 5\\
				\hline
				\hline
				Mariano Sciacco		 	& 0 & 0 & 3 & 0 & 0 & 2 & 5\\
				\hline
				\hline
				Alessandro Tommasin & 0 & 0 & 5 (+1) & 0 & 0 & 0 (-1) & 5\\
				\hline
				\hline
				Giovanni Vidotto 		 & 2 & 0 & 0 & 0 & 0 & 3 & 5\\
				\hline 
				\textbf{Totale} 			& 3 (-1) &  5 & 14 (+2) & 0 & 0 & 13 (-1) & 35\\
				\hline
				\caption{Tabella contenente il prospetto orario preventivato per la fase di consolidamento dei requisiti}
			\end{longtable}
			\pagebreak	
			
			La tabella può essere riassunta nel seguente istogramma:
			
			\begin{figure}[H]
				\centering
				\includegraphics[width=0.8\linewidth]{images/consuntivo/ConsReqCons.png}
				\caption{Grafico consuntivo ore/ruolo componenti della fase di consolidamento dei Requisiti}
				\label{fig:consuntivo grafico suddivisione ruoli fase di consolidamento dei requisiti}
			\end{figure}
			
		\subsubsection{Prospetto economico}
			In base al prospetto orario, quello economico sarà il seguente: 
			
			\rowcolors{2}{white}{lightest-grayest}
			\begin{longtable}{|l|c|c|c|c|c|c|c|}
				\hline
				\rowcolor{lighter-grayer}
				\textbf{Ruolo} & \textbf{Ore} & \textbf{Costo in €} \\
				\hline
				\endfirsthead
				
				\hline
				Responsabile & 3 (-1) & 90,00 (-30,00)\\
				\hline
				\hline
				Amministratore & 5 & 100,00\\
				\hline
				\hline
				Analista & 14 (+2) & 350,00 (+50,00)\\
				\hline
				\hline
				Progettista & - & -\\
				\hline
				\hline
				Programmatore & -  & -\\
				\hline
				\hline
				Verificatore & 13 (-1) & 195,00 (-15,00)\\
				\hline
				\textbf{Totale} & 35 & 735,00 (+5,00)\\
				\hline
				\caption{Tabella contenente il prospetto economico in riferimento al prospetto orario}
			\end{longtable}
			\pagebreak
			
			La tabella può essere riassunta nel seguente areogramma:
			\begin{figure}[H]
				\centering
				\includegraphics[width=0.8\linewidth]{images/consuntivo/ConsReqCons2.png}
				\caption{Grafico percentuale ore/ruolo nella fase di consolidamento dei requisiti}
				\label{fig:consuntivo grafico costi ruolo fase di consolidamento dei requisiti}
			\end{figure}
		
		\subsubsection{Conclusioni }
			In questa fase il gruppo ha investito il numero di ore che erano state preventivate. È stato necessario però svolgere alcuni cambiamenti nella suddivisione oraria per ruolo, in particolare:
			\begin{itemize}
				\item \textbf{responsabile:} il ruolo di responsabile ha richiesto alcune ore in meno in quanto l'approvazione dei documenti e la redazione delle sezioni sono state meno complesse di ciò che era stato preventivato;
				\item \textbf{analista:} per questo ruolo è stato necessario spendere qualche ora in più in quanto alcuni requisiti, per una corretta comprensione, hanno richiesto una delucidazione esterna con i proponenti del progetto, avvenuta con alcune difficoltà di comunicazione;
				\item \textbf{verificatore:} poiché il lavoro degli analisti è stato svolto con precisione, il lavoro dei verificatori è stato leggermente più veloce di ciò che era stato preventivato.
			\end{itemize}
			Alla luce dei cambiamenti effettuati il risultato è che il gruppo ha risparmiato in totale € 15,00 investendo le stesse ore preventivate.
		\subsubsection{Preventivo a finire}
			Il preventivo a finire non modifica quanto preventivato in \S5 in quanto, al netto di alcune ridistribuzioni orarie, il monte ore previsto per questa fase è stato rispettato.

		\subsection{Fase di progettazione della technology baseline}
		Le ore di lavoro svolte in questo periodo sono volte alla progettazione della \textit{technology baseline} e alla realizzazione della \glock{proof of concept}. 

		\subsubsection{Prospetto orario}
			Nella tabella in seguito viene illustrato il cambiamento nel numero d'ore di ogni persona, per ogni ruolo ricoperto:
			
			\rowcolors{2}{lightest-grayest}{white}
			\begin{longtable}{|l|c|c|c|c|c|c|c|}
				\hline
				\rowcolor{lighter-grayer}
				\textbf{Nome} & \textbf{Re} & \textbf{Am} & \textbf{An} & \textbf{Pg}  & \textbf{Pr}   & \textbf{Ve} & \textbf{Totale} \\
				\hline
				\endfirsthead
				\hline
				Giuseppe Vito Bitetti & 2 (-1) & 0 & 7 & 0 & 0 & 4 (+1) & 13\\
				\hline
				\hline
				Lorenzo Dei Negri & 0 & 5 & 1 & 5 & 0 & 2 & 13\\
				\hline
				\hline
				Nicolò Frison & 0 & 3 & 5 & 5 & 0 & 0 & 13\\
				\hline
				\hline
				Fouad Mouad & 0 & 4 & 3 & 2 & 0 & 4 & 13 \\
				\hline
				\hline
				Mariano Sciacco & 3 & 0 & 1 & 9 & 0 & 0 & 13\\
				\hline
				\hline
				Alessandro Tommasin & 0 & 4 & 6 & 0 & 0 & 3  & 13\\
				\hline
				\hline
				Giovanni Vidotto & 2 & 0 & 5 & 2 & 0 & 4 & 13\\
				\hline 
				\textbf{Totale} & 7 (-1) &  16 & 28 & 23 & 0 & 17 (+1) & 91 \\
				\hline
				
				\caption{Tabella contenente il prospetto orario preventivato per la fase di progettazione della technology baseline}
			\end{longtable}
			
			La tabella può essere riassunta nel seguente istogramma:
			
			\begin{figure}[H]
				\centering
				\includegraphics[width=0.8\linewidth]{images/consuntivo/ConsCorrez1.png}
				\caption{Grafico consuntivo ore/ruolo componenti della fase di progettazione della technology baseline}
				\label{fig:consuntivo grafico suddivione ruoli fase di progettazione della technology baseline}
			\end{figure}
		\pagebreak
			
		\subsubsection{Prospetto economico}
			In base al prospetto orario, quello economico sarà il seguente: 
			
			\rowcolors{2}{white}{lightest-grayest}
			\begin{longtable}{|l|c|c|c|c|c|c|c}
				\hline
				\rowcolor{lighter-grayer}
				\textbf{Ruolo} & \textbf{Ore} & \textbf{Costo in €} \\
				\hline
				\endfirsthead
				\hline
			Responsabile 	    & 7 (-1) & 210,00 (-30,00)\\
			\hline 
			\hline
			Amministratore	  & 16 & 320,00\\
			\hline
			\hline
			Analista 				& 28 & 700,00\\
			\hline
			\hline
			Progettista 		  & 23 & 506,00\\
			\hline
			\hline
			Programmatore 	 & 0 & 0,00\\
			\hline
			\hline
			Verificatore 		  & 17 (+1) & 255,00 (+15,00)\\
			\hline
			\textbf{Totale} 	& 91 & 1.991 (-15,00)\\
			\hline
				
				\caption{Tabella contenente il prospetto economico in riferimento al prospetto orario}
			\end{longtable}
			
			La tabella può essere riassunta nel seguente areogramma:
			\begin{figure}[H]
				\centering
				\includegraphics[width=0.8\linewidth]{images/consuntivo/ConsCorrez2.png}
				\caption{Grafico percentuale ore/ruolo nella fase di progettazione della technology baseline}
				\label{fig:consuntivo grafico costi ruolo fase progettazione della technology baseline}
			\end{figure}

		\subsubsection{Conclusioni}
			Durante la fase di progettazione della technology baseline il gruppo ha lavorato mantenendo il numero di ore che erano state preventivate. È stato necessario però svolgere alcuni cambiamenti nella suddivisione oraria per ruolo, in particolare:
			\begin{itemize}
				\item \textbf{responsabile:} il ruolo del responsabile è stato svolto più velocemente di ciò che era stato preventivato richiedendo quindi un'ora in meno;
				\item \textbf{verificatore:} si è deciso di aggiungere un'ora di verifica aggiuntiva per ricontrollare il piano di progetto e il piano di qualifica, alla luce delle modifiche fatte in seguito alle osservazioni del committente.
			\end{itemize}
			Alla luce dei cambiamenti effettuati il risultato è che il gruppo ha risparmiato € 15,00 investendo le stesse ore preventivate.
		
		\subsubsection{Preventivo a finire}
			A seguito dei recenti cambiamenti apportati alla pianificazione, nonostante alcune lievi modifiche all'assegnazione oraria dei ruoli per la fase corrente, e tenendo in considerazione il leggero risparmio maturato, da parte del committente, sul costo totale del progetto, è stato deciso di non intervenire sul preventivo.	
		
		\subsection{Incremento I}
		Le ore di lavoro svolte in questo periodo sono volte alla configurazione dei container e dell'immagine \glock{Docker} di Kafka. 
		\subsubsection{Prospetto orario}
			Nella tabella in seguito viene illustrato il cambiamento nel numero d'ore di ogni persona, per ogni ruolo ricoperto:
			
			\rowcolors{2}{lightest-grayest}{white}
			\begin{longtable}{|l|c|c|c|c|c|c|c|}
				\hline
				\rowcolor{lighter-grayer}
				\textbf{Nome} & \textbf{Re} & \textbf{Am} & \textbf{An} & \textbf{Pg}  & \textbf{Pr}   & \textbf{Ve} & \textbf{Totale} \\
				\hline
				\endfirsthead
				\hline
				Giuseppe Vito Bitetti & 0 & 0 & 3 & 0 & 2 & 0 & 5\\
				\hline
				\hline
				Lorenzo Dei Negri & 0 & 0 & 0 & 1 (-1) & 0 & 4 (+1) & 5\\
				\hline
				\hline
				Nicolò Frison & 0 & 2 (-1) & 0 & 0 & 0 & 3(+1) & 5\\
				\hline
				\hline
				Fouad Mouad & 3 & 0 & 2 & 0 & 0 & 0 & 5\\
				\hline
				\hline
				Mariano Sciacco & 0 & 0 & 0 & 3 & 2 & 0 & 5\\
				\hline
				\hline
				Alessandro Tommasin & 0 & 2 & 0 & 3 & 0 & 0 & 5\\
				\hline
				\hline
				Giovanni Vidotto & 2 & 0 & 0 & 0 & 4 & 1 & 5\\
				\hline 
				\textbf{Totale} & 3 &  4 (-1) & 5 & 7 (-1) & 8 & 8 (+2) & 35\\
				\hline 
				
				\caption{Tabella contenente il prospetto orario preventivato per l'incremento I}
			\end{longtable}
			\pagebreak
			
			La tabella può essere riassunta nel seguente istogramma:
			
			\begin{figure}[H]
				\centering
				\includegraphics[width=0.8\linewidth]{images/consuntivo/ConsIncr1-1.png}
				\caption{Grafico consuntivo ore/ruolo componenti dell'incremento I}
				\label{fig:consuntivo grafico suddivione ruoli incremento I}
			\end{figure}
			
		\subsubsection{Prospetto economico}
			In base al prospetto orario, quello economico sarà il seguente: 
			
			\rowcolors{2}{white}{lightest-grayest}
			\begin{longtable}{|l|c|c|c|c|c|c|c|}
				\hline
				\rowcolor{lighter-grayer}
				\textbf{Ruolo} & \textbf{Ore} & \textbf{Costo in €} \\
				\hline
				\endfirsthead
				\hline
			Responsabile 	    & 3 & 90,00\\
			\hline 
			\hline
			Amministratore	  & 4 (-1)& 80,00 (-20,00)\\
			\hline
			\hline
			Analista 				& 5 & 125,00\\
			\hline
			\hline
			Progettista 		  & 7 (-1) & 154,00 (-22,00)\\
			\hline
			\hline
			Programmatore 	 & 8 & 120,00\\
			\hline
			\hline
			Verificatore 		  & 8 (+2) & 120,00 (+30,00)\\
			\hline
			\textbf{Totale} 	& 35 & 689,00 (-12,00)\\
			\hline
				
				\caption{Tabella contenente il prospetto economico in riferimento al prospetto orario}
			\end{longtable}
		\pagebreak
			
			La tabella può essere riassunta nel seguente areogramma:
			\begin{figure}[H]
				\centering
				\includegraphics[width=0.8\linewidth]{images/consuntivo/ConsIncr1-2.png}
				\caption{Grafico percentuale ore/ruolo nell'incremento I}
				\label{fig:consuntivo grafico costi ruolo incremento I}
			\end{figure}

		\subsubsection{Conclusioni}
			Durante l'incremento I il gruppo ha lavorato mantenendo il numero di ore che erano state preventivate. È stato necessario però svolgere alcuni cambiamenti nella suddivisione oraria per ruolo, in particolare:
			\begin{itemize}
				\item \textbf{amministratore:} la configurazione degli strumenti da parte dell'amministratore ha richiesto meno tempo di ciò che era stato preventivato.
				\item \textbf{progettista:} grazie all'analisi approfondita svolta nelle fase precedenti per quanto riguarda l'architettura del sistema, il ruolo del progettista ha richiesto un'ora in meno di ciò che era stato preventivato.
				\item \textbf{verificatore:} si è deciso di assegnare il surplus di due ore al verificatore per garantire una qualità superiore del prodotto al committente e al proponente.
			\end{itemize}
			Alla luce dei cambiamenti effettuati il risultato è che il gruppo ha risparmiato in totale € 27,00 investendo le stesse ore preventivate.
		
		\subsubsection{Preventivo a finire}
			L'inizio dello sviluppo sta procedendo in modo positivo, specialmente nel primo periodo di progettazione dell'incremento, che ha richiesto meno risorse di quanto preventivato, segno di una progettazione troppo prudente.
			\newline
			È stato scelto di impiegare le risorse rimaste disponibili per potenziare la verifica in modo da garantire una maggiore qualità nel primo incremento che costituisce la base del prodotto.
			\newline
			Di conseguenza, anche a seguito dei lievi cambiamenti riportati, è stato deciso di non modificare il preventivo in quanto si è ritenuto che la pianificazione svolta garantisca il raggiungimento degli obiettivi preposti nei tempi stabiliti.	

		\subsection{Incremento II }
		Le ore di lavoro svolte in questo periodo sono volte alla creazione della struttura base del \glock{gateway} ed alla decisione della struttura JSON dei dati da inviare a Kafka. Verrà inoltre implementato un primo protocollo di comunicazione.
		\subsubsection{Prospetto orario}
			Nella tabella in seguito viene illustrato il cambiamento nel numero d'ore di ogni persona, per ogni ruolo ricoperto:
			
			\rowcolors{2}{lightest-grayest}{white}
			\begin{longtable}{|l|c|c|c|c|c|c|c|}
				\hline
				\rowcolor{lighter-grayer}
				\textbf{Nome} & \textbf{Re} & \textbf{Am} & \textbf{An} & \textbf{Pg}  & \textbf{Pr}   & \textbf{Ve} & \textbf{Totale} \\
				\hline
				\endfirsthead
				\hline
				Giuseppe Vito Bitetti  & 1 & 0 & 0 & 3 & 1 (+1) & 1 (-1) & 6\\
				\hline
				\hline
				Lorenzo Dei Negri      & 0 & 0 & 0 & 0 & 3 & 3 & 6 \\
				\hline
				\hline
				Nicolò Frison 			  & 3 & 0 & 0 & 0 & 3 & 0 & 6 \\
				\hline
				\hline
				Fouad Mouad 			& 0 & 3 & 2 (-1) & 1 (+1) & 0 & 0 & 6 \\
				\hline
				\hline
				Mariano Sciacco			 & 0 & 0 & 0 & 2 & 0 & 4 & 6 \\
				\hline
				\hline
				Alessandro Tommasin & 0 & 1 & 0 & 0 & 5 & 0 & 6 \\
				\hline
				\hline
				Giovanni Vidotto 		& 0 & 2 & 0 & 4 & 0 & 0 & 6\\
				\hline 
				\textbf{Totale} 		   & 4 &  6 & 2 (-1) & 11 (+1) & 12 (+1) & 8 (-1) & 42\\
				\hline 
				
				\caption{Tabella contenente il prospetto orario preventivato per l'incremento II}
			\end{longtable}
			
			La tabella può essere riassunta nel seguente istogramma:
			
			\begin{figure}[H]
				\centering
				\includegraphics[width=0.8\linewidth]{images/consuntivo/ConsIncr2-1.png}
				\caption{Grafico consuntivo ore/ruolo componenti dell'incremento II}
				\label{fig:consuntivo grafico suddivione ruoli incremento II}
			\end{figure}
			\pagebreak
			
		\subsubsection{Prospetto economico}
			In base al prospetto orario, quello economico sarà il seguente: 
			
			\rowcolors{2}{white}{lightest-grayest}
			\begin{longtable}{|l|c|c|c|c|c|c|c|}
				\hline
				\rowcolor{lighter-grayer}
				\textbf{Ruolo} & \textbf{Ore} & \textbf{Costo in €} \\
				\hline
				\endfirsthead
				\hline
			Responsabile 	    & 4 & 120,00\\
			\hline 
			\hline
			Amministratore	  & 6 & 120,00\\
			\hline
			\hline
			Analista 				& 2 (-1) & 50,00 (-25,00) \\
			\hline
			\hline
			Progettista 		  & 10 (+1) & 220,00 (+22,00)\\
			\hline
			\hline
			Programmatore 	 & 12 (+1) & 180,00 (+15,00)\\
			\hline
			\hline
			Verificatore 		  & 8 (-1) & 120,00 (-15,00) \\
			\hline
			\textbf{Totale} 	& 42 & 810,00 (-3,00) \\
			\hline
				
				\caption{Tabella contenente il prospetto economico in riferimento al prospetto orario}
			\end{longtable}
			
			La tabella può essere riassunta nel seguente areogramma:
			\begin{figure}[H]
				\centering
				\includegraphics[width=0.8\linewidth]{images/consuntivo/ConsIncr2-2.png}
				\caption{Grafico percentuale ore/ruolo nell'incremento II}
				\label{fig:consuntivo grafico costi ruolo incremento II}
			\end{figure}
			\pagebreak

			\subsubsection{Conclusioni}
			Durante l'incremento II il gruppo ha lavorato mantenendo il numero di ore che erano state preventivate. È stato necessario però svolgere alcuni cambiamenti nella suddivisione oraria per ruolo, in particolare:
			\begin{itemize}
				\item \textbf{analista:} la comprensione dei requisiti in riferimento agli incrementi da svolgere ha richiesto un'ora in meno del suddetto ruolo;
				\item \textbf{progettista:} la progettazione del protocollo di comunicazione ha richiesto un'ora in più del ruolo di progettista per idearne la struttura base indispensabile alla comunicazione;
				\item \textbf{programmatore:} la complessità del gateway e l'uso del framework utilizzato ha fatto si che fosse necessaria un'ora in più del ruolo di programmatore;
				\item \textbf{verificatore:} il lavoro in più che è stato necessario per il progettista ha reso più veloce il processo di verifica ed ha portato quindi a risparmiare un'ora del ruolo di verificatore.
			\end{itemize}
			Alla luce dei cambiamenti effettuati il risultato è che il gruppo ha risparmiato in totale € 30,00 investendo le stesse ore preventivate.
		
		\subsubsection{Preventivo a finire}
			Lo sviluppo sta procedendo in modo positivo, nonostante alcune difficoltà riscontrate nell'utilizzo di nuove tecnologie e strumenti necessari per la realizzazione dell'incremento pianificato.
			\newline
			Tuttavia, avendo potuto intensificare la verifica nello scorso incremento, si è ritenuto ragionevole sopportare la sua riduzione in questa fase. In questo modo è stato possibile non intervenire attivamente sul preventivo, potendo, quindi, dedicare più tempo allo studio delle nuove tecnologie utilizzate.

%%%%%%%%%%%%%%%%%%%%%%%%%%%%%%%%%%%%%%%%%%%%%%%%%%%%%%%%%%%%%%%%%%%%%%%%%%%%%%%%%%%%%%%%%%%%%%%%%%%%%%%%%%%%%%%%%%
		
		\subsection{Incremento III }
		Le ore di lavoro svolte in questo periodo sono volte alla progettazione dell'interfaccia delle \glock{API} ed all'implementazione delle operazioni di lettura dei dati tramite le \glock{API}.
		\subsubsection{Prospetto orario}
			Nella tabella in seguito viene illustrato il cambiamento nel numero d'ore di ogni persona, per ogni ruolo ricoperto:
			
			\rowcolors{2}{lightest-grayest}{white}
			\begin{longtable}{|l|c|c|c|c|c|c|c|}
				\hline
				\rowcolor{lighter-grayer}
				\textbf{Nome} & \textbf{Re} & \textbf{Am} & \textbf{An} & \textbf{Pg}  & \textbf{Pr}   & \textbf{Ve} & \textbf{Totale} \\
				\hline
				\endfirsthead
				\hline
				Giuseppe Vito Bitetti & 0 & 0 & 0 & 1 (+1) & 2 (-1) & 3 & 6\\
				\hline
				\hline
				Lorenzo Dei Negri & 0 & 0 & 0 & 3 & 0 & 3 & 6\\
				\hline
				\hline
				Nicolò Frison & 3 & 0 & 0 & 2 & 0 & 1 & 6 \\
				\hline
				\hline
				Fouad Mouad & 0 & 0 & 0 & 3 & 3 & 0 & 6\\
				\hline
				\hline
				Mariano Sciacco & 0 & 3 & 0 & 0 & 3 & 0 & 6 \\
				\hline
				\hline
				Alessandro Tommasin & 0 & 0 & 0 & 2 & 2 & 0 & 6\\
				\hline
				\hline
				Giovanni Vidotto & 0 & 0 & 0 & 3 (+1) & 0 & 3 (-1) & 6 \\
				\hline 
				\textbf{Totale} & 3 &  3 & 0 & 14 (+2) & 10 (-1) & 12 (-1) & 42 \\
				\hline 
				
				\caption{Tabella contenente il prospetto orario preventivato per l'incremento III}
			\end{longtable}
			
			La tabella può essere riassunta nel seguente istogramma:
			
			\begin{figure}[H]
				\centering
				\includegraphics[width=0.8\linewidth]{images/consuntivo/ConsIncr3-1.png}
				\caption{Grafico consuntivo ore/ruolo componenti dell'incremento III}
				\label{fig:consuntivo grafico suddivione ruoli incremento III}
			\end{figure}
			\pagebreak
			
		\subsubsection{Prospetto economico}
			In base al prospetto orario, quello economico sarà il seguente: 
			
			\rowcolors{2}{white}{lightest-grayest}
			\begin{longtable}{|l|c|c|c|c|c|c|c|}
				\hline
				\rowcolor{lighter-grayer}
				\textbf{Ruolo} & \textbf{Ore} & \textbf{Costo in €} \\
				\hline
				\endfirsthead
				\hline
			Responsabile 	    & 3 & 90,00 \\
			\hline 
			\hline
			Amministratore	  & 3 & 60,00\\
			\hline
			\hline
			Analista 				& - & -\\
			\hline
			\hline
			Progettista 		  & 14 (+2) & 308,00 (+44,00)\\
			\hline
			\hline
			Programmatore 	 & 10 (-1) & 150,00 (-15,00) \\
			\hline
			\hline
			Verificatore 		  & 12 (-1) & 180,00 (-15,00)\\
			\hline
			\textbf{Totale} 	& 42 & 788,00 (+14,00)\\
			\hline 
				
				\caption{Tabella contenente il prospetto economico in riferimento al prospetto orario}
			\end{longtable}
			
			La tabella può essere riassunta nel seguente areogramma:
			\begin{figure}[H]
				\centering
				\includegraphics[width=0.8\linewidth]{images/consuntivo/ConsIncr3-2.png}
				\caption{Grafico percentuale ore/ruolo nell'incremento III}
				\label{fig:consuntivo grafico costi ruolo incremento III}
			\end{figure}
			\pagebreak

		\subsubsection{Conclusioni}
			Durante l'incremento III il gruppo ha lavorato mantenendo il numero di ore che erano state preventivate. È stato necessario però svolgere alcuni cambiamenti nella suddivisione oraria per ruolo, in particolare:
			\begin{itemize}
				\item \textbf{progettista:} la progettazione delle API ha richiesto due ore in più del ruolo di progettista per creare la struttura base delle API stesse;
				\item \textbf{programmatore:} la qualità del lavoro svolto in fase di progettazione ha fatto si che fosse necessaria un'ora in meno del ruolo di programmatore per implementare l'incremento pianificato;
				\item \textbf{verificatore:} il lavoro in più che è stato necessario per il progettista ha reso più veloce il processo di verifica ed ha portato quindi a risparmiare un'ora del ruolo di verificatore.
			\end{itemize}
			Alla luce dei cambiamenti effettuati il risultato è che il gruppo ha risparmiato in totale € 16,00 investendo le stesse ore preventivate.
		
		\subsubsection{Preventivo a finire}
			Lo sviluppo continua a procedere in modo positivo, nonostante alcune difficoltà riscontrate nella progettazione di dettaglio dell'incremento pianificato. Tuttavia, l'intensificazione della progettazione ha favorito la codifica di quanto progettato, permettendo un risparmio di tempo da parte del programmatore.
			\newline
			D'altro canto, questo ha anche comportato la riduzione di un'ora del tempo previsto per la verifica, ma, sebbene sia il secondo incremento consecutivo che risente di questo indebolimento della verifica, si ritiene che ciò possa essere recuperato negli incrementi successivi.
			\newline
			Di conseguenza non è stato ritenuto necessario modificare il preventivo definito in \S5.
				
		%%%%%%%%%%%%%%%%%%%%%%%%%%%%%%%%%%%%%%%%%%%%%%%%%%%%%%%%%%%%%%%%%%%%%%%%%%%%%%%%%%%%%%%%%%%%%%%%%%%%%%%%%%%%%%%%%%
		
		\subsection{Incremento IV}
		Le ore di lavoro svolte in questo periodo sono volte alla configurazione di base della web app e reperimento dati dalle \glock{API}.
		\subsubsection{Prospetto orario}
			Nella tabella in seguito viene illustrato il cambiamento nel numero d'ore di ogni persona, per ogni ruolo ricoperto:
			
			\rowcolors{2}{lightest-grayest}{white}
			\begin{longtable}{|l|c|c|c|c|c|c|c|}
				\hline
				\rowcolor{lighter-grayer}
				\textbf{Nome} & \textbf{Re} & \textbf{Am} & \textbf{An} & \textbf{Pg}  & \textbf{Pr}   & \textbf{Ve} & \textbf{Totale} \\
				\hline
				\endfirsthead
				\hline
				Giuseppe Vito Bitetti & 2 & 0 & 0 & 0 & 4 & 0 & 6\\
				\hline
				\hline
				Lorenzo Dei Negri & 2 & 0 & 0 & 1 (-1) & 0 & 3 (+1) & 6\\
				\hline
				\hline
				Nicolò Frison & 0 & 0 & 0 & 1 (-1) & 4 & 1 (+1) & 6\\
				\hline
				\hline
				Fouad Mouad & 0 & 3 & 0 & 0 & 0 & 3 & 6\\
				\hline
				\hline
				Mariano Sciacco & 3 & 0 & 0 & 0 & 3 & 0 & 6\\
				\hline
				\hline
				Alessandro Tommasin & 0 & 0 & 2 & 0 & 0 & 4 & 6\\
				\hline
				\hline
				Giovanni Vidotto & 0 & 0 & 0 & 5 & 0 & 1 & 6\\
				\hline 
				\textbf{Totale} & 7 &  3 & 4 & 7 (-2) & 11  & 12 (+2) & 42 \\
				\hline 
				
				\caption{Tabella contenente il prospetto orario preventivato per l'incremento IV}
			\end{longtable}
			
			La tabella può essere riassunta nel seguente istogramma:
			
			\begin{figure}[H]
				\centering
				\includegraphics[width=0.8\linewidth]{images/consuntivo/ConsIncr4-1.png}
				\caption{Grafico consuntivo ore/ruolo componenti dell'incremento IV}
				\label{fig:consuntivo grafico suddivisione ruoli incremento IV}
			\end{figure}
			\pagebreak
			
		\subsubsection{Prospetto economico}
			In base al prospetto orario, quello economico sarà il seguente: 
			
			\rowcolors{2}{white}{lightest-grayest}
			\begin{longtable}{|l|c|c|c|c|c|c|c|}
				\hline
				\rowcolor{lighter-grayer}
				\textbf{Ruolo} & \textbf{Ore} & \textbf{Costo in €} \\
				\hline
				\endfirsthead
				\hline
			Responsabile 	    & 7 & 210,00\\
			\hline 
			\hline
			Amministratore	  & 3 & 60,00\\
			\hline
			\hline
			Analista 				& 2 & 50,00\\
			\hline
			\hline
			Progettista 		  & 7 (-2) & 154,00 (-44,00)\\
			\hline
			\hline
			Programmatore 	 & 11 & 165,00\\
			\hline
			\hline
			Verificatore 		  & 10 (+2) & 180,00 (+30,00) \\
			\hline
			\textbf{Totale} 	& 42 & 819,00 (-14,00)\\
			\hline
				
				\caption{Tabella contenente il prospetto economico in riferimento al prospetto orario}
			\end{longtable}
			
			La tabella può essere riassunta nel seguente areogramma:
			\begin{figure}[H]
				\centering
				\includegraphics[width=0.8\linewidth]{images/consuntivo/ConsIncr4-2.png}
				\caption{Grafico percentuale ore/ruolo nell'incremento IV}
				\label{fig:consuntivo grafico costi ruolo incremento IV}
			\end{figure}
			\pagebreak	
		\subsubsection*{Conclusioni}
			Durante l'incremento IV il gruppo ha lavorato mantenendo il numero di ore che erano state preventivate. È stato necessario però svolgere alcuni cambiamenti nella suddivisione oraria per ruolo, in particolare:
			\begin{itemize}
				\item \textbf{progettista:} la progettazione della porzione di web app necessaria all'incremento ha richiesto due ore in meno del ruolo di progettista, grazie all'esperienza maturata precedentemente da parte del progettista;
				\item \textbf{verificatore:} il risparmio maturato dal progettista ha reso possibile investire due ore in più nel ruolo di verificatore.
			\end{itemize}
			Alla luce dei cambiamenti effettuati il risultato è che il gruppo ha risparmiato in totale € 30,00 investendo le stesse ore preventivate.
		
		\subsubsection{Preventivo a finire}
			Lo sviluppo procede bene, grazie all'esperienza pregressa del progettista è stato possibile risparmiare del tempo da investire nella verifica, in modo da pareggiare le ore sacrificate in precedenza, senza risentirne a livello di sviluppo dell'incremento.
			\newline
			Di conseguenza il preventivo rimane in linea con quanto definito in \S5.

		%%%%%%%%%%%%%%%%%%%%%%%%%%%%%%%%%%%%%%%%%%%%%%%%%%%%%%%%%%%%%%%%%%%%%%%%%%%%%%%%%%%%%%%%%%%%%%%%%%%%%%%%%%%%%%%%%%
		
		\subsection{Incremento V }
		Le ore di lavoro svolte in questo periodo sono volte all'implementazione della configurazione dinamica di un \glock{gateway}.
		\subsubsection{Prospetto orario}
			Nella tabella in seguito viene illustrato il cambiamento nel numero d'ore di ogni persona, per ogni ruolo ricoperto:
			
			\rowcolors{2}{lightest-grayest}{white}
			\begin{longtable}{|l|c|c|c|c|c|c|c|}
				\hline
				\rowcolor{lighter-grayer}
				\textbf{Nome} & \textbf{Re} & \textbf{Am} & \textbf{An} & \textbf{Pg}  & \textbf{Pr}   & \textbf{Ve} & \textbf{Totale} \\
				\hline
				\endfirsthead
				\hline
				Giuseppe Vito Bitetti & 0 & 2 & 2 & 0 & 0 & 2 & 6\\
				\hline
				\hline
				Lorenzo Dei Negri & 0 & 0 & 0 & 2 & 4 & 0 & 6\\
				\hline
				\hline
				Nicolò Frison & 2 & 0 & 0 & 0 & 0 & 4 & 6\\
				\hline
				\hline
				Fouad Mouad & 3 (+1) & 0 & 0 & 0 & 3 (-1) & 0 & 6 \\
				\hline
				\hline
				Mariano Sciacco & 0 & 0 & 0 & 3(+2) & 3 & 0(-2) & 6\\
				\hline
				\hline
				Alessandro Tommasin & 0 & 0 & 0 & 3 & 3 & 0 & 6\\
				\hline
				\hline
				Giovanni Vidotto & 0 & 2 & 0 & 0 & 4 & 0 & 6\\
				\hline 
				\textbf{Totale} & 5 (+1) & 4 & 2 & 8 (+2) & 17 (-1) & 6(-2) & 42 \\
				\hline 
				
				\caption{Tabella contenente il prospetto orario preventivato per l'incremento V}
			\end{longtable}	
			
			La tabella può essere riassunta nel seguente istogramma:
			
			\begin{figure}[H]
				\centering
				\includegraphics[width=0.8\linewidth]{images/consuntivo/ConsIncr5-1.png}
				\caption{Grafico consuntivo ore/ruolo componenti dell'incremento V}
				\label{fig:consuntivo grafico suddivisione ruoli incremento V}
			\end{figure}
			\pagebreak
			
		\subsubsection{Prospetto economico}
			In base al prospetto orario, quello economico sarà il seguente: 
			
			\rowcolors{2}{white}{lightest-grayest}
			\begin{longtable}{|l|c|c|c|c|c|c|c|}
				\hline
				\rowcolor{lighter-grayer}
				\textbf{Ruolo} & \textbf{Ore} & \textbf{Costo in €} \\
				\hline
				\endfirsthead
				\hline
			Responsabile 	    & 5 (+1) & 150,00 (+30,00)\\
			\hline 
			\hline
			Amministratore	  & 4 & 80,00\\
			\hline
			\hline
			Analista 				& 2 & 50,00\\
			\hline
			\hline
			Progettista 		  & 8 (+2) & 176,00 (+44,00)\\
			\hline
			\hline
			Programmatore 	 & 17 (-1)  & 255 (-15),00 \\
			\hline
			\hline
			Verificatore 		  & 6 (-2) & 90,00 (-30)\\
			\hline
			\textbf{Totale} 	&  42 & 801,00 (+29,00)\\
			\hline
				
				\caption{Tabella contenente il prospetto economico in riferimento al prospetto orario}
			\end{longtable}
			
			La tabella può essere riassunta nel seguente areogramma:
			\begin{figure}[H]
				\centering
				\includegraphics[width=0.8\linewidth]{images/consuntivo/ConsIncr5-2.png}
				\caption{Grafico percentuale ore/ruolo nell'incremento V}
				\label{fig:consuntivo grafico costi ruolo incremento V}
			\end{figure}

		%Conclusioni per la revisione di progettazione
		\subsubsection{Conclusioni}
			Durante l'incremento V il gruppo ha lavorato mantenendo il numero di ore che erano state preventivate. È stato necessario però svolgere alcuni cambiamenti nella suddivisione oraria per ruolo, in particolare:
			\begin{itemize}
				\item \textbf{responsabile:} il ruolo del responsabile è stato svolto più lentamente di ciò che era stato preventivato, richiedendo un'ora in più, a causa dell'onere di approvazione dei documenti per il rilascio esterno che si è rivelato più impegnativo del previsto;
				\item \textbf{progettista:} la progettazione della gestione dinamica della configurazione dei gateway ha richiesto due ore in più del ruolo di progettista, a causa della complessità di aggiornamento della configurazione stessa;
				\item \textbf{programmatore:} la qualità del lavoro svolto in fase di progettazione, unito alla solida base di partenza dello sviluppo del gateway, ha fatto si che fosse necessaria un'ora in meno del ruolo di programmatore per implementare l'incremento pianificato;
				\item \textbf{verificatore:} la qualità del lavoro svolto in fase di codifica, unito alla solida base di partenza del gateway, abbondantemente verificate negli scorsi incrementi, ha fatto si che fossero necessarie due ore in meno del ruolo di programmatore per implementare l'incremento pianificato.
			\end{itemize}
			Alla luce dei cambiamenti effettuati il risultato è che il gruppo ha risparmiato 1,00 € investendo le stesse ore preventivate.
		
		\subsubsection{Preventivo a finire}
			Lo sviluppo procede in modo soddisfacente, nonostante le iniziali difficoltà nella progettazione di dettaglio dell'incremento, la qualità del lavoro svolto ha permesso di ridurre le ore di codifica e verifica senza risentirne a livello di sviluppo e qualità dello stesso.
			\newline
			Quindi, nonostante i cambiamenti effettuati al piano orario, l'incremento è stato realizzato secondo la pianificazione preventivata, perciò il preventivo rimane in linea con quanto definito in \S5.
			
			% Di seguito il preventivo a finire:
			% \rowcolors{2}{white}{lightest-grayest}
			% \begin{longtable}{|c|c|c|c|c|c|c|c}
			% 	\hline
			% 	\rowcolor{lighter-grayer}
			% 	\textbf{Fase} & \textbf{Preventivo in €} & \textbf{consuntivo in €} \\
			% 	\hline
			%	\endfirsthead
			% 	
			% 	\hline
			% 	Analisi & 4.410,00 & 4.390,00\\
			% 	\hline
			% 	\hline
			% 	\textbf{Totale rendicontato} & 13.806,00 & 13.806,00\\
			% 	\hline
			% 	\hline
			% 	\textbf{Totale} & 18.966,00 & 18.966,00\\
			% 	\hline
			% 	\caption{Tabella contenente il preventivo a finire}
			% \end{longtable}

	 	\subsection{Tabella riassuntiva del preventivo a finire}
	 		Di seguito il confronto tra preventivo e consultivo per le varie fasi:
			 \rowcolors{2}{lightest-grayest}{white}
			 \begin{longtable}{|l|c|c|c|}
			 	\hline
			 	\rowcolor{lighter-grayer}
			 	\textbf{Fase} & \textbf{Preventivo in €} & \textbf{Consuntivo in €} & \textbf{Differenza in €}\\
			 	\hline
				\endfirsthead
			 	
			 	\hline
			 	Analisi & 4.430,00 & 4.410,00 & -20,00\\
			 	\hline
			 	\hline
			 	Consolidamento dei requisiti & 730,00 & 735,00 & +5,00\\
			 	\hline
			 	\hline
			 	Technology baseline & 2.006,00 & 1.991,00 & -15,00\\
			 	\hline
			 	\hline
			 	Incremento I & 701,00 & 689,00 & -12,00\\
			 	\hline
			 	\hline
			 	Incremento II & 813,00 & 810,00 & -3,00\\
			 	\hline
			 	\hline
			 	Incremento III & 774,00 & 788,00 & 14,00\\
			 	\hline
			 	\hline
			 	Incremento IV & 833,00 & 819,00 & -14,00\\
			 	\hline
			 	\hline
			 	Incremento V & 772,00 & 801,00 & +29,00\\
			 	\hline
			 	\hline
			 	\textbf{Totale rendicontato} & 13.786,00 & 13.785,00 & -1,00\\
			 	\hline
			 	\hline
			 	\textbf{Totale} & 18.946,00 & 18.930,00 & -16,00\\
			 	\hline
			 	\caption{Tabella contenente il preventivo a finire}
			 \end{longtable}
