\appendix
\addcontentsline{toc}{section}{Appendice}

	\section{Riscontro dei rischi}

		\begin{center}
			\rowcolors{2}{white}{lightest-grayest}
			\begin{longtable}{|c|p{3cm}|p{4cm}|p{4cm}|}
			\hline
			\rowcolor{lighter-grayer}
			\textbf{Codice} & \textbf{Periodo} & \textbf{Problema} & \textbf{Soluzione} \\
			\hline
			\endfirsthead

			\hline
			RSK-OAI-2 & Analisi dei requisiti & Alcuni membri del gruppo durante il periodo fino alla RR hanno avvertito per tempo che non sarebbero riusciti a terminare i loro compiti entro la milestone prefissata. & Sono stati riassegnati i compiti a coloro che avevano maggior tempo a disposizione in modo da completare i compiti assegnati entro la milestone prefissata. \\
			\hline
			RSK-OAI-2 & Analisi dei requisiti & Alcuni membri del gruppo, a causa delle imminenti 3 prove parziali di Cybersecurity all'interno del periodo delle lezioni, hanno riscontrato dei problemi nel terminare le loro mansioni o ad incontrarsi. & I compiti sono stati riassegnati per agevolare i membri del gruppo con difficoltà oppure sono state posticipate di alcuni giorni le date di milestone per la consegna dei compiti assegnati.  \\
			\hline
			RSK-OAI-2 & Terzo periodo, analisi dei requisiti & Alcuni membri del gruppo, dovendo svolgere oltre al progetto di Ingegneria del Software, un altro progetto in parallelo, hanno riscontrato delle difficoltà a svolgere entrambi. & I compiti sono stati riassegnati per agevolare i membri del gruppo con difficoltà.  \\
			\hline
			RSK-OAI-3 & Analisi dei requisiti & Alcuni membri del gruppo, a causa di impegni personali improrogabili, hanno trovato delle difficoltà a partecipare alle riunioni o a restare per tutta la durata dell'incontro. & Gli incontri sono stati spostati in modo che almeno la maggior parte del gruppo fosse stata presente o nel caso di mancanza di uno o più membri le decisioni sarebbero state prese dai membri presenti agli incontri. \\
			\hline
			RSK-PBI-1 & Terzo periodo, analisi dei requisiti & La casella di posta elettronica creata inizialmente per le comunicazioni ufficiali non ha funzionato a dovere. & Si è dovuto ritardare l'inizio della stesura dei casi d'uso per mancata comunicazione con l'azienda e si è cambiato indirizzo email per le comunicazioni ufficiali. \\
			\hline
			RSK-TBT-1 & Quarto periodo, analisi dei requisiti & Un membro del gruppo ha avuto problemi con il suo laptop ed ha riscontrato delle difficoltà a svolgere per tempo le mansioni assegnategli. & Le attività sono state riassegnate per agevolare il proseguo del progetto. \\
			\hline
			RSK-OAI-2 & Progettazione technology baseline &Alcuni membri del gruppo hanno dovuto concludere degli esami universitari all'inizio della fase di progettazione della technology baseline. & Le attività sono state distribuite in modo equo per permettere di svolgere tutta la fase in modo completo, senza rallentamenti. \\
			\hline
			RSK-TAI-1 & Progettazione technology baseline &Alcuni membri del gruppo hanno avuto difficoltà nella configurazione di \textit{PHPStorm}, come strumento di codifica per la webapp. & Il \textit{responsabile}, dopo aver parlato a coloro che avevano problemi sono stati tempestivamente affiancati da alcuni membri del gruppo che avevano già configurato il programma e in breve tempo è stato risolto il problema, usando \glock{Discord} per comunicare. \\
			\hline
			RSK-TBT-1 & Incremento I & Un membro del gruppo ha avuto un malfunzionamento al proprio computer principale, che è stato tempestivamente mandato in assistenza dalla casa produttrice. & Dopo una discussione con il \textit{responsabile}, il membro del gruppo ha provveduto a utilizzare un secondo computer di riserva, meno potente, nel quale sono stati configurati gli strumenti principali che gli hanno permesso comunque di portare a termine i compiti assegnati. \\
			\hline
			RSK-OBT-1 & Incremento III & A causa dell'emergenza COVID-19, il proponente ha deciso di non fare un incontro di persona con il gruppo per vedere un primo \textit{proof of concept}. & È stata accolta la proposta dell'azienda di fare un video dimostrativo, come documentato nella decisione VI\_2020-02-24\_14.1 del verbale VI\_2020-02-24\_14. \\
			\hline
			RSK-TBT-1 & Incremento VI & Il computer da lavoro utilizzato per svolgere le attività di progetto di un membro del gruppo ha avuto un malfunzionamento inaspettato che lo ha costretto a cambiare macchina. & Il responsabile è stato contattato immediatamente dell'accaduto e, trattandosi della parte finale dell'incremento, non sono stati necessari particolari riassegnamenti, visto che i task assegnati erano già stati portati a termine. \\
			\hline
			RSK-TMI-1 & Incremento VII & A seguito della configurazione degli IDE, alcuni programmatori hanno avuto difficoltà nell'uso dei programmi di controllo dello stile di codifica (\textit{linter}). & L'amministratore, dopo essere stato avvisato tramite le comunicazioni automatiche su \glock{Slack}, ha provveduto a dare una mano ai programmatori che presentavano problemi con l'uso dei \textit{linter}, indicando i comandi da effettuare, come riportato nelle normativa di progetto. \\
			\hline
			RSK-TMI-2 & Incremento VIII & A seguito del colloquio con il professor Cardin, sono state riscontrate delle dipendenze circolari che non erano in linea i design pattern da applicare. & Il responsabile ha messo in moto i progettisti e i programmatori affinché risolvessero tempestivamente il problema, cercando una soluzione per rimuovere le dipendenze riscontrate. \\
			\hline
			RSK-OBT-1 & Incremento IX & A causa dell'emergenza COVID-19, non è stato possibile prendere contatti diretti con il proponente per mostrare le nuove specifiche API realizzate fino a quel momento. & Il responsabile ha messo a disposizione del proponente una risorsa interna dei programmatori (\href{api.docs.redroundrobin.site}{https://api.docs.redroundrobin.site}) che mostra la specifica delle API aggiornata in tempo reale in base agli aggiornamenti sulla repository, permettendo di discutere agevolmente su \glock{Slack} di eventuali cambiamenti. \\
			\hline
			RSK-OBT-1 & Incremento X & A causa dell'emergenza COVID-19, non è stato possibile prendere contatti diretti con il proponente per mostrare lo sviluppo della webapp fino a quel momento. & Il responsabile ha prontamente inoltrato al proponente i video tutorial del manuale utente per mostrare i risultati ottenuti fino a quel momento. \\
			\hline

			\caption{Tabella contenente i rischi incontrati}
			\end{longtable}
		\end{center}

	\section{Organigramma}
		
		\subsection{Redazione}
			
			\rowcolors{2}{white}{white}
			\begin{table}[!h]
				\centering
					\begin{tabular}{|l|l|l|}
						\hline
						\textbf{Nominativo} & \textbf{Data di redazione} & \textbf{Firma} \\ \hline
						Lorenzo Dei Negri & 2020-01-10 &  \includegraphics[scale=0.6]{images/firme/lorenzo} \\ \hline
					\end{tabular}
				\caption{Redazione}
			\end{table}
		
		\subsection{Approvazione}
			
			\rowcolors{2}{white}{white}
			\begin{table}[!h]
				\centering
				\begin{tabular}{|l|l|l|}
					\hline
					\textbf{Nominativo} & \textbf{Data di approvazione} & \textbf{Firma} \\ \hline
					\textit{Mariano Sciacco} & 2020-01-13 &  \includegraphics[scale=0.6]{images/firme/mariano}  \\ \hline
					Tullio Vardanega &  & \\ \hline
				\end{tabular}
				\caption{Firme e data di approvazione}
			\end{table}
			
		\subsection{Accettazione dei componenti}
			
			\rowcolors{2}{white}{white}
			\begin{table}[H]
				\centering
				\begin{tabular}{|l|l|l|}
					\hline
					\textbf{Nominativo} & \textbf{Data di accettazione} & \textbf{Firma} \\ \hline
					Giuseppe Vito Bitetti & 2020-01-13 & \includegraphics[scale=0.1]{images/firme/peppe} \\ \hline
					Lorenzo Dei Negri & 2020-01-13 & \includegraphics[scale=0.6]{images/firme/lorenzo} \\ \hline
					Nicolò Frison & 2020-01-13 & \includegraphics[scale=0.6]{images/firme/nicolo} \\ \hline
					Fouad Mouad & 2020-01-13 & \includegraphics[scale=0.6]{images/firme/fouad} \\ \hline
					Mariano Sciacco & 2020-01-13 & \includegraphics[scale=0.6]{images/firme/mariano}  \\ \hline
					Alessandro Tommasin & 2020-01-13 & \includegraphics[scale=0.6]{images/firme/alessandro} \\ \hline
					Giovanni Vidotto & 2020-01-13 & \includegraphics[scale=0.6]{images/firme/giovanni} \\ \hline
				\end{tabular}
				\caption{Firme e date di accettazione dei componenti}
			\end{table}
			
		\subsection{Componenti}
			
			\begin{table}[H]
				\rowcolors{2}{lightest-grayest}{white}
				\centering
				\begin{tabular}{|l|l|l|}
					\hline
					\textbf{Nominativo} & \textbf{Matricola} & \textbf{Indirizzo di posta elettronica} \\ \hline
					Giuseppe Vito Bitetti & 1143329 & giuseppevito.bitetti@studenti.unipd.it \\ \hline
					Lorenzo Dei Negri & 1161729 & lorenzo.deinegri@studenti.unipd.it \\ \hline
					Nicolò Frison & 1147682 & nicolo.frison.1@studenti.unipd.it \\ \hline
					Fouad Mouad & 1170480 & fouad.mouad@studenti.unipd.it \\ \hline
					Mariano Sciacco & 1142498 & mariano.sciacco@studenti.unipd.it \\ \hline
					Alessandro Tommasin & 1189293 & alessandro.tommasin.2@studenti.unipd.it \\ \hline
					Giovanni Vidotto & 1072642 & giovanni.vidotto@studenti.unipd.it \\ \hline
				\end{tabular}
				\caption{Componenti del gruppo di progetto}
			\end{table}
		
