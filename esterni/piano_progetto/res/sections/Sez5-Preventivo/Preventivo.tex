\section{Preventivo}
	In questa sezione verranno riportati i preventivi per le varie fasi di lavoro, considerando che la fasi di analisi e consolidamento dei requisiti non verranno rendicontate nel preventivo finale. 
	La suddivisione oraria dei ruoli per ogni membro del gruppo dovrà rispettare le seguenti regole:
		\begin{itemize}
		\item ogni componente dovrà ricoprire almeno una volta ogni ruolo e per almeno otto ore;
		\item le ore lavorative totali per ogni fase dovranno essere le stesse per ogni componente.
	\end{itemize}
	 Per facilitare la comprensione dei ruoli nelle tabelle, essi verranno abbreviati con le seguenti sigle identificative:
			\begin{itemize}
			\item\textbf{Re:} responsabile;
			\item\textbf{Am:} amministratore;
			\item\textbf{An:} analista;
			\item\textbf{Pg:} progettista;
			\item\textbf{Pr:} programmatore;
			\item\textbf{Ve:} verificatore.
		\end{itemize}
	
	
	
	\subsection{Fase di analisi dei requisiti}
		\subsubsection{Prospetto orario}
			Durante la fase di analisi dei requisiti la distribuzione oraria preventivata dei ruoli di ogni componente del gruppo sarà la seguente:
			
			\rowcolors{2}{white}{lightest-grayest}
			\begin{longtable}{|l|c|c|c|c|c|c|c|}
				\hline
				\rowcolor{lighter-grayer}
				\textbf{Nome} & \textbf{Re} & \textbf{Am} & \textbf{An} & \textbf{Pg}  & \textbf{Pr}   & \textbf{Ve} & \textbf{Totale} \\
				\hline
				\endfirsthead
				
				\hline
				Giuseppe Vito Bitetti & 0 & 9 & 9 & 0 & 0 & 12 & 30\\
				\hline
				\hline
				Lorenzo Dei Negri & 8 & 0 & 13 & 0 & 0 & 9 & 30\\
				\hline
				\hline
				Nicolò Frison & 0 & 10 & 8 & 0 & 0 & 12 & 30\\
				\hline
				\hline
				Fouad Mouad & 0 & 7 & 11 & 0 & 0 & 12 & 30\\
				\hline
				\hline
				Mariano Sciacco & 8 & 0 & 12 & 0 & 0 & 10 & 30\\
				\hline
				\hline
				Alessandro Tommasin & 11 & 0 & 10 & 0 & 0 & 9 & 30\\
				\hline
				\hline
				Giovanni Vidotto & 0 & 7 & 8 & 0 & 0 & 15 & 30\\
				\hline 
				\caption{Tabella contenente il prospetto orario preventivato per la fase di analisi dei requisiti}
			\end{longtable}
			\pagebreak
		
			La tabella può essere riassunta nel seguente istogramma:
		
			\begin{figure}[H]
				\centering
				\includegraphics[width=0.8\linewidth]{./images/preventivo/analisi1.png}
				\caption{Diagramma ore/ruolo componenti nella fase di analisi dei requisiti}
				\label{fig:diagramma suddivione ruoli fase analisi dei requisiti}
			\end{figure}
		
			\subsubsection{Prospetto economico}
			In base al prospetto orario, quello economico sarà il seguente: 
			
			\rowcolors{2}{white}{lightest-grayest}
			\begin{longtable}{|l|c|c|c|c|c|c|c|}
				\hline
				\rowcolor{lighter-grayer}
				\textbf{Ruolo} & \textbf{Ore} & \textbf{Costo in €} \\
				\hline
				\endfirsthead
				
				\hline
				Responsabile & 27 & 810,00\\
				\hline
				\hline
				Amministratore & 33 & 660,00\\
				\hline
				\hline
				Analista & 71 & 1.775,00\\
				\hline
				\hline
				Progettista & - & -\\
				\hline
				\hline
				Programmatore & - & -\\
				\hline
				\hline
				Verificatore & 79 & 1.185,00\\
				\hline
				\textbf{Totale} & 210 & 4.430,00\\
				\hline
				\caption{Tabella contenente il prospetto economico in riferimento al prospetto orario}
			\end{longtable}
			\pagebreak
		
			La tabella può essere riassunta nel seguente areogramma:
			\begin{figure}[H]
				\centering
				\includegraphics[width=0.8\linewidth]{./images/preventivo/analisi2.png}
				\caption{Diagramma percentuale ore/ruolo nella fase di analisi dei requisiti}
				\label{fig:diagramma costi ruolo fase analisi dei requisiti}
			\end{figure}
		
		
		
	\subsection{Fase di consolidamento dei requisiti}
			\subsubsection{Prospetto orario}
			Durante la fase di consolidamento dei requisiti la distribuzione oraria preventivata dei ruoli di ogni componente del gruppo sarà la seguente:
			
			\rowcolors{2}{lightest-grayest}{white}
			\begin{longtable}{|l|c|c|c|c|c|c|c|}
				\hline
				\rowcolor{lighter-grayer}
				\textbf{Nome} & \textbf{Re} & \textbf{Am} & \textbf{An} & \textbf{Pg}  & \textbf{Pr}   & \textbf{Ve} & \textbf{Totale} \\
				\hline
				\endfirsthead
				
				\hline
				Giuseppe Vito Bitetti & 0 & 0 & 5 & 0 & 0 & 0 & 5\\
				\hline
				\hline
				Lorenzo Dei Negri & 0 & 5 & 0 & 0 & 0 & 0 & 5\\
				\hline
				\hline
				Nicolò Frison & 0 & 0 & 0 & 0 & 0 & 5 & 5\\
				\hline
				\hline
				Fouad Mouad & 2 & 0 & 0 & 0 & 0 & 3 & 5\\
				\hline
				\hline
				Mariano Sciacco & 0 & 0 & 3 & 0 & 0 & 2 & 5\\
				\hline
				\hline
				Alessandro Tommasin & 0 & 0 & 4 & 0 & 0 & 1 & 5\\
				\hline
				\hline
				Giovanni Vidotto & 2 & 0 & 0 & 0 & 0 & 3 & 5\\
				\hline 
				\textbf{Totale} & 4 &  5 & 12 & 0 & 0 & 14 & 35\\
				\hline
				\caption{Tabella contenente la distribuzione oraria preventivata per il periodo di consolidamento dei requisiti}
			\end{longtable}
			\pagebreak
		
			La tabella può essere riassunta nel seguente istogramma:
			\begin{figure}[H]
				\centering
				\includegraphics[width=0.8\linewidth]{./images/preventivo/consRequisiti1.png}
				\caption{Diagramma ore/ruolo componenti nella fase di consolidamento dei requisiti}
				\label{fig:diagramma suddivione ruoli fase consolidamento requisiti}
			\end{figure}
		
		\subsubsection{Prospetto economico}
		In base al prospetto orario, quello economico sarà il seguente: 
		
			\rowcolors{2}{white}{lightest-grayest}
			\begin{longtable}{|l|c|c|c|c|c|c|c|}
				\hline
				\rowcolor{lighter-grayer}
				\textbf{Ruolo} & \textbf{Ore} & \textbf{Costo in €} \\
				\hline
				\endfirsthead
				
				\hline
				Responsabile & 4 & 120,00\\
				\hline
				\hline
				Amministratore & 5 & 100,00\\
				\hline
				\hline
				Analista & 12 & 300,00\\
				\hline
				\hline
				Progettista & - & -\\
				\hline
				\hline
				Programmatore & -  & -\\
				\hline
				\hline
				Verificatore & 14 & 210,00\\
				\hline
				\textbf{Totale} & 35 & 730,00\\
				\hline
				\caption{Tabella contenente il prospetto economico in riferimento al prospetto orario}
			\end{longtable}
			\pagebreak
			
			La tabella può essere riassunta nel seguente areogramma:
			\begin{figure}[H]
				\centering
				\includegraphics[width=0.8\linewidth]{./images/preventivo/consRequisiti2.png}
				\caption{Diagramma percentuale ore/ruolo nella fase di consolidamento dei requisiti}
				\label{fig:diagramma costi ruolo fase consolidamento dei requisiti}
			\end{figure}
	
	
	
	\subsection{Fase di progettazione della technology baseline e correzioni documenti}
		\subsubsection{Prospetto orario}
		Durante questa fase verranno stabiliti gli incrementi, corretti i documenti e progettata la \textit{technology baseline}. La distribuzione oraria preventivata dei ruoli di ogni componente del gruppo sarà la seguente:
		
		\rowcolors{2}{lightest-grayest}{white}
		\begin{longtable}{|l|c|c|c|c|c|c|c|}
			\hline
			\rowcolor{lighter-grayer}
			\textbf{Nome} & \textbf{Re} & \textbf{Am} & \textbf{An} & \textbf{Pg}  & \textbf{Pr}   & \textbf{Ve} & \textbf{Totale} \\
			\hline
			\endfirsthead
			
			\hline
			Giuseppe Vito Bitetti 		& 3 & 0 & 7 & 0 & 0 & 3 & 13\\
			\hline
			\hline
			Lorenzo Dei Negri			& 0 & 5 & 1 & 5 & 0 & 2 & 13\\
			\hline
			\hline
			Nicolò Frison				   & 0 & 3 & 5 & 5 & 0 & 0 & 13\\
			\hline
			\hline
			Fouad Mouad 				& 0 & 4 & 3 & 2 & 0 & 4 & 13\\
			\hline
			\hline
			Mariano Sciacco 			& 3& 0 & 1 & 9 & 0 & 0 & 13\\
			\hline
			\hline
			Alessandro Tommasin    & 0 & 4 & 6 & 0 & 0 & 3 & 13\\
			\hline
			\hline
			Giovanni Vidotto 			& 2 & 0 & 5 & 2 & 0 & 4 & 13\\
			\hline 
			\textbf{Totale}			 & 8 &  16 & 28 & 23 & 0 & 16 & 91\\
			\hline
			\caption{Tabella contenente il prospetto orario preventivato per il periodo di progettazione della technology baseline}
		\end{longtable}
		\pagebreak
		
		La tabella può essere riassunta nel seguente istogramma:
		\begin{figure}[H]
			\centering
			\includegraphics[width=0.8\linewidth]{./images/preventivo/progArch1.png}
			\caption{Diagramma ore/ruolo componenti nella fase di progettazione della technology baseline}
			\label{fig:diagramma suddivione ruoli fase progettazione della technology baseline}
		\end{figure}
	
		\subsubsection{Prospetto economico}
		In base al prospetto orario, quello economico sarà il seguente: 
		
		\rowcolors{2}{white}{lightest-grayest}
		\begin{longtable}{|l|c|c|c|c|c|c|c|}
			\hline
			\rowcolor{lighter-grayer}
			\textbf{Ruolo} & \textbf{Ore} & \textbf{Costo in € } \\
			\hline
			\endfirsthead
			
			\hline
			Responsabile 	    & 8 & 240,00\\
			\hline 
			\hline
			Amministratore	  & 16 & 320,00\\
			\hline
			\hline
			Analista 				& 28 & 700,00\\
			\hline
			\hline
			Progettista 		  & 23 & 506,00\\
			\hline
			\hline
			Programmatore 	 & - & -\\
			\hline
			\hline
			Verificatore 		  & 16 & 240,00\\
			\hline
			\textbf{Totale} 	& 91 & 2.006,00\\
			\hline
			\caption{Tabella contenente il prospetto economico in riferimento al prospetto orario}
		\end{longtable}
		\pagebreak
		
		La tabella può essere riassunta nel seguente areogramma:
		\begin{figure}[H]
			\centering
			\includegraphics[width=0.8\linewidth]{./images/preventivo/progArch2.png}
			\caption{Diagramma percentuale ore/ruolo nella fase di progettazione della technology baseline}
			\label{fig:diagramma costi ruolo fase progettazione della technology baseline}
		\end{figure}
	
	\subsection{Progettazione e codifica del Proof of Concept e funzionalità essenziali}
	I successivi quattro incrementi, che vedremo in dettaglio, comporranno la fase stessa.  
		\subsubsection{Incremento I}
			\paragraph{Prospetto orario}
			Durante il primo incremento la distribuzione oraria preventivata dei ruoli di ogni componente del gruppo sarà la seguente:
			
			\rowcolors{2}{lightest-grayest}{white}
			\begin{longtable}{|l|c|c|c|c|c|c|c|}
				\hline
				\rowcolor{lighter-grayer}
				\textbf{Nome} & \textbf{Re} & \textbf{Am} & \textbf{An} & \textbf{Pg}  & \textbf{Pr}   & \textbf{Ve} & \textbf{Totale} \\
				\hline
				\endfirsthead
				
				\hline
				Giuseppe Vito Bitetti 		 & 0 & 0 & 3 & 0 & 2 & 0 & 5\\
				\hline
				\hline
				Lorenzo Dei Negri			 & 0 & 0 & 0 & 2 & 0 & 3 & 5\\
				\hline
				\hline
				Nicolò Frison				    & 0 & 3 & 0 & 0 & 0 & 2 & 5\\
				\hline
				\hline
				Fouad Mouad 				 & 3 & 0 & 2 & 0 & 0 & 0 & 5\\
				\hline
				\hline
				Mariano Sciacco 			 & 0 & 0 & 0 & 3 & 2 & 0 & 5\\
				\hline
				\hline
				Alessandro Tommasin     & 0 & 2 & 0 & 3 & 0 & 0 & 5\\
				\hline
				\hline
				Giovanni Vidotto 			 & 0 & 0 & 0 & 0 & 4 & 1 & 5\\
				\hline 
				\textbf{Totale}			 		& 3 & 5 & 5 & 8 & 8 & 6 & 35\\
				\hline
				\caption{Tabella contenente il prospetto orario preventivato per il primo incremento}
			\end{longtable}
			\pagebreak
			
			La tabella può essere riassunta nel seguente istogramma:
			\begin{figure}[H]
				\centering
				\includegraphics[width=0.8\linewidth]{./images/preventivo/incremento1-1.png}
				\caption{Diagramma ore/ruolo componenti nel primo incremento}
				\label{fig:diagramma suddivione ruoli incremento I}
			\end{figure}
		
			\paragraph{Prospetto economico}
			In base al prospetto orario, quello economico sarà il seguente: 
			
			\rowcolors{2}{white}{lightest-grayest}
			\begin{longtable}{|l|c|c|c|c|c|c|c|}
				\hline
				\rowcolor{lighter-grayer}
				\textbf{Ruolo} & \textbf{Ore} & \textbf{Costo in € } \\
				\hline
				\endfirsthead
				
				\hline
				Responsabile 	    & 3 & 90,00\\
				\hline 
				\hline
				Amministratore	   & 5 & 100,00\\
				\hline
				\hline
				Analista 				& 5 & 125,00\\
				\hline
				\hline
				Progettista 		   & 8 & 176,00\\
				\hline
				\hline
				Programmatore 	  & 8 & 120,00\\
				\hline
				\hline
				Verificatore 		   & 6 & 90,00\\
				\hline
				\textbf{Totale} 	 & 35 & 701,00\\
				\hline
				\caption{Tabella contenente il prospetto economico in riferimento al prospetto orario}
			\end{longtable}
			\pagebreak
			
			La tabella può essere riassunta nel seguente areogramma:
			\begin{figure}[H]
				\centering
				\includegraphics[width=0.8\linewidth]{./images/preventivo/incremento1-2.png}
				\caption{Diagramma percentuale ore/ruolo del primo incremento}
				\label{fig:diagramma costi ruolo incremento I}
			\end{figure}
		
		
		
		\subsubsection{Incremento II}
			\paragraph{Prospetto orario}
			Durante il secondo incremento la distribuzione oraria preventivata dei ruoli di ogni componente del gruppo sarà la seguente:
			
			\rowcolors{2}{lightest-grayest}{white}
			\begin{longtable}{|l|c|c|c|c|c|c|c|}
				\hline
				\rowcolor{lighter-grayer}
				\textbf{Nome} & \textbf{Re} & \textbf{Am} & \textbf{An} & \textbf{Pg}  & \textbf{Pr}   & \textbf{Ve} & \textbf{Totale} \\
				\hline
				\endfirsthead
				
				\hline
				Giuseppe Vito Bitetti 		 & 1 & 0 & 0 & 3 & 0 & 2 & 6\\
				\hline
				\hline
				Lorenzo Dei Negri			 & 0 & 0 & 0 & 0 & 3 & 3 & 6\\
				\hline
				\hline
				Nicolò Frison				    & 3 & 0 & 0 & 0 & 3 & 0 & 6\\
				\hline
				\hline
				Fouad Mouad 				 & 0 & 3 & 3 & 0 & 0 & 0 & 6\\
				\hline
				\hline
				Mariano Sciacco 			 & 0 & 0 & 0 & 2 & 0 & 4 & 6\\
				\hline
				\hline
				Alessandro Tommasin    & 0 & 1 & 0 & 0 & 5 & 0 & 6\\
				\hline
				\hline
				Giovanni Vidotto 			 & 0 & 2 & 0 & 4 & 0 & 0 & 6\\
				\hline 
				\textbf{Totale}			 		& 4 & 6 & 3 & 9 & 11 & 9 & 42\\
				\hline
				\caption{Tabella contenente il prospetto orario preventivato per il secondo incremento}
			\end{longtable}
			\pagebreak
			
			La tabella può essere riassunta nel seguente istogramma:
			\begin{figure}[H]
				\centering
				\includegraphics[width=0.8\linewidth]{./images/preventivo/incremento2-1.png}
				\caption{Diagramma ore/ruolo componenti nel secondo incremento}
				\label{fig:diagramma suddivione ruoli incremento II}
			\end{figure}
			
			\paragraph{Prospetto economico}
			In base al prospetto orario, quello economico sarà il seguente: 
			
			\rowcolors{2}{white}{lightest-grayest}
			\begin{longtable}{|l|c|c|c|c|c|c|c|}
				\hline
				\rowcolor{lighter-grayer}
				\textbf{Ruolo} & \textbf{Ore} & \textbf{Costo in € } \\
				\hline
				\endfirsthead
				
				\hline
				Responsabile 	    & 4 & 120,00\\
				\hline 
				\hline
				Amministratore	   & 6 & 120,00\\
				\hline
				\hline
				Analista 				& 3 & 75,00\\
				\hline
				\hline
				Progettista 		   & 9 & 198,00\\
				\hline
				\hline
				Programmatore 	  & 11 & 165,00\\
				\hline
				\hline
				Verificatore 		   & 9 & 135,00\\
				\hline
				\textbf{Totale} 	 & 42 & 813,00\\
				\hline
				\caption{Tabella contenente il prospetto economico in riferimento al prospetto orario}
			\end{longtable}
			\pagebreak
			
			La tabella può essere riassunta nel seguente areogramma:
			\begin{figure}[H]
				\centering
				\includegraphics[width=0.8\linewidth]{./images/preventivo/incremento2-2.png}
				\caption{Diagramma percentuale ore/ruolo del secondo incremento}
				\label{fig:diagramma costi ruolo incremento II}
			\end{figure}
	
	
			
		\subsubsection{Incremento III}
			\paragraph{Prospetto orario}
			Durante il terzo incremento la distribuzione oraria preventivata dei ruoli di ogni componente del gruppo sarà la seguente:
			
			\rowcolors{2}{lightest-grayest}{white}
			\begin{longtable}{|l|c|c|c|c|c|c|c|}
				\hline
				\rowcolor{lighter-grayer}
				\textbf{Nome} & \textbf{Re} & \textbf{Am} & \textbf{An} & \textbf{Pg}  & \textbf{Pr}   & \textbf{Ve} & \textbf{Totale} \\
				\hline
				\endfirsthead
				
				\hline
				Giuseppe Vito Bitetti 		 & 0 & 0 & 0 & 0 & 3 & 3 & 6\\
				\hline
				\hline
				Lorenzo Dei Negri			 & 0 & 0 & 0 & 3 & 0 & 3 & 6\\
				\hline
				\hline
				Nicolò Frison				    & 3 & 0 & 0 & 2 & 0 & 1 & 6\\
				\hline
				\hline
				Fouad Mouad 				 & 0 & 0 & 0 & 3 & 3 & 0 & 6\\
				\hline
				\hline
				Mariano Sciacco 			 & 0 & 3 & 0 & 0 & 3 & 0 & 6\\
				\hline
				\hline
				Alessandro Tommasin     & 0 & 0 & 0 & 2 & 2 & 2 & 6\\
				\hline
				\hline
				Giovanni Vidotto 			 & 0 & 0 & 0 & 2 & 0 & 4 & 6\\
				\hline 
				\textbf{Totale}			 		& 3 & 3 & 0 & 12 & 11 & 13 & 42\\
				\hline
				\caption{Tabella contenente il prospetto orario preventivato per il terzo incremento}
			\end{longtable}
			\pagebreak
			
			La tabella può essere riassunta nel seguente istogramma:
			\begin{figure}[H]
				\centering
				\includegraphics[width=0.8\linewidth]{./images/preventivo/incremento3-1.png}
				\caption{Diagramma ore/ruolo componenti nel terzo incremento}
				\label{fig:diagramma suddivione ruoli incremento III}
			\end{figure}
			
			\paragraph{Prospetto economico}
			In base al prospetto orario, quello economico sarà il seguente: 
			
			\rowcolors{2}{white}{lightest-grayest}
			\begin{longtable}{|l|c|c|c|c|c|c|c|}
				\hline
				\rowcolor{lighter-grayer}
				\textbf{Ruolo} & \textbf{Ore} & \textbf{Costo in € } \\
				\hline
				\endfirsthead
				
				\hline
				Responsabile 	    & 3 & 90,00\\
				\hline 
				\hline
				Amministratore	   & 3 & 60,00\\
				\hline
				\hline
				Analista 				& - & -\\
				\hline
				\hline
				Progettista 		   & 12 & 264,00\\
				\hline
				\hline
				Programmatore 	  & 11 & 165,00\\
				\hline
				\hline
				Verificatore 		   & 13 & 195,00\\
				\hline
				\textbf{Totale} 	 & 42 & 774,00\\
				\hline
				\caption{Tabella contenente il prospetto economico in riferimento al prospetto orario}
			\end{longtable}
			\pagebreak
			
			La tabella può essere riassunta nel seguente areogramma:
			\begin{figure}[H]
				\centering
				\includegraphics[width=0.8\linewidth]{./images/preventivo/incremento3-2.png}
				\caption{Diagramma percentuale ore/ruolo del terzo incremento}
				\label{fig:diagramma costi ruolo incremento III}
			\end{figure}
			
			
		\subsubsection{Incremento IV}
			\paragraph{Prospetto orario}
			Durante il quarto incremento la distribuzione oraria preventivata dei ruoli di ogni componente del gruppo sarà la seguente:
			
			\rowcolors{2}{lightest-grayest}{white}
			\begin{longtable}{|l|c|c|c|c|c|c|c|}
				\hline
				\rowcolor{lighter-grayer}
				\textbf{Nome} & \textbf{Re} & \textbf{Am} & \textbf{An} & \textbf{Pg}  & \textbf{Pr}   & \textbf{Ve} & \textbf{Totale} \\
				\hline
				\endfirsthead
				
				\hline
				Giuseppe Vito Bitetti 		 & 2 & 0 & 0 & 0 & 4 & 0 & 6\\
				\hline
				\hline
				Lorenzo Dei Negri			 & 2 & 0 & 0 & 2 & 0 & 2 & 6\\
				\hline
				\hline
				Nicolò Frison				    & 0 & 0 & 0 & 2 & 4 & 0 & 6\\
				\hline
				\hline
				Fouad Mouad 				 & 0 & 3 & 0 & 0 & 0 & 3 & 6\\
				\hline
				\hline
				Mariano Sciacco 			 & 3 & 0 & 0 & 0 & 3 & 0 & 6\\
				\hline
				\hline
				Alessandro Tommasin     & 0 & 0 & 2 & 0 & 0 & 4 & 6\\
				\hline
				\hline
				Giovanni Vidotto 			 & 0 & 0 & 0 & 5 & 0 & 1 & 6\\
				\hline 
				\textbf{Totale}			 		& 7 & 3 & 2 & 9 & 11 & 10 & 42\\
				\hline
				\caption{Tabella contenente il prospetto orario preventivato per il quarto incremento}
			\end{longtable}
			\pagebreak
			
			La tabella può essere riassunta nel seguente istogramma:
			\begin{figure}[H]
				\centering
				\includegraphics[width=0.8\linewidth]{./images/preventivo/incremento4-1.png}
				\caption{Diagramma ore/ruolo componenti nel quarto  incremento}
				\label{fig:diagramma suddivione ruoli incremento IV}
			\end{figure}
			
			\paragraph{Prospetto economico}
			In base al prospetto orario, quello economico sarà il seguente: 
			
			\rowcolors{2}{white}{lightest-grayest}
			\begin{longtable}{|l|c|c|c|c|c|c|c|}
				\hline
				\rowcolor{lighter-grayer}
				\textbf{Ruolo} & \textbf{Ore} & \textbf{Costo in € } \\
				\hline
				\endfirsthead
				
				\hline
				Responsabile 	    & 7 & 210,00\\
				\hline 
				\hline
				Amministratore	   & 3 & 60,00\\
				\hline
				\hline
				Analista 				& 2 & 50,00\\
				\hline
				\hline
				Progettista 		   & 9 & 198,00\\
				\hline
				\hline
				Programmatore 	  & 11 & 165,00\\
				\hline
				\hline
				Verificatore 		   & 10 & 150,00\\
				\hline
				\textbf{Totale} 	 & 42 & 833,00\\
				\hline
				\caption{Tabella contenente il prospetto economico in riferimento al prospetto orario}
			\end{longtable}
			\pagebreak
			
			La tabella può essere riassunta nel seguente areogramma:
			\begin{figure}[H]
				\centering
				\includegraphics[width=0.8\linewidth]{./images/preventivo/incremento4-2.png}
				\caption{Diagramma percentuale ore/ruolo del quarto incremento}
				\label{fig:diagramma costi ruolo incremento IV}
			\end{figure}
		
		
		\subsubsection{Fase complessiva}
		\paragraph{Riepilogo prospetto orario}
		Durante i primi  quattro incrementi la distribuzione oraria preventivata dei ruoli di ogni componente del gruppo può essere riassunta nella seguente tabella:
		
		\rowcolors{2}{lightest-grayest}{white}
		\begin{longtable}{|l|c|c|c|c|c|c|c|}
			\hline
			\rowcolor{lighter-grayer}
			\textbf{Nome} & \textbf{Re} & \textbf{Am} & \textbf{An} & \textbf{Pg}  & \textbf{Pr}   & \textbf{Ve} & \textbf{Totale} \\
			\hline
			\endfirsthead
			
			\hline
			Giuseppe Vito Bitetti 		 & 3 & 0 & 3 & 3 & 9 & 5 & 23\\
			\hline
			\hline
			Lorenzo Dei Negri			 & 2 & 0 & 0 & 7 & 3 & 11 & 23\\
			\hline
			\hline
			Nicolò Frison				    & 6 & 3 & 0 & 4 & 7 & 3 & 23\\
			\hline
			\hline
			Fouad Mouad 				 & 3 & 6 & 5 & 3 & 3 & 3 & 23\\
			\hline
			\hline
			Mariano Sciacco 			 & 3 & 3 & 0 & 5 & 8 & 4 & 23\\
			\hline
			\hline
			Alessandro Tommasin     & 0 & 3 & 2 & 5 & 7 & 6 & 23\\
			\hline
			\hline
			Giovanni Vidotto 			 & 0 & 2 & 0 & 11 & 4 & 6 & 23\\
			\hline 
			\textbf{Totale}			 		& 17 & 17 & 10 & 38 & 41 & 38 & 161\\
			\hline
			\caption{Tabella contenente il prospetto orario preventivato per i primi quattro incrementi}
		\end{longtable}
		\pagebreak
		
		La tabella può essere riassunta nel seguente istogramma:
		\begin{figure}[H]
			\centering
			\includegraphics[width=0.8\linewidth]{./images/preventivo/incremento1-4-1.png}
			\caption{Diagramma ore/ruolo componenti nei primi quattro incrementi}
			\label{fig:diagramma suddivione ruoli incrementi I-IV}
		\end{figure}
		
		\paragraph{Riepilogo prospetto economico}
		In base al prospetto orario, quello economico sarà il seguente: 
		
		\rowcolors{2}{white}{lightest-grayest}
		\begin{longtable}{|l|c|c|}
			\hline
			\rowcolor{lighter-grayer}
			\textbf{Ruolo} & \textbf{Ore} & \textbf{Costo in € } \\
			\hline
			\endfirsthead
			
			\hline
			Responsabile 	    & 17 & 510,00\\
			\hline 
			\hline
			Amministratore	   & 17 & 340,00\\
			\hline
			\hline
			Analista 				& 10 & 250,00\\
			\hline
			\hline
			Progettista 		   & 38 & 836,00\\
			\hline
			\hline
			Programmatore 	  & 41 & 615,00\\
			\hline
			\hline
			Verificatore 		   & 38 & 570,00\\
			\hline
			\textbf{Totale} 	 & 161 & 3121,00\\
			\hline
			\caption{Tabella contenente il prospetto economico in riferimento al prospetto orario}
		\end{longtable}
		\pagebreak
		
		La tabella può essere riassunta nel seguente areogramma:
		\begin{figure}[H]
			\centering
			\includegraphics[width=0.8\linewidth]{./images/preventivo/incremento1-4-2.png}
			\caption{Diagramma percentuale ore/ruolo dei primi quattro incrementi}
			\label{fig:diagramma costi ruolo incrementi I-IV}
		\end{figure}
		
		
	%=============================================%
	\pagebreak
	\subsection{Progettazione completa dell'architettura e implementazione delle funzionalità}
	I successivi quattro incrementi, che vedremo in dettaglio, comporranno la fase stessa.
		\subsubsection{Incremento V}
			\paragraph{Prospetto orario}
			Durante il quinto incremento la distribuzione oraria preventivata dei ruoli di ogni componente del gruppo sarà la seguente:
			
			\rowcolors{2}{lightest-grayest}{white}
			\begin{longtable}{|l|c|c|c|c|c|c|c|}
				\hline
				\rowcolor{lighter-grayer}
				\textbf{Nome} & \textbf{Re} & \textbf{Am} & \textbf{An} & \textbf{Pg}  & \textbf{Pr}   & \textbf{Ve} & \textbf{Totale} \\
				\hline
				\endfirsthead
				
				\hline
				Giuseppe Vito Bitetti 		 & 0 & 2 & 2 & 0 & 0 & 2 & 6\\
				\hline
				\hline
				Lorenzo Dei Negri			 & 0 & 0 & 0 & 2 & 4 & 0 & 6\\
				\hline
				\hline
				Nicolò Frison				    & 2 & 0 & 0 & 0 & 0 & 4 & 6\\
				\hline
				\hline
				Fouad Mouad 				 & 2 & 0 & 0 & 0 & 4 & 0 & 6\\
				\hline
				\hline
				Mariano Sciacco 			 & 0 & 0 & 0 & 1 & 3 & 2 & 6\\
				\hline
				\hline
				Alessandro Tommasin     & 0 & 0 & 0 & 3 & 3 & 0 & 6\\
				\hline
				\hline
				Giovanni Vidotto 			 & 0 & 2 & 0 & 0 & 4 & 0 & 6\\
				\hline 
				\textbf{Totale}			 		& 4 & 4 & 2 & 6 & 18 & 8 & 42\\
				\hline
				\caption{Tabella contenente il prospetto orario preventivato per il quinto incremento}
			\end{longtable}
			
			La tabella può essere riassunta nel seguente istogramma:
			\begin{figure}[H]
				\centering
				\includegraphics[width=0.8\linewidth]{./images/preventivo/incremento5-1.png}
				\caption{Diagramma ore/ruolo componenti nel quinto incremento}
				\label{fig:diagramma suddivione ruoli incremento V}
			\end{figure}
			\pagebreak
			
			\paragraph{Prospetto economico}
			In base al prospetto orario, quello economico sarà il seguente: 
			
			\rowcolors{2}{white}{lightest-grayest}
			\begin{longtable}{|l|c|c|c|c|c|c|c|}
				\hline
				\rowcolor{lighter-grayer}
				\textbf{Ruolo} & \textbf{Ore} & \textbf{Costo in € } \\
				\hline
				\endfirsthead
				
				\hline
				Responsabile 	    & 4 & 120,00\\
				\hline 
				\hline
				Amministratore	   & 4 & 80,00\\
				\hline
				\hline
				Analista 				& 2 & 50,00\\
				\hline
				\hline
				Progettista 		   & 6 & 132,00\\
				\hline
				\hline
				Programmatore 	  & 18 & 270,00\\
				\hline
				\hline
				Verificatore 		   & 8 & 120,00\\
				\hline
				\textbf{Totale} 	 & 42 & 772,00\\
				\hline
				\caption{Tabella contenente il prospetto economico in riferimento al prospetto orario}
			\end{longtable}
			
			
			La tabella può essere riassunta nel seguente areogramma:
			\begin{figure}[H]
				\centering
				\includegraphics[width=0.8\linewidth]{./images/preventivo/incremento5-2.png}
				\caption{Diagramma percentuale ore/ruolo del quinto incremento}
				\label{fig:diagramma costi ruolo incremento V}
			\end{figure}
			\pagebreak
			
			
		\subsubsection{Incremento VI}
			\paragraph{Prospetto orario}
			Durante il sesto incremento la distribuzione oraria preventivata dei ruoli di ogni componente del gruppo sarà la seguente:
			
			\rowcolors{2}{lightest-grayest}{white}
			\begin{longtable}{|l|c|c|c|c|c|c|c|}
				\hline
				\rowcolor{lighter-grayer}
				\textbf{Nome} & \textbf{Re} & \textbf{Am} & \textbf{An} & \textbf{Pg}  & \textbf{Pr}   & \textbf{Ve} & \textbf{Totale} \\
				\hline
				\endfirsthead
				
				\hline
				Giuseppe Vito Bitetti 		 & 0 & 0 & 0 & 2 & 2 & 2 & 6\\
				\hline
				\hline
				Lorenzo Dei Negri			 & 0 & 0 & 0 & 0 & 4 & 2 & 6\\
				\hline
				\hline
				Nicolò Frison				      & 0 & 3 & 0 & 0 & 3 & 0 & 6\\
				\hline
				\hline
				Fouad Mouad 				   & 2 & 0 & 0 & 0 & 2 & 2 & 6\\
				\hline
				\hline
				Mariano Sciacco 			 & 0 & 0 & 0 & 0 & 2 & 4 & 6\\
				\hline
				\hline
				Alessandro Tommasin    & 0 & 0 & 0 & 0 & 2 & 4 & 6\\
				\hline
				\hline
				Giovanni Vidotto 			  & 0 & 0 & 0 & 4 & 2 & 0 & 6\\
				\hline 
				\textbf{Totale}			 		& 2 & 3 & 0 & 6 & 17 & 14 & 42\\
				\hline
				\caption{Tabella contenente il prospetto orario preventivato per il sesto incremento}
			\end{longtable}
			
			La tabella può essere riassunta nel seguente istogramma:
			\begin{figure}[H]
				\centering
				\includegraphics[width=0.8\linewidth]{./images/preventivo/incremento6-1.png}
				\caption{Diagramma ore/ruolo componenti nel sesto incremento}
				\label{fig:diagramma suddivione ruoli incremento VI}
			\end{figure}
		\pagebreak
			
			\paragraph{Prospetto economico}
			In base al prospetto orario, quello economico sarà il seguente: 
			
			\rowcolors{2}{white}{lightest-grayest}
			\begin{longtable}{|l|c|c|c|c|c|c|c|}
				\hline
				\rowcolor{lighter-grayer}
				\textbf{Ruolo} & \textbf{Ore} & \textbf{Costo in € } \\
				\hline
				\endfirsthead
				
				\hline
				Responsabile 	    & 3 & 90,00\\
				\hline 
				\hline
				Amministratore	   & 3 & 60,00\\
				\hline
				\hline
				Analista 				& 0 & 0,00\\
				\hline
				\hline
				Progettista 		   & 12 & 264,00\\
				\hline
				\hline
				Programmatore 	  & 11 & 165,00\\
				\hline
				\hline
				Verificatore 		   & 13 & 195,00\\
				\hline
				\textbf{Totale} 	 & 42 & 774,00\\
				\hline
				\caption{Tabella contenente il prospetto economico in riferimento al prospetto orario}
			\end{longtable}
			
			La tabella può essere riassunta nel seguente areogramma:
			\begin{figure}[H]
				\centering
				\includegraphics[width=0.8\linewidth]{./images/preventivo/incremento6-2.png}
				\caption{Diagramma percentuale ore/ruolo del sesto incremento}
				\label{fig:diagramma costi ruolo incremento VI}
			\end{figure}
		\pagebreak
			
			
		\subsubsection{Incremento VII}
			\paragraph{Prospetto orario}
			Durante il settimo incremento la distribuzione oraria preventivata dei ruoli di ogni componente del gruppo sarà la seguente:
			
			\rowcolors{2}{lightest-grayest}{white}
			\begin{longtable}{|l|c|c|c|c|c|c|c|}
				\hline
				\rowcolor{lighter-grayer}
				\textbf{Nome} & \textbf{Re} & \textbf{Am} & \textbf{An} & \textbf{Pg}  & \textbf{Pr}   & \textbf{Ve} & \textbf{Totale} \\
				\hline
				\endfirsthead
				
				\hline
				Giuseppe Vito Bitetti 		 & 0 & 0 & 0 & 0 & 4 & 2 & 6\\
				\hline
				\hline
				Lorenzo Dei Negri			 & 2 & 0 & 0 & 2 & 0 & 2 & 6\\
				\hline
				\hline
				Nicolò Frison				      & 0 & 0 & 0 & 3 & 0 & 3 & 6\\
				\hline
				\hline
				Fouad Mouad 				   & 0 & 0 & 0 & 2 & 0 & 4 & 6\\
				\hline
				\hline
				Mariano Sciacco 			 & 0 & 0 & 0 & 0 & 6 & 0 & 6\\
				\hline
				\hline
				Alessandro Tommasin    & 0 & 2 & 0 & 4 & 0 & 0 & 6\\
				\hline
				\hline
				Giovanni Vidotto 			  & 0 & 0 & 0 & 0 & 6 & 0 & 6\\
				\hline 
				\textbf{Totale}			 		& 2 & 2 & 0 & 11 & 16 & 11 & 42\\
				\hline
				\caption{Tabella contenente il prospetto orario preventivato per il settimo incremento}
			\end{longtable}
			
			La tabella può essere riassunta nel seguente istogramma:
			\begin{figure}[H]
				\centering
				\includegraphics[width=0.8\linewidth]{./images/preventivo/incremento7-1.png}
				\caption{Diagramma ore/ruolo componenti nel settimo incremento}
				\label{fig:diagramma suddivione ruoli incremento VII}
			\end{figure}
			\pagebreak
			
			\paragraph{Prospetto economico}
			In base al prospetto orario, quello economico sarà il seguente: 
			
			\rowcolors{2}{white}{lightest-grayest}
			\begin{longtable}{|l|c|c|}
				\hline
				\rowcolor{lighter-grayer}
				\textbf{Ruolo} & \textbf{Ore} & \textbf{Costo in € } \\
				\hline
				\endfirsthead
				
				\hline
				Responsabile 	    & 2 & 60,00\\
				\hline 
				\hline
				Amministratore	   & 2 & 40,00\\
				\hline
				\hline
				Analista 				 & 0 & 0,00\\
				\hline
				\hline
				Progettista 		   & 11 & 242,00\\
				\hline
				\hline
				Programmatore 	  & 16 & 240,00\\
				\hline
				\hline
				Verificatore 		   & 11 & 165,00\\
				\hline
				\textbf{Totale} 	 & 42 & 747,00\\
				\hline
				\caption{Tabella contenente il prospetto economico in riferimento al prospetto orario}
			\end{longtable}

			La tabella può essere riassunta nel seguente areogramma:
			\begin{figure}[H]
				\centering
				\includegraphics[width=0.8\linewidth]{./images/preventivo/incremento7-2.png}
				\caption{Diagramma percentuale ore/ruolo del settimo incremento}
				\label{fig:diagramma costi ruolo incremento VII}
			\end{figure}
			\pagebreak
			
			
		\subsubsection{Incremento VIII}
			\paragraph{Prospetto orario}
			Durante l'ottavo incremento la distribuzione oraria preventivata dei ruoli di ogni componente del gruppo sarà la seguente:
			
			\rowcolors{2}{lightest-grayest}{white}
			\begin{longtable}{|l|c|c|c|c|c|c|c|}
				\hline
				\rowcolor{lighter-grayer}
				\textbf{Nome} & \textbf{Re} & \textbf{Am} & \textbf{An} & \textbf{Pg}  & \textbf{Pr}   & \textbf{Ve} & \textbf{Totale} \\
				\hline
				\endfirsthead
				
				\hline
				Giuseppe Vito Bitetti 		 & 2 & 0 & 0 & 2 & 0 & 2 & 6\\
				\hline
				\hline
				Lorenzo Dei Negri			 & 0 & 0 & 0 & 2 & 4 & 0 & 6\\
				\hline
				\hline
				Nicolò Frison				      & 0 & 0 & 0 & 2 & 2 & 2 & 6\\
				\hline
				\hline
				Fouad Mouad 				   & 0 & 0 & 0 & 2 & 2 & 2 & 6\\
				\hline
				\hline
				Mariano Sciacco 			 & 0 & 0 & 0 & 1 & 0 & 5 & 6\\
				\hline
				\hline
				Alessandro Tommasin    & 0 & 0 & 0 & 3 & 0 & 3 & 6\\
				\hline
				\hline
				Giovanni Vidotto 			  & 0 & 2 & 0 & 2 & 0 & 2 & 6\\
				\hline 
				\textbf{Totale}			 		 & 2 & 2 & 0 & 14 & 8 & 16 & 42\\
				\hline
				\caption{Tabella contenente il prospetto orario preventivato per l'ottavo incremento}
			\end{longtable}
			
			La tabella può essere riassunta nel seguente istogramma:
			\begin{figure}[H]
				\centering
				\includegraphics[width=0.8\linewidth]{./images/preventivo/incremento8-1.png}
				\caption{Diagramma ore/ruolo componenti nell'ottavo incremento}
				\label{fig:diagramma suddivione ruoli incremento VIII}
			\end{figure}
			\pagebreak
			
			\paragraph{Prospetto economico}
			In base al prospetto orario, quello economico sarà il seguente: 
			
			\rowcolors{2}{white}{lightest-grayest}
			\begin{longtable}{|l|c|c|}
				\hline
				\rowcolor{lighter-grayer}
				\textbf{Ruolo} & \textbf{Ore} & \textbf{Costo in € } \\
				\hline
				\endfirsthead
				
				\hline
				Responsabile 	    & 2 & 60,00\\
				\hline 
				\hline
				Amministratore	   & 2 & 40,00\\
				\hline
				\hline
				Analista 				& 0 & 0,00\\
				\hline
				\hline
				Progettista 		   & 14 & 308,00\\
				\hline
				\hline
				Programmatore 	  & 8 & 120,00\\
				\hline
				\hline
				Verificatore 		   & 16 & 240,00\\
				\hline
				\textbf{Totale} 	 & 42 & 768,00\\
				\hline
				\caption{Tabella contenente il prospetto economico in riferimento al prospetto orario}
			\end{longtable}
			
			La tabella può essere riassunta nel seguente areogramma:
			\begin{figure}[H]
				\centering
				\includegraphics[width=0.8\linewidth]{./images/preventivo/incremento8-2.png}
				\caption{Diagramma percentuale ore/ruolo dell'ottavo incremento}
				\label{fig:diagramma costi ruolo incremento VIII}
			\end{figure}
		\pagebreak
		
		
	\subsubsection{Fase complessiva}
	\paragraph{Riepilogo prospetto orario}
	Dal quinto all'ottavo incremento la distribuzione oraria preventivata dei ruoli di ogni componente del gruppo può essere riassunta nella seguente tabella:
	
	\rowcolors{2}{lightest-grayest}{white}
	\begin{longtable}{|l|c|c|c|c|c|c|c|}
		\hline
		\rowcolor{lighter-grayer}
		\textbf{Nome} & \textbf{Re} & \textbf{Am} & \textbf{An} & \textbf{Pg}  & \textbf{Pr}   & \textbf{Ve} & \textbf{Totale} \\
		\hline
		\endfirsthead
		
		\hline
		Giuseppe Vito Bitetti 		 & 2 & 2 & 2 & 4 & 6 & 8 & 24\\
		\hline
		\hline
		Lorenzo Dei Negri			 & 2 & 0 & 0 & 6 & 12 & 4 & 24\\
		\hline
		\hline
		Nicolò Frison				    & 2 & 3 & 0 & 5 & 5 & 9 & 24\\
		\hline
		\hline
		Fouad Mouad 				 & 4 & 0 & 0 & 4 & 8 & 8 & 24\\
		\hline
		\hline
		Mariano Sciacco 			 & 0 & 0 & 0 & 2 & 11 & 11 & 24\\
		\hline
		\hline
		Alessandro Tommasin     & 0 & 2 & 0 & 10 & 5 & 7 & 24\\
		\hline
		\hline
		Giovanni Vidotto 			 & 0 & 4 & 0 & 6 & 12 & 2 & 24\\
		\hline 
		\textbf{Totale}			 		& 10 & 11 & 2 & 37 & 59 & 49 & 168\\
		\hline
		\caption{Tabella contenente il prospetto orario preventivato dal quinto all'ottavo incremento}
	\end{longtable}
	
	La tabella può essere riassunta nel seguente istogramma:
	\begin{figure}[H]
		\centering
		\includegraphics[width=0.8\linewidth]{./images/preventivo/incremento5-8-1.png}
		\caption{Diagramma ore/ruolo componenti dal quinto all'ottavo incremento}
		\label{fig:diagramma suddivione ruoli incrementi V-VIII}
	\end{figure}
	\pagebreak
	
	\paragraph{Riepilogo prospetto economico}
	In base al prospetto orario, quello economico sarà il seguente: 
	
	\rowcolors{2}{white}{lightest-grayest}
	\begin{longtable}{|l|c|c|}
		\hline
		\rowcolor{lighter-grayer}
		\textbf{Ruolo} & \textbf{Ore} & \textbf{Costo in € } \\
		\hline
		\endfirsthead
		
		\hline
		Responsabile 	    & 10 & 300,00\\
		\hline 
		\hline
		Amministratore	   & 11 & 220,00\\
		\hline
		\hline
		Analista 				& 2 & 50,00\\
		\hline
		\hline
		Progettista 		   & 37 & 814,00\\
		\hline
		\hline
		Programmatore 	  & 59 & 885,00\\
		\hline
		\hline
		Verificatore 		   & 49 & 735,00\\
		\hline
		\textbf{Totale} 	 & 168 & 3004,00\\
		\hline
		\caption{Tabella contenente il prospetto economico in riferimento al prospetto orario}
	\end{longtable}
	
	La tabella può essere riassunta nel seguente areogramma:
	\begin{figure}[H]
		\centering
		\includegraphics[width=0.8\linewidth]{./images/preventivo/incremento5-8-2.png}
		\caption{Diagramma percentuale ore/ruolo dal quinto all'ottavo incremento}
		\label{fig:diagramma costi ruolo incrementi V-VIII}
	\end{figure}	
		
		
	%=======================================================%	
	\pagebreak
	\subsection{Completamento dell'implementazione e raffinamento delle funzionalità}
	I successivi quattro incrementi, che vedremo in dettaglio, comporranno la fase stessa.  		
	\subsubsection{Incremento IX}
		\paragraph{Prospetto orario}
		Nel nono incremento la distribuzione oraria preventivata dei ruoli dei componenti del gruppo sarà:
		
		\rowcolors{2}{lightest-grayest}{white}
		\begin{longtable}{|l|c|c|c|c|c|c|c|}
			\hline
			\rowcolor{lighter-grayer}
			\textbf{Nome} & \textbf{Re} & \textbf{Am} & \textbf{An} & \textbf{Pg}  & \textbf{Pr}   & \textbf{Ve} & \textbf{Totale} \\
			\hline
			\endfirsthead
			
			\hline
			Giuseppe Vito Bitetti 		 & 0 & 0 & 0 & 0 & 4 & 2 & 6\\
			\hline
			\hline
			Lorenzo Dei Negri			 & 0 & 0 & 0 & 2 & 0 & 4 & 6\\
			\hline
			\hline
			Nicolò Frison				      & 0 & 2 & 0 & 0 & 3 & 1 & 6\\
			\hline
			\hline
			Fouad Mouad 				   & 0 & 0 & 0 & 5 & 1 & 0 & 6\\
			\hline
			\hline
			Mariano Sciacco 			 & 0 & 0 & 0 & 2 & 4 & 0 & 6\\
			\hline
			\hline
			Alessandro Tommasin    & 0 & 0 & 0 & 2 & 4 & 0 & 6\\
			\hline
			\hline
			Giovanni Vidotto 			  & 4 & 0 & 0 & 0 & 2 & 0 & 6\\
			\hline 
			\textbf{Totale}			 		& 4 & 2 & 0 & 11 & 18 & 7 & 42\\
			\hline
			\caption{Tabella contenente il prospetto orario preventivato per il nono incremento}
		\end{longtable}
		
		La tabella può essere riassunta nel seguente istogramma:
		\begin{figure}[H]
			\centering
			\includegraphics[width=0.75\linewidth]{./images/preventivo/incremento9-1.png}
			\caption{Diagramma ore/ruolo componenti nel nono incremento}
			\label{fig:diagramma suddivione ruoli incremento IX}
		\end{figure}
%		\pagebreak
		
		\paragraph{Prospetto economico}
		In base al prospetto orario, quello economico sarà il seguente: 
		
		\rowcolors{2}{white}{lightest-grayest}
		\begin{longtable}{|l|c|c|}
			\hline
			\rowcolor{lighter-grayer}
			\textbf{Ruolo} & \textbf{Ore} & \textbf{Costo in € } \\
			\hline
			\endfirsthead
			
			\hline
			Responsabile 	    & 4 & 120,00\\
			\hline 
			\hline
			Amministratore	   & 2 & 40,00\\
			\hline
			\hline
			Analista 				& 0 & 0,00\\
			\hline
			\hline
			Progettista 		   & 11 & 242,00\\
			\hline
			\hline
			Programmatore 	  & 18 & 270,00\\
			\hline
			\hline
			Verificatore 		   & 7 & 105,00\\
			\hline
			\textbf{Totale} 	 & 42 & 777,00\\
			\hline
			\caption{Tabella contenente il prospetto economico in riferimento al prospetto orario}
		\end{longtable}
		
		La tabella può essere riassunta nel seguente areogramma:
		\begin{figure}[H]
			\centering
			\includegraphics[width=0.8\linewidth]{./images/preventivo/incremento9-2.png}
			\caption{Diagramma percentuale ore/ruolo del nono incremento}
			\label{fig:diagramma costi ruolo incremento IX}
		\end{figure}
		\pagebreak
		
		
	\subsubsection{Incremento X}
		\paragraph{Prospetto orario}
		Durante il decimo incremento la distribuzione oraria preventivata dei ruoli di ogni componente del gruppo sarà la seguente:
		
		\rowcolors{2}{lightest-grayest}{white}
		\begin{longtable}{|l|c|c|c|c|c|c|c|}
			\hline
			\rowcolor{lighter-grayer}
			\textbf{Nome} & \textbf{Re} & \textbf{Am} & \textbf{An} & \textbf{Pg}  & \textbf{Pr}   & \textbf{Ve} & \textbf{Totale} \\
			\hline
			\endfirsthead
			
			\hline
			Giuseppe Vito Bitetti 		 & 0 & 0 & 0 & 3 & 0 & 3 & 6\\
			\hline
			\hline
			Lorenzo Dei Negri			 & 0 & 2 & 0 & 4 & 0 & 0 & 6\\
			\hline
			\hline
			Nicolò Frison				      & 0 & 0 & 0 & 3 & 0 & 3 & 6\\
			\hline
			\hline
			Fouad Mouad 				   & 0 & 0 & 0 & 0 & 4 & 2 & 6\\
			\hline
			\hline
			Mariano Sciacco 			 & 0 & 0 & 0 & 2 & 0 & 4 & 6\\
			\hline
			\hline
			Alessandro Tommasin    & 2 & 0 & 0 & 2 & 0 & 2 & 6\\
			\hline
			\hline
			Giovanni Vidotto 			  & 0 & 0 & 0 & 2 & 2 & 2 & 6\\
			\hline 
			\textbf{Totale}			 		& 2 & 2 & 0 & 16 & 6 & 16 & 42\\
			\hline
			\caption{Tabella contenente il prospetto orario preventivato per il decimo incremento}
		\end{longtable}
		
		La tabella può essere riassunta nel seguente istogramma:
		\begin{figure}[H]
			\centering
			\includegraphics[width=0.8\linewidth]{./images/preventivo/incremento10-1.png}
			\caption{Diagramma ore/ruolo componenti nel decimo incremento}
			\label{fig:diagramma suddivione ruoli incremento X}
		\end{figure}
		\pagebreak
		
		\paragraph{Prospetto economico}
		In base al prospetto orario, quello economico sarà il seguente: 
		
		\rowcolors{2}{white}{lightest-grayest}
		\begin{longtable}{|l|c|c|c|c|c|c|c|}
			\hline
			\rowcolor{lighter-grayer}
			\textbf{Ruolo} & \textbf{Ore} & \textbf{Costo in € } \\
			\hline
			\endfirsthead
			
			\hline
			Responsabile 	    & 2 & 60,00\\
			\hline 
			\hline
			Amministratore	   & 2 & 40,00\\
			\hline
			\hline
			Analista 				& 0 & 0,00\\
			\hline
			\hline
			Progettista 		   & 16 & 352,00\\
			\hline
			\hline
			Programmatore 	  & 6 & 90,00\\
			\hline
			\hline
			Verificatore 		   & 16 & 240,00\\
			\hline
			\textbf{Totale} 	 & 42 & 782,00\\
			\hline
			\caption{Tabella contenente il prospetto economico in riferimento al prospetto orario}
		\end{longtable}
		
		La tabella può essere riassunta nel seguente areogramma:
		\begin{figure}[H]
			\centering
			\includegraphics[width=0.8\linewidth]{./images/preventivo/incremento10-2.png}
			\caption{Diagramma percentuale ore/ruolo del decimo incremento}
			\label{fig:diagramma costi ruolo incremento X}
		\end{figure}
		\pagebreak
		
		
	\subsubsection{Incremento XI}
		\paragraph{Prospetto orario}
		Durante l'undicesimo incremento la distribuzione oraria preventivata dei ruoli di ogni componente del gruppo sarà la seguente:
		
		\rowcolors{2}{lightest-grayest}{white}
		\begin{longtable}{|l|c|c|c|c|c|c|c|}
			\hline
			\rowcolor{lighter-grayer}
			\textbf{Nome} & \textbf{Re} & \textbf{Am} & \textbf{An} & \textbf{Pg}  & \textbf{Pr}   & \textbf{Ve} & \textbf{Totale} \\
			\hline
			\endfirsthead
			
			\hline
			Giuseppe Vito Bitetti 		 & 0 & 0 & 0 & 0 & 3 & 3 & 6\\
			\hline
			\hline
			Lorenzo Dei Negri			 & 0 & 2 & 0 & 0 & 3 & 1 & 6\\
			\hline
			\hline
			Nicolò Frison				      & 0 & 0 & 0 & 0 & 3 & 3 & 6\\
			\hline
			\hline
			Fouad Mouad 				   & 0 & 0 & 0 & 5 & 0 & 1 & 6\\
			\hline
			\hline
			Mariano Sciacco 			 & 0 & 0 & 0 & 2 & 4 & 0 & 6\\
			\hline
			\hline
			Alessandro Tommasin    & 2 & 0 & 0 & 2 & 0 & 2 & 6\\
			\hline
			\hline
			Giovanni Vidotto 			  & 0 & 0 & 0 & 0 & 4 & 2 & 6\\
			\hline 
			\textbf{Totale}			 		& 2 & 2 & 0 & 9 & 17 & 12 & 42\\
			\hline
			\caption{Tabella contenente il prospetto orario preventivato per l'undicesimo incremento}
		\end{longtable}
		
		La tabella può essere riassunta nel seguente istogramma:
		\begin{figure}[H]
			\centering
			\includegraphics[width=0.8\linewidth]{./images/preventivo/incremento11-1.png}
			\caption{Diagramma ore/ruolo componenti nell'undicesimo incremento}
			\label{fig:diagramma suddivione ruoli incremento XI}
		\end{figure}
		\pagebreak
		
		\paragraph{Prospetto economico}
		In base al prospetto orario, quello economico sarà il seguente: 
		
		\rowcolors{2}{white}{lightest-grayest}
		\begin{longtable}{|l|c|c|c|c|c|c|c|}
			\hline
			\rowcolor{lighter-grayer}
			\textbf{Ruolo} & \textbf{Ore} & \textbf{Costo in € } \\
			\hline
			\endfirsthead
			
			\hline
			Responsabile 	    & 2 & 60,00\\
			\hline 
			\hline
			Amministratore	   & 2 & 40,00\\
			\hline
			\hline
			Analista 				& 0 & 0,00\\
			\hline
			\hline
			Progettista 		   & 9 & 198,00\\
			\hline
			\hline
			Programmatore 	  & 17 & 255,00\\
			\hline
			\hline
			Verificatore 		   & 12 & 180,00\\
			\hline
			\textbf{Totale} 	 & 42 & 733,00\\
			\hline
			\caption{Tabella contenente il prospetto economico in riferimento al prospetto orario}
		\end{longtable}
		
		La tabella può essere riassunta nel seguente areogramma:
		\begin{figure}[H]
			\centering
			\includegraphics[width=0.8\linewidth]{./images/preventivo/incremento11-2.png}
			\caption{Diagramma percentuale ore/ruolo dell'undicesimo incremento}
			\label{fig:diagramma costi ruolo incremento XI }
		\end{figure}
		\pagebreak
		
		
	\subsubsection{Incremento XII}
		\paragraph{Prospetto orario}
		Durante il dodicesimi incremento la distribuzione oraria preventivata dei ruoli di ogni componente del gruppo sarà la seguente:
		
		\rowcolors{2}{lightest-grayest}{white}
		\begin{longtable}{|l|c|c|c|c|c|c|c|}
			\hline
			\rowcolor{lighter-grayer}
			\textbf{Nome} & \textbf{Re} & \textbf{Am} & \textbf{An} & \textbf{Pg}  & \textbf{Pr}   & \textbf{Ve} & \textbf{Totale} \\
			\hline
			\endfirsthead
			
			\hline
			Giuseppe Vito Bitetti 		 & 0 & 0 & 0 & 4 & 0 & 6 & 10\\
			\hline
			\hline
			Lorenzo Dei Negri			 & 1 & 0 & 0 & 3 & 0 & 6 & 10\\
			\hline
			\hline
			Nicolò Frison				      & 0 & 0 & 0 & 4 & 3 & 3 & 10\\
			\hline
			\hline
			Fouad Mouad 				   & 0 & 0 & 0 & 2 & 5 & 3 & 10\\
			\hline
			\hline
			Mariano Sciacco 			 & 0 & 1 & 0 & 3 & 2 & 4 & 10\\
			\hline
			\hline
			Alessandro Tommasin    & 0 & 0 & 0 & 3 & 2 & 5 & 10\\
			\hline
			\hline
			Giovanni Vidotto 			  & 0 & 0 & 0 & 3 & 3 & 4 & 10\\
			\hline 
			\textbf{Totale}			 		& 1 & 1 & 0 & 22 & 15 & 31 & 70\\
			\hline
			\caption{Tabella contenente il prospetto orario preventivato per il dodicesimo incremento}
		\end{longtable}
		
		La tabella può essere riassunta nel seguente istogramma:
		\begin{figure}[H]
			\centering
			\includegraphics[width=0.8\linewidth]{./images/preventivo/incremento12-1.png}
			\caption{Diagramma ore/ruolo componenti nel dodicesimo incremento}
			\label{fig:diagramma suddivione ruoli incremento XII}
		\end{figure}
		\pagebreak
		
		\paragraph{Prospetto economico}
		In base al prospetto orario, quello economico sarà il seguente: 
		
		\rowcolors{2}{white}{lightest-grayest}
		\begin{longtable}{|l|c|c|c|c|c|c|c|}
			\hline
			\rowcolor{lighter-grayer}
			\textbf{Ruolo} & \textbf{Ore} & \textbf{Costo in € } \\
			\hline
			\endfirsthead
			
			\hline
			Responsabile 	    & 1 & 30,00\\
			\hline 
			\hline
			Amministratore	   & 1 & 20,00\\
			\hline
			\hline
			Analista 				& 0 & 0,00\\
			\hline
			\hline
			Progettista 		   & 22 & 484,00\\
			\hline
			\hline
			Programmatore 	  & 15 & 225,00\\
			\hline
			\hline
			Verificatore 		   & 31 & 465,00\\
			\hline
			\textbf{Totale} 	 & 70 & 1.224,00\\
			\hline
			\caption{Tabella contenente il prospetto economico in riferimento al prospetto orario}
		\end{longtable}
		
		La tabella può essere riassunta nel seguente areogramma:
		\begin{figure}[H]
			\centering
			\includegraphics[width=0.8\linewidth]{./images/preventivo/incremento12-2.png}
			\caption{Diagramma percentuale ore/ruolo del dodicesimo incremento}
			\label{fig:diagramma costi ruolo incremento XII}
		\end{figure}
		\pagebreak
	
	\subsubsection{Fase complessiva}
	\paragraph{Riepilogo prospetto orario}
	Durante gli ultimi quattro incrementi la distribuzione oraria preventivata dei ruoli di ogni componente del gruppo può essere riassunta nella seguente tabella:
	
	\rowcolors{2}{lightest-grayest}{white}
	\begin{longtable}{|l|c|c|c|c|c|c|c|}
		\hline
		\rowcolor{lighter-grayer}
		\textbf{Nome} & \textbf{Re} & \textbf{Am} & \textbf{An} & \textbf{Pg}  & \textbf{Pr}   & \textbf{Ve} & \textbf{Totale} \\
		\hline
		\endfirsthead
		
		\hline
		Giuseppe Vito Bitetti 		 & 0 & 0 & 0 & 7 & 7 & 14 & 28\\
		\hline
		\hline
		Lorenzo Dei Negri			 & 1 & 4 & 0 & 9 & 3 & 11 & 28\\
		\hline
		\hline
		Nicolò Frison				    & 0 & 2 & 0 & 7 & 9 & 10 & 28\\
		\hline
		\hline
		Fouad Mouad 				 & 0 & 0 & 0 & 12 & 10 & 6 & 28\\
		\hline
		\hline
		Mariano Sciacco 			 & 0 & 1 & 0 & 9 & 10 & 6 & 28\\
		\hline
		\hline
		Alessandro Tommasin     & 4 & 0 & 0 & 9 & 6 & 9 & 28\\
		\hline
		\hline
		Giovanni Vidotto 			 & 4 & 0 & 0 & 5 & 11 & 8 & 28\\
		\hline 
		\textbf{Totale}			 		& 9 & 7 & 0 & 58 & 56 & 66 & 196\\
		\hline
		\caption{Tabella contenente il prospetto orario preventivato per gli ultimi quattro incrementi}
	\end{longtable}
	
	La tabella può essere riassunta nel seguente istogramma:
	\begin{figure}[H]
		\centering
		\includegraphics[width=0.8\linewidth]{./images/preventivo/incremento9-12-1.png}
		\caption{Diagramma ore/ruolo componenti negli ultimi quattro incrementi}
		\label{fig:diagramma suddivione ruoli incrementi IX-XII}
	\end{figure}
	\pagebreak
	
	\paragraph{Riepilogo prospetto economico}
	In base al prospetto orario, quello economico sarà il seguente: 
	
	\rowcolors{2}{white}{lightest-grayest}
	\begin{longtable}{|l|c|c|}
		\hline
		\rowcolor{lighter-grayer}
		\textbf{Ruolo} & \textbf{Ore} & \textbf{Costo in € } \\
		\hline
		\endfirsthead
		
		\hline
		Responsabile 	    & 9 & 270,00\\
		\hline 
		\hline
		Amministratore	   & 7 & 140,00\\
		\hline
		\hline
		Analista 				& - & -\\
		\hline
		\hline
		Progettista 		   & 58 & 1276,00\\
		\hline
		\hline
		Programmatore 	  & 56 & 840,00\\
		\hline
		\hline
		Verificatore 		   & 66 & 990,00\\
		\hline
		\textbf{Totale} 	 & 196 & 3516,00\\
		\hline
		\caption{Tabella contenente il prospetto economico in riferimento al prospetto orario}
	\end{longtable}
	
	La tabella può essere riassunta nel seguente areogramma:
	\begin{figure}[H]
		\centering
		\includegraphics[width=0.8\linewidth]{./images/preventivo/incremento9-12-2.png}
		\caption{Diagramma percentuale ore/ruolo degli ultimi quattro incrementi}
		\label{fig:diagramma costi ruolo incrementi IX-XII}
	\end{figure}
	\pagebreak
	
	%====================================%
	
	\subsection{Fase di validazione e collaudo}
		\subsubsection{Prospetto orario}
		Durante questa fase la distribuzione oraria preventivata dei ruoli di ogni componente del gruppo sarà la seguente:
		
		\rowcolors{2}{lightest-grayest}{white}
		\begin{longtable}{|l|c|c|c|c|c|c|c|}
			\hline
			\rowcolor{lighter-grayer}
			\textbf{Nome} & \textbf{Re} & \textbf{Am} & \textbf{An} & \textbf{Pg}  & \textbf{Pr}   & \textbf{Ve} & \textbf{Totale} \\
			\hline
			\endfirsthead
			
			\hline
			Giuseppe Vito Bitetti 		 & 0 & 3 & 2 & 0 & 4 & 6 & 15\\
			\hline
			\hline
			Lorenzo Dei Negri			 & 5 & 0 & 0 & 3 & 5 & 2 & 15\\
			\hline
			\hline
			Nicolò Frison				      & 0 & 4 & 4 & 1 & 2 & 4 & 15\\
			\hline
			\hline
			Fouad Mouad 				   & 5 & 0 & 2 & 0 & 0 & 8 & 15\\
			\hline
			\hline
			Mariano Sciacco 			 & 0 & 6 & 2 & 2 & 5 & 0 & 15\\
			\hline
			\hline
			Alessandro Tommasin    & 5 & 0 & 2 & 4 & 0 & 3 & 15\\
			\hline
			\hline
			Giovanni Vidotto 			  & 0 & 6 & 3 & 2 & 2 & 2 & 15\\
			\hline 
			\textbf{Totale}			 	& 15 & 19 & 16 & 12 & 18 & 25 & 105\\
			\hline
			\caption{Tabella contenente il prospetto orario preventivato per la fase di validazione e collaudo}
		\end{longtable}
		
		La tabella può essere riassunta nel seguente istogramma:
		\begin{figure}[H]
			\centering
			\includegraphics[width=0.8\linewidth]{./images/preventivo/validColl1.png}
			\caption{Diagramma ore/ruolo componenti nella fase di validazione e collaudo}
			\label{fig:diagramma suddivione ruoli fase di validazione e collaudo}
		\end{figure}
		\pagebreak
		
		\subsubsection{Prospetto economico}
		In base al prospetto orario, quello economico sarà il seguente: 
		
		\rowcolors{2}{white}{lightest-grayest}
		\begin{longtable}{|l|c|c|c|c|c|c|c|}
			\hline
			\rowcolor{lighter-grayer}
			\textbf{Ruolo} & \textbf{Ore} & \textbf{Costo in € } \\
			\hline
			\endfirsthead
			
			\hline
			Responsabile 	    & 15 & 450,00\\
			\hline 
			\hline
			Amministratore	   & 19 & 380,00\\
			\hline
			\hline
			Analista 				& 16 & 400,00\\
			\hline
			\hline
			Progettista 		   & 12 & 264,00\\
			\hline
			\hline
			Programmatore 	  & 18 & 270,00\\
			\hline
			\hline
			Verificatore 		   & 25 & 375,00\\
			\hline
			\textbf{Totale} 	 & 105 & 2.139,00\\
			\hline
			\caption{Tabella contenente il prospetto economico in riferimento al prospetto orario}
		\end{longtable}
		
		La tabella può essere riassunta nel seguente areogramma:
		\begin{figure}[H]
			\centering
			\includegraphics[width=0.8\linewidth]{./images/preventivo/validColl2.png}
			\caption{Diagramma percentuale ore/ruolo della fase di validazione e collaudo}
			\label{fig:diagramma costi ruolo fase di validazione e collaudo}
		\end{figure}	
	\pagebreak	
		

	\subsection{Riepilogo ore totali}
		\subsubsection{Totale ore complessive}
			\paragraph{Totale prospetto orario complessivo}
			Riepilogo della distribuzione oraria di tutte le fasi:
			\rowcolors{2}{lightest-grayest}{white}
			\begin{longtable}{|l|c|c|c|c|c|c|c|}
				\hline
				\rowcolor{lighter-grayer}
				\textbf{Nome} & \textbf{Re} & \textbf{Am} & \textbf{An} & \textbf{Pg}  & \textbf{Pr}   & \textbf{Ve} & \textbf{Totale} \\
				\hline
				\endfirsthead
				
				\hline
				Giuseppe Vito Bitetti 		& 8 & 14 & 28 & 14 & 26 & 48 & 138\\
				\hline
				\hline
				Lorenzo Dei Negri			& 18 & 14 & 14 & 30 & 23 & 39 & 138\\
				\hline
				\hline
				Nicolò Frison				    & 8 & 25 & 17 & 22 & 23 & 43 & 138\\
				\hline
				\hline
				Fouad Mouad 				 & 14 & 17 & 21 & 21 & 21 & 44 & 138\\
				\hline
				\hline
				Mariano Sciacco 			& 14 & 10 & 18 & 27 & 34 & 35 & 138\\
				\hline
				\hline
				Alessandro Tommasin    & 20 & 9 & 25 & 28 & 18 & 38 & 138\\
				\hline
				\hline
				Giovanni Vidotto 			 & 8 & 19 & 16 & 26 & 29 & 40 & 138\\
				\hline 
				\textbf{Totale}				 & 90 & 108 & 139 & 168 & 174 & 287 & 966\\
				\hline
				\caption{Tabella contenente un prospetto orario riepilogativo di tutti i periodi trattati in precedenza}
			\end{longtable}
			
			La tabella può essere riassunta nel seguente istogramma:
			\begin{figure}[H]
				\centering
				\includegraphics[width=0.8\linewidth]{./images/preventivo/totOre1.png}
				\caption{Diagramma ore/ruolo componenti nel totale delle ore}
				\label{fig:diagramma suddivione ruoli totale ore}
			\end{figure}
			\pagebreak
			
			\paragraph{Totale prospetto economico complessivo}
			In base al prospetto orario, quello economico sarà il seguente: 
			
			\rowcolors{2}{white}{lightest-grayest}
			\begin{longtable}{|l|c|c|c|c|c|c|c|}
				\hline
				\rowcolor{lighter-grayer}
				\textbf{Ruolo} & \textbf{Ore} & \textbf{Costo in € } \\
				\hline
				\endfirsthead
				
				\hline
				Responsabile 	    & 90 & 2.700,00\\
				\hline 
				\hline
				Amministratore	  & 108 & 2.160,00\\
				\hline
				\hline
				Analista 				& 139 & 3.475,00\\
				\hline
				\hline
				Progettista 		  & 168 & 3.696,00\\
				\hline
				\hline
				Programmatore 	 & 174 & 2.610,00\\
				\hline
				\hline
				Verificatore 		  & 287 & 4.305,00\\
				\hline
				\textbf{Totale} 	& 966 & 18.946,00\\
				\hline
				\caption{Tabella contenente il prospetto economico in riferimento al prospetto orario}
			\end{longtable}
			
			La tabella può essere riassunta nel seguente areogramma:
			\begin{figure}[H]
				\centering
				\includegraphics[width=0.8\linewidth]{./images/preventivo/totOre2.png}
				\caption{Diagramma percentuale ore/ruolo nel totale delle ore}
				\label{fig:diagramma costi ruolo fase totale ore}
			\end{figure}
			\pagebreak
		
			\subsubsection{Totale ore rendicontate}
				\paragraph{Totale prospetto orario rendicontato}
				Riepilogo della distribuzione oraria delle fasi a carico del committente, escludendo quindi le fasi di analisi e consolidamento dei requisiti:
				
				\rowcolors{2}{lightest-grayest}{white}
				\begin{longtable}{|l|c|c|c|c|c|c|c|}
					\hline
					\rowcolor{lighter-grayer}
					\textbf{Nome} & \textbf{Re} & \textbf{Am} & \textbf{An} & \textbf{Pg}  & \textbf{Pr}   & \textbf{Ve} & \textbf{Totale} \\
					\hline
					\endfirsthead
					
					\hline
					Giuseppe Vito Bitetti 		& 8 & 5 & 14 & 14 & 26 & 36 & 103\\
					\hline
					\hline
					Lorenzo Dei Negri			& 10 & 9 & 1 & 30 & 23 & 30 & 103\\
					\hline
					\hline
					Nicolò Frison				    & 8 & 15 & 9 & 22 & 23 & 26 & 103\\
					\hline
					\hline
					Fouad Mouad 				 & 12 & 10 & 10 & 21 & 21 & 29 & 103\\
					\hline
					\hline
					Mariano Sciacco 			& 6 & 10 & 3 & 27 & 34 & 23 & 103\\
					\hline
					\hline
					Alessandro Tommasin    & 9 & 9 & 11 & 28 & 18 & 28 & 103\\
					\hline
					\hline
					Giovanni Vidotto 			 & 6 & 12 & 8 & 26 & 29 & 22 & 103\\
					\hline 
					\textbf{Totale}				 & 59 &  70 & 56 & 168 & 174 & 194 & 721\\
					\hline
					\caption{Tabella contenente il prospetto orario preventivato a carico del committente}
				\end{longtable}
				
				La tabella può essere riassunta nel seguente istogramma:
				\begin{figure}[H]
					\centering
					\includegraphics[width=0.8\linewidth]{./images/preventivo/totOreRed1.png}
					\caption{Diagramma ore/ruolo componenti nel totale delle ore rendicontate}
					\label{fig:diagramma suddivione ruoli totale ore rendicontete}
				\end{figure}
				\pagebreak
			
				\paragraph{Totale prospetto economico rendicontato}
				In base al prospetto orario, quello economico sarà il seguente: 
				
				\rowcolors{2}{white}{lightest-grayest}
				\begin{longtable}{|l|c|c|c|c|c|c|c|}
					\hline
					\rowcolor{lighter-grayer}
					\textbf{Ruolo} & \textbf{Ore} & \textbf{Costo in €} \\
					\hline
					\endfirsthead
					
					\hline
					Responsabile 	    & 59 & 1.770,00\\
					\hline 
					\hline
					Amministratore	  & 70 & 1.400,00\\
					\hline
					\hline
					Analista 				& 56 & 1.400,00\\
					\hline
					\hline
					Progettista 		  & 168 & 3.696,00\\
					\hline
					\hline
					Programmatore 	 & 174 & 2.610,00\\
					\hline
					\hline
					Verificatore 		  & 194 & 2.910,00\\
					\hline
					\textbf{Totale} 	& 721 & 13.786,00\\
					\hline
					\caption{Tabella contenente il prospetto economico in riferimento al prospetto orario}
				\end{longtable}

				
				La tabella può essere riassunta nel seguente areogramma:
				\begin{figure}[H]
					\centering
					\includegraphics[width=0.8\linewidth]{./images/preventivo/totOreRed2.png}
					\caption{Diagramma percentuale ore/ruolo nel totale delle ore rendicontate}
					\label{fig:diagramma costi ruolo fase totale ore rendicontate}
				\end{figure}
				
			
			\subsection{Conclusioni}
				Il costo totale preventivato dal gruppo per lo sviluppo del progetto ammonta a 13.786,00 €.
				
				
		
	
	