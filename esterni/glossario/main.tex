%!TEX output_directory = .cache
% ---------------------------
%	  [ GLOSSARIO ]
%	----------------------
%	    Red Round Robin
%  Progetto di SWE (2019-20)
%    Template by Maxelweb
% ---------------------------

% Configurazione primaria del documento

% --------------

\newcommand{\docNome}{ GLOSSARIO }
\newcommand{\docNomeProgetto}{ ThiReMa Project }
\newcommand{\docVersione}{0.0.1}
\newcommand{\docStatus}{in redazione}
\newcommand{\docUso}{esterno}
\newcommand{\docDescrizione}{
	Raccolta di tutti i termini inerenti al progetto con la relativa definizione.
}

% --------------

\newcommand{\docDestinatari}{
	nome cognome \\&
    nome cognome
}
\newcommand{\docRedattori}{
	nome cognome \\&
    nome cognome
}
\newcommand{\docVerificatori}{
	nome cognome \\&
    nome cognome
}
\newcommand{\docApprovazione}{
    nome cognome
}

% ----- NON MODIFICARE SOTTO QUESTA RIGA -----

% ---------------------------
% Configurazioni
% ---------------------------

\documentclass[11pt,a4paper,table]{article}


% Configurazione delle dipendenze e dei package

\usepackage{geometry}
\usepackage{graphicx} 
\usepackage[utf8]{inputenc}
\usepackage{eurosym}
\usepackage[italian]{babel}
\usepackage{float}
\usepackage{subcaption}
\usepackage{wrapfig}
\usepackage{fancyhdr} 
\usepackage{lastpage}
\usepackage{amsfonts}
\usepackage{fancyvrb}
\usepackage{background}
\usepackage{xcolor}
\usepackage{hyperref}   
\usepackage{listings}
\usepackage{longtable}
\usepackage{colortbl}
\usepackage{tikz}
\usepackage{lipsum}
\usepackage[T1]{fontenc}

% Impostazioni pagina e margini

\geometry{
    margin=1.0in,
    top=19.2mm, % NON TOCCARE
    bottom=30mm,
    left=20mm,
    right=20mm
}

% Definizione colori

\definecolor{footer-gray}{HTML}{808080}
\definecolor{light-gray}{gray}{0.6} 
\definecolor{light-grayer}{gray}{0.75} 
\definecolor{lighter-grayer}{gray}{0.85} 
\definecolor{lightest-grayest}{gray}{0.94} 
\definecolor{codegreen}{rgb}{0,0.4,0.2}
\definecolor{codegray}{rgb}{0.5,0.5,0.5}
\definecolor{codepurple}{rgb}{0.58,0,0.82}
\definecolor{backcolour}{rgb}{0.95,0.95,0.96}


% Impostazione header e footer

\pagestyle{fancy}
\setlength\headheight{33pt}
\renewcommand{\headrulewidth}{0pt}
\fancyhead{}
\lhead{\includegraphics[height=10mm]{res/images/logo.png}}
\rhead{\raisebox{1.4\height}{\leftmark}}


\renewcommand{\footrulewidth}{0.1pt}
\fancyfoot{}
\lfoot{ \textcolor{footer-gray}{\docNome - v\docVersione} }

\renewcommand{\footrule}{\hbox to\headwidth{\color{light-grayer}\leaders\hrule height \footrulewidth\hfill}}
\rfoot{ \textcolor{footer-gray}{Pagina \thepage \hspace{1pt} di \pageref*{LastPage}} }

% Grandezza paragrafi e spaziatura frasi

\setlength{\parindent}{1.8em}
\setlength{\parskip}{1.2em}
\renewcommand{\baselinestretch}{1.1}

% Colori link

\hypersetup{
    colorlinks,
    linkcolor=[HTML]{404040},
    citecolor={blue!50!black},
    urlcolor={red!50!black}
}
\PassOptionsToPackage{hyphens}{url}\usepackage{hyperref}

% Equivalente a <hr>

\newcommand{\hr}{\par\vspace{-.1\ht\strutbox}\noindent\hrulefill\par}

% Tabelle e tabulazione

\setlength{\tabcolsep}{10pt}
\renewcommand{\arraystretch}{1.4}

% Unicode per simbolo euro

\DeclareUnicodeCharacter{20AC}{\euro}

% Configurazione sfondo

\newcommand\DeactivateBG{\backgroundsetup{contents={}}}
\newcommand\ActivateBG{ \backgroundsetup{
    scale=1.0,
    color=black,
    opacity=1.0,
    angle=0,
    contents={%
      \includegraphics[height=297mm]{res/images/background.png}
      }%
}}

% Codice e snippet

\renewcommand{\lstlistingname}{Snippet}
\renewcommand{\lstlistlistingname}{Lista di \lstlistingname s}    
 

\lstdefinestyle{chungusHighlight}{
    frame=tb,
    backgroundcolor=\color{backcolour},   
    commentstyle=\color{codegreen},
    keywordstyle=\color{magenta}\textbf,
    numberstyle=\color{codegray},
    stringstyle=\color{codepurple},
    basicstyle={\ttfamily},
    breakatwhitespace=false,         
    breaklines=true,                 
    captionpos=b,                    
    keepspaces=true,                 
    numbers=left,                    
    numbersep=5pt,                  
    showspaces=false,                
    showstringspaces=false,
    showtabs=false,
    numbers=none,                  
    tabsize=2
}

\lstset{style=chungusHighlight}


% Comando per aggiungere le pagine di ogni sezione

\newcommand{\yetAnotherSectionNamed}[1]{%
    \newpage
    \input{res/sections/#1}
}%
    

% ---------------------------
% + CONFIGURAZIONE AGGIUNTIVA
% ---------------------------

\setcounter{secnumdepth}{0} % No section number
\setcounter{tocdepth}{2} % No section number
\usepackage{sectsty}

\makeatletter
\renewcommand\subsection{\@startsection{subsection}{2}{\z@}%
                                     {-3.25ex\@plus -1ex \@minus -.2ex}%
                                     {-1.5ex \@plus .2ex}%
                                     {\normalfont\large\bfseries}}
\makeatother

% ---------------------------
% Dati frontespizio
% ---------------------------

\title{\hr \huge \textsc{\docNome} \\ 
        \vspace{11pt} \large \textsc{\docNomeProgetto} \hr}

\author{} % Non toccare
\date{} % Non toccare

% ---------------------------
% Composizione del documento
% ---------------------------

\begin{document} 

% Frontespizio

\pagenumbering{gobble}
\DeactivateBG

% FRONTESPIZIO

% Logo aziendale

\begin{figure}[t!]
    \centering
    \includegraphics[height=8.5em]{res/images/logo.png}
\end{figure}

\vspace{-7.5em}

% Titolo principale

\maketitle 
\thispagestyle{empty}


% Riferimenti email e sito web

\vspace{-7em}

\begin{center}
    \href{https://www.redroundrobin.site}{www.redroundrobin.site} --- \href{mailto:redroundrobin.site@gmail.com}{redroundrobin.site@gmail.com}
\end{center}

\vspace{1em}

% Informazioni documento

\begin{table}[ht]
  \begin{center}
    \label{tab:Informazioni_Documento}
    \begin{tabular}{r|l}
        \multicolumn{2}{c}{ \textsc{Informazioni sul documento} } \\
        \hline
    	\textbf{Versione} &  \docVersione \\
		\textbf{Uso} &  \docUso \\
        \textbf{Stato} & \docStatus \\
		\textbf{Destinatari} & \docDestinatari \\
		\textbf{Redattori} & \docRedattori \\
		\textbf{Verificatori} & \docVerificatori \\
		\textbf{Approvazione} &  \docApprovazione \\
    \end{tabular}
  \end{center}
\end{table}


% Descrizione del documento

%\vspace{0em}

%\begin{center}
%   \textbf{Descrizione}\\
%  \docDescrizione
%\end{center}



% Registro delle modifiche

\newpage
\ActivateBG
\pagenumbering{arabic}
\section*{Registro delle modifiche}

\begin{center}
	\rowcolors{2}{lightest-grayest}{white}
	\begin{longtable}{|c|p{3.5cm}|c|p{3cm}|p{3cm}|}
	\hline
	\rowcolor{lighter-grayer}
	\textbf{Versione} & \textbf{Descrizione} & \textbf{Data} & \textbf{Autore} & \textbf{Ruolo} \\
	\hline
	\endfirsthead

	% ----- Modificare da qui -----
	1.0.0+b0.18 & Approvazione del documento & 2020-05-14 & Lorenzo Dei Negri & Responsabile \\
	\hline
	0.0.2+b0.18 & Stesura e revisione del documento & 2020-05-14 & Giuseppe Vito Bitetti e Alessandro Tommasin & Amministratore e verificatore \\
	\hline
	0.0.1+b0.18 & Creazione iniziale del documento & 2020-05-14 & Giuseppe Vito Bitetti & Redattore \\
	\hline

	\end{longtable}
\end{center}


% Tabella dei contenuti

\newpage
\tableofcontents

% Sezioni (adattato per glossario)

\sectionfont{\centering \huge \fbox}

\newpage
% -----------------------
% Sezioni da inserire
% -----------------------
% Pro tip: usare il comando \yetAnotherSectionNamed{nome_file}

% NOTA BENE: AGGIUNGI LE SEZIONI IN ORDINE ALFABETICO

\yetAnotherSectionNamed{a}
\yetAnotherSectionNamed{b}


\end{document}


% EOF