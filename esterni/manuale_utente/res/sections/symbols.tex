\section{Legenda simboli identificativi}

All'interno della web app e del bot Telegram sono presenti diversi simboli usati per identificare dispositivi, sensori e altre entità. Di seguito sono riportati i principali simboli con i relativi significati e riferimenti.

\subsection{Identificativi per dispositivi}

\begin{center}
	\rowcolors{2}{lightest-grayest}{white}
	\begin{longtable}{|c|c|p{12cm}|}
	\hline
	\rowcolor{lighter-grayer}
	\textbf{Formato} & \textbf{Esempio} & \textbf{Descrizione} \\
	\hline
	\endfirsthead
	
	\verb!D#{n}! & D\#1, D\#23  & Identificativo numerico per il dispositivo nel database. \\
	\hline

	\verb!D@{n}! & D@1, D@2  & Identificativo numerico per il dispositivo reale, cui il gateway si mette in comunicazione. \\
	\hline

	\caption{Simboli per dispositivi}
	%%%%%%%%%%%%%%%%%%%%%%
	\end{longtable}
\end{center}


\subsection{Identificativi per sensori}

\begin{center}
	\rowcolors{2}{lightest-grayest}{white}
	\begin{longtable}{|c|c|p{12cm}|}
	\hline
	\rowcolor{lighter-grayer}
	\textbf{Formato} & \textbf{Esempio} & \textbf{Descrizione} \\
	\hline
	\endfirsthead
	
	\verb!S#{n}! & S\#1, S\#23  & Identificativo numerico per il sensore nel database. \\
	\hline

	\verb!S@{n}! & S@1, S@2  & Identificativo numerico per il sensore reale, cui il gateway si mette in comunicazione. \\
	\hline

	\caption{Simboli per sensori}
	%%%%%%%%%%%%%%%%%%%%%%
	\end{longtable}
\end{center}


\subsection{Identificativi per altre entità}

\begin{center}
	\rowcolors{2}{lightest-grayest}{white}
	\begin{longtable}{|c|c|p{12cm}|}
	\hline
	\rowcolor{lighter-grayer}
	\textbf{Formato} & \textbf{Esempio} & \textbf{Descrizione} \\
	\hline
	\endfirsthead
	
	\verb!#{n}! & \#42, \#103  & Identificativo numerico per un entità nel database, come ad esempio un utente o un alert. \\
	\hline

	\verb!gw_! & \verb!gw_prova!  & Prefisso utilizzato per i nomi univoci dei gateway. \\
	\hline

	\caption{Simboli per sensori}
	%%%%%%%%%%%%%%%%%%%%%%
	\end{longtable}
\end{center}