\section{Generale / Membro}
Questa sezione contiene tutte le informazioni di carattere generale dell'applicazione: è quindi indicata a tutti gli utenti che devono svolgere svolgere compiti base nell'applicazione. Ad esempio modificare le proprie impostazioni o tenere sotto controllo le rilevazioni visualizzate sotto forma di grafico o di correlazione all'interno della \glock{web app}.

\subsection{Accesso alla web app}
Per visualizzare il tutorial recati al seguente link: 
\url{https://www.youtube.com/watch?v=PjySMOLCtMA&list=PLPKYjnuIh1FA3b3jn_bwY_ztYzaFn2mIT&index=2&t=0s}

\subsection{Accesso alla web app con TFA}
Per visualizzare il tutorial recati al seguente link: 
\url{https://www.youtube.com/watch?v=DsfMva_ZtGo&list=PLPKYjnuIh1FA3b3jn_bwY_ztYzaFn2mIT&index=2}

\subsection{Visualizzazione dashboard e supporto tecnico}
Per visualizzare il tutorial recati al seguente link: 
\url{https://www.youtube.com/watch?v=feDx1bGgwOY&list=PLPKYjnuIh1FA3b3jn_bwY_ztYzaFn2mIT&index=3}

\subsection{Impostazioni account}
Per visualizzare il tutorial recati al seguente link:
\url{https://www.youtube.com/watch?v=tXF_dbfbd4s&list=PLPKYjnuIh1FA3b3jn_bwY_ztYzaFn2mIT&index=4}

	\subsubsection{Modifica email}
	Tempo 0:15

	\subsubsection{Modifica password}
	Tempo 0:36

	\subsubsection{Attivazione e disattivazione alert}
	Tempo 0:47

\subsection{Configurazione Telegram}
	Per usufruire delle funzionalità di alert è necessario scaricare l'applicazione Telegram dallo store sul proprio dispositivo.
	Dopo aver avviato l'applicazione è necessario entrare nelle impostazioni cliccando prima sul menù ad hamburger in altro a sinistra sullo schermo.
	\begin{figure}[H]
		\centering
		\includegraphics[scale=0.100]{res/images/telegram1.jpg}
		\caption{Come entrare nella sezione impostazioni}
		\label{Screenshot1}
	\end{figure}
	\newpage
	In seguito è necessario cliccare nel punto indicato dalla freccia per poter inserire e/o modificare il proprio username
	\begin{figure}[H]
		\centering
		\includegraphics[scale=0.100]{res/images/telegram2.jpg}
		\caption{Come inserire uno username}
		\label{Screenshot2}
	\end{figure}
	Infine è necessario inserire il nome nel campo indicato prestando attenzione ad inserire almeno 5 caratteri e salvare cliccando la spunta in alto a destra.
	\begin{figure}[H]
		\centering
		\includegraphics[scale=0.100]{res/images/telegram3.jpg}
		\caption{Come salvare lo username inserito}
		\label{Screenshot3}
	\end{figure}

\subsection{Abilitazione notifiche alert e TFA}
Per visualizzare il tutorial recati al seguente link: 
\url{https://www.youtube.com/watch?v=yY3eAmf-qE4&list=PLPKYjnuIh1FA3b3jn_bwY_ztYzaFn2mIT&index=5}

	\subsubsection{Modifica username Telegram su WebApp}
	Tempo 0:11

	\subsubsection{Attivazione bot Telegram}
	Tempo 0:32

	\subsubsection{Abilitazione ricezione notifiche su WebApp}
	Tempo 0:54

\subsection{Alerts e impostazioni di notifica}
Per visualizzare il tutorial recati al seguente link: 
\url{https://www.youtube.com/watch?v=p4cjw_fPQHs&list=PLPKYjnuIh1FA3b3jn_bwY_ztYzaFn2mIT&index=6}

	\subsubsection{Visualizzazione lista alert attivi}
	Tempo 0:05

	\subsubsection{Modifica preferenze di ricezione alert}
	Tempo 0:38


\subsection{Gestione pagine view}
Per visualizzare il tutorial recati al seguente link: 
\url{https://www.youtube.com/watch?v=hny4ZP4W3KQ&list=PLPKYjnuIh1FA3b3jn_bwY_ztYzaFn2mIT&index=7}

	\subsubsection{Creazione nuova view}
	Tempo 0:08

	\subsubsection{Creazione nuovo grafico}
	Tempo 0:33

	\subsubsection{Eliminazione grafico}
	Tempo 1:17

	\subsubsection{Eliminazione view}
	Tempo 2:00

\subsection{Gestione dispositivi e sensori}
Per visualizzare il tutorial recati al seguente link: 
\url{https://www.youtube.com/watch?v=0V0dy97eWCY&list=PLPKYjnuIh1FA3b3jn_bwY_ztYzaFn2mIT&index=8}

	\subsubsection{Visualizzazione lista dispositivi}
	Tempo 0:05

	\subsubsection{ID logico e ID reale}
	Tempo 0:20

	\subsubsection{Visualizzazione sensori di un dispositivo}
	Tempo 0:40

	\subsubsection{Visualizzazione grafico di un sensore}
	Tempo 1:00

	\subsubsection{Scorciatoie per dispositivi e sensori}
	Tempo 1:57

