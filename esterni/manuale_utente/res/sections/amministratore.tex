\section{Amministratore}
Questa sezione contiene le informazioni necessarie ad un amministratore di sistema per utilizzare l'interfaccia web di RIoT.

\subsection{Gestione utenti}

	\subsubsection{Entrare nella gestione degli utenti}
		Per entrare nella sezione di gestione utenti è necessario cliccare gestione utenti dalla sezione amministrazione del menù. 

	\subsubsection{Visualizzazione lista utenti}	
		All'interno della sezione di gestione utenti è possibile vedere un elenco di tutti gli utenti del sistema con relativo id, nome e cognome, stato, email, ruolo ed eventuale ente di appartenenza. 

	\subsubsection{Visualizzazione informazioni utente}	
		Per visualizzare le informazioni di un utente è necessario cliccare sul nome o sull'id dell'utente all'interno della lista della sezione gestione utenti.
		All'interno della pagina di dettaglio di un utente è possibile vedere informazioni più dettagliate dei singoli membri ed eventualmente resettare la password, disattivare l'utente o entrare nella sezione di modifica.

	\subsubsection{Creazione utente}
		Per creare un nuovo utente è sufficiente cliccare su "crea nuovo utente" all'interno della sezione di gestione degli utenti. Infine è necessario inserire i dati richiesti nel form e cliccare su "crea utente". 
		A questo punto verrà creato un nuovo account e l'utente verrà inserito nella lista.

	\subsubsection{Modifica utente}

		Per modificare le informazioni di un utente è necessario cliccare sul bottone "modifica" presente nella tabella contenente tutti gli utenti presente nella sezione gestione utenti. All'interno della sezione modifica è possibile modificare i dati dell'utente associato ma non solo: è possibile anche inserire il nome utente Telegram, resettare la password, disattivare l'account e attivare l'autenticazione a due fattori

\subsection{Gestione enti}
	
	\subsubsection{Entrare nella gestione enti}
		Per gestire gli enti è necessario entrare nella sezione apposita tramite il menù amministratore posto a sinistra.

	\subsubsection{Visualizzazione lista enti}
		Per visualizzare un elenco con gli enti presenti nel sistema è necessario entrare nella sezione di gestione degli enti. All'interno di questa sezione è possibile vedere un elenco degli enti presenti nel sistema visualizzandone il nome, il luogo e lo status.			

	\subsubsection{Visualizzazione informazioni ente}
		Per visualizzare le informazioni di un ente è possibile cliccare sul bottone Gestisci presente nella lista globale degli enti all'interno della sezione di gestione degli enti. A questo punto è possibile vedere per l'ente selezionato le sue informazioni, la lista degli utenti appartenenti a quell'ente e la lista dei sensori associati.

	\subsubsection{Aggiunta e rimozione sensori}
		All'interno della pagina di dettaglio di un ente, raggiungibile seguendo i passaggi elencati nelle sezioni \textbf{Entrare nella gestione enti} e \textbf{Visualizzazione informazioni ente}, è possibile aggiungere e rimuovere sensori ad un ente. Per inserire un sensore è necessario selezionare un sensore dal form presente sulla destra denominato Aggiungi sensore e cliccare su Aggiungi sensore.
		Per rimuovere un sensore è sufficiente premere l'icona a forma di cestino a fianco del sensore che si vuole rimuovere. Infine per rendere la rimozione effettiva è necessario cliccare su Salva modifiche. 

	\subsubsection{Modifica ente}	
		Per modificare le informazioni di un ente è necessario cliccare sul bottone Modifica all'interno della sezione di gestione degli enti.
		In questa sezione è possibile cambiarne il nome e il luogo. Per salvare le modifiche effettuate cliccare poi su Salva modifiche.

	\subsubsection{Eliminazione ente}	
		Per eliminare un ente è necessario cliccare sul bottone Modifica all'interno della sezione di gestione degli enti.
		Infine cliccare sul bottone Elimina ente posto in alto a destra.

	\subsubsection{Creazione ente}
		Per aggiungere un nuovo ente è necessario cliccare su "Aggiungi ente all'interno della sezione di gestione degli enti. Compilare poi il form presente nella sezione e premere "Salva".

\subsection{Gestione dispositivi}

	\subsubsection{Entrare nella gestione dispositivi}
		Per entrare entrare nella sezione di gestione dei dispositivi è sufficiente cliccare la voce Dispositivi e sensori del menù laterale.

	\subsubsection{Visualizzazione lista dispositivi}
		All'interno della sezione di gestione di dispositivi è possibile vedere per ogni gateway la lista dei dispositivi associati cliccando sul nome del gateway prescelto. 

	\subsubsection{Visualizzazione informazioni dispositivo}
		Per visualizzare le informazioni di un dispositivo, cliccare sul nome o l'id del dispositivo prescelto all'interno della lista di dispositivi visualizzabile nella sezione di gestione dei dispositivi.

	\subsubsection{Creazione dispositivo}
		All'interno della sezione di gestione dei dispositivi è possibile creare un nuovo dispositivo cliccando sul bottone Aggiungi dispositivo. A questo punto è possibile creare un dispositivo inserendo il suo id reale, il nome, il gateway di appartenenza e la frequenza di ricezione dei dati. Per inserire inserire uno o più sensori è necessario compilare il form sottostante inserendo l'id reale del sensore, la sua tipologia e scegliendo se il sensore può o meno ricevere dei comandi. Infine cliccare su Aggiungi sensore.
		Per salvare i dati inseriti nel nuovo dispositivo cliccare su Salva modifiche.

	\subsubsection{Modifica dispositivo}
		Per modificare un dispositivo è necessario entrare nella sezione di gestione dei dispositivi e cliccare sul bottone Modifica associato al dispositivo prescelto.
		A questo punto è necessario modificare i dati ed infine cliccare su Salva modifiche. 

	\subsubsection{Eliminazione dispositivo}	
		Per eliminare un dispositivo è necessario entrare nella sezione di gestione dei dispositivi e cliccare sul bottone Modifica associato al dispositivo prescelto. All'interno di questa pagina è poi sufficiente cliccare sul bottone Elimina dispositivo posto sotto al form di inserimento dei dati del dispositivo.

\subsection{Gestione gateway}

	\subsubsection{Entrare nella gestione gateway}
		Per entrare entrare nella sezione di gestione dei gateway è sufficiente cliccare la voce Gestione gateways del menù laterale.

	\subsubsection{Visualizzazione lista gateway}
		All'interno della sezione di gestione gateway è possibile visualizzare la lista dei gateway disponibili nel sistema e le relative informazioni.

	\subsubsection{Visualizzazione informazioni gateway}
		Le informazioni più dettagliate dei gateway sono disponibili a partire dalla sezione Gestione gateways in cui è presente la lista dei gateways presenti nel sistema.	
		Da questa sezione è possibile cliccare sul nome o l'ID del gateway prescelto e quindi visualizzarne le informazioni e lista dei dispositivi associati.

	\subsubsection{Creazione gateway}	
		Per creare un nuovo gateway è necessario cliccare su Aggiungi gateway a partire dalla lista dei gateway disponibili.
		Nella nuova pagina poi compilare il form presente e cliccare su Conferma aggiunta. 

	\subsubsection{Modifica gateway}	
		Per modificare le informazioni di un gateway preesistente è necessario cliccare sul bottone Modifica presente nella lista dei gateway presenti nel sistema e successivamente modificarne le informazioni. Infine è necessario cliccare su Conferma modifiche per renderle permanenti.

	\subsubsection{Eliminazione gateway}
		Per eliminare un gateway preesistente è necessario cliccare sul bottone Modifica presente nella lista dei gateway presenti nel sistema e successivamente premere sul bottone Elimina.

	\subsubsection{Invio configurazione}
		Per inviare una nuova configurazione di un gateway, entrare nella sezione di gestione dei gateways.
		Da qui è necessario cliccare sul bottone Invia config associato al gateway di cui si vuole inviare la configurazione.


\subsection{Logs}
	
	\subsubsection{Visualizzazione logs}
		Per visualizzare i logs degli utenti del sistema è necassario entrare nella sezione Logs del menù di amministrazione. In questa sezione è possibile vedere le azioni compiute da tutti gli utenti, visualizzandone la data e l'ore, il nome e cognome, l'azione, il rango dell'utente e l'indirizzo IP.