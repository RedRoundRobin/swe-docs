\section{Amministrazione}
Questa sezione contiene le informazioni necessarie ad un amministratore di sistema per utilizzare l'interfaccia web di RIoT.

\subsection{11. Admin - Gestione utenti}

	\begin{itemize}
		\item \href{https://www.youtube.com/watch?v=PjySMOLCtMA&list=PLPKYjnuIh1FA3b3jn_bwY_ztYzaFn2mIT&index=14}{Visualizza il video tutorial su YouTube} 
	\end{itemize}

	\subsubsection{Entrare nella gestione degli utenti}		
		\begin{figure}[H]
		\centering
		\includegraphics[scale=0.600]{res/images/admin/menuUtente.png}
		\caption{Selezione sezione utenti}
	\end{figure}

		Per entrare nella sezione di gestione utenti è necessario cliccare gestione utenti dalla sezione amministrazione del menù. 

	\subsubsection{Visualizzazione lista utenti}	
		\begin{figure}[H]
		\centering
		\includegraphics[scale=0.600]{res/images/admin/listaUtenti.png}
		\caption{Lista degli utenti presenti nel sistema}
	\end{figure}

		All'interno della sezione di gestione utenti è possibile vedere un elenco di tutti gli utenti del sistema con relativo id, nome e cognome, stato, email, ruolo ed eventuale ente di appartenenza. 

	\subsubsection{Visualizzazione informazioni utente}	

		\begin{figure}[H]
		\centering
		\includegraphics[scale=0.600]{res/images/admin/selDettUtente.png}
		\caption{Selezione di un utente}
	\end{figure}

		Per visualizzare le informazioni di un utente è necessario cliccare sul nome o sull'id dell'utente all'interno della lista della sezione gestione utenti.
		\begin{figure}[H]
		\centering
		\includegraphics[scale=0.600]{res/images/admin/dettUtente.png}
		\caption{Informazioni di un utente}
	\end{figure}
		All'interno della pagina di dettaglio di un utente è possibile vedere informazioni più dettagliate dei singoli membri ed eventualmente resettare la password, disattivare l'utente o entrare nella sezione di modifica.

	\subsubsection{Creazione utente}

		\begin{figure}[H]
		\centering
		\includegraphics[scale=0.600]{res/images/admin/selCreazUtente.png}
		\caption{Creazione di un utente}
	\end{figure}

		Per creare un nuovo utente è sufficiente cliccare su "crea nuovo utente" all'interno della sezione di gestione degli utenti.

		\begin{figure}[H]
		\centering
		\includegraphics[scale=0.600]{res/images/admin/creazUtente.png}
		\caption{Form di creazione di un utente}
	\end{figure}

		Infine è necessario inserire i dati richiesti nel form e cliccare su "crea utente". 
		A questo punto verrà creato un nuovo account e l'utente verrà inserito nella lista.

	\subsubsection{Modifica utente}

		\begin{figure}[H]
		\centering
		\includegraphics[scale=0.600]{res/images/admin/selModUtente.png}
		\caption{Selezione di un utente}
	\end{figure}

		Per modificare le informazioni di un utente è necessario cliccare sul bottone "modifica" presente nella tabella contenente tutti gli utenti presente nella sezione gestione utenti.

		\begin{figure}[H]
		\centering
		\includegraphics[scale=0.600]{res/images/admin/modUtente.png}
		\caption{Form di modifica di un utente}
	\end{figure}

		All'interno della sezione modifica è possibile modificare i dati dell'utente associato ma non solo: è possibile anche inserire il nome utente Telegram, resettare la password, disattivare l'account e attivare l'autenticazione a due fattori

\newpage \subsection{12. Admin - Gestione enti}

	\begin{itemize}
		\item \href{https://www.youtube.com/watch?v=PjySMOLCtMA&list=PLPKYjnuIh1FA3b3jn_bwY_ztYzaFn2mIT&index=15}{Visualizza il video tutorial su YouTube} 
	\end{itemize}
	
	\subsubsection{Entrare nella gestione enti}

		\begin{figure}[H]
		\centering
		\includegraphics[scale=0.600]{res/images/admin/menuEnti.png}
		\caption{Selezione sezione enti}
	\end{figure}

		Per gestire gli enti è necessario entrare nella sezione apposita tramite il menù amministratore posto a sinistra.

	\subsubsection{Visualizzazione lista enti}

		\begin{figure}[H]
		\centering
		\includegraphics[scale=0.600]{res/images/admin/listaEnti.png}
		\caption{Lista degli enti presenti nel sistema}
	\end{figure}

		Per visualizzare un elenco con gli enti presenti nel sistema è necessario entrare nella sezione di gestione degli enti. All'interno di questa sezione è possibile vedere un elenco degli enti presenti nel sistema visualizzandone il nome, il luogo e lo status.			

	\subsubsection{Visualizzazione informazioni ente}

		\begin{figure}[H]
		\centering
		\includegraphics[scale=0.600]{res/images/admin/selDettEnte.png}
		\caption{Selezione di un ente}
	\end{figure}

		Per visualizzare le informazioni di un ente è possibile cliccare sul bottone Gestisci presente nella lista globale degli enti all'interno della sezione di gestione degli enti. 

		\begin{figure}[H]
		\centering
		\includegraphics[scale=0.600]{res/images/admin/dettEnte.png}
		\caption{Dettagli di un ente}
	\end{figure}

		A questo punto è possibile vedere per l'ente selezionato le sue informazioni, la lista degli utenti appartenenti a quell'ente e la lista dei sensori associati.
		Alternativamente è possibile accedere ai dettagli cliccando sul nome dell'ente prescelto.

	\subsubsection{Aggiunta e rimozione sensori}
		All'interno della pagina di dettaglio di un ente, raggiungibile seguendo i passaggi elencati nelle sezioni \textbf{Entrare nella gestione enti} e \textbf{Visualizzazione informazioni ente}, è possibile aggiungere e rimuovere sensori ad un ente.

		\begin{figure}[H]
		\centering
		\includegraphics[scale=0.600]{res/images/admin/aggSensoreEnte.png}
		\caption{Aggiunta di un sensore ad un ente}
	\end{figure}

		Per inserire un sensore è necessario selezionare un sensore dal form presente sulla destra denominato Aggiungi sensore e cliccare su Aggiungi sensore.

		\begin{figure}[H]
		\centering
		\includegraphics[scale=0.600]{res/images/admin/rimSensoreEnte.png}
		\caption{Rimozione di un sensore ad un ente}
	\end{figure}

		Per rimuovere un sensore è sufficiente premere l'icona a forma di cestino a fianco del sensore che si vuole rimuovere. Infine per rendere la rimozione effettiva è necessario cliccare su Salva modifiche. 

	\subsubsection{Creazione ente}

		\begin{figure}[H]
		\centering
		\includegraphics[scale=0.600]{res/images/admin/selCreazEnte.png}
		\caption{Creazione di un nuovo ente}
		\end{figure}

		Per aggiungere un nuovo ente è necessario cliccare su "Aggiungi ente" all'interno della sezione di gestione degli enti. 

		\begin{figure}[H]
		\centering
		\includegraphics[scale=0.600]{res/images/admin/creazEnte.png}
		\caption{Form di creazione di un ente}
	\end{figure}

		Compilare poi il form presente nella sezione e premere "Salva".


	\subsubsection{Modifica ente}

		\begin{figure}[H]
		\centering
		\includegraphics[scale=0.600]{res/images/admin/selModEnte.png}
		\caption{Modifica di un ente}
	\end{figure}

		Per modificare le informazioni di un ente è necessario cliccare sul bottone Modifica all'interno della sezione di gestione degli enti.

		\begin{figure}[H]
		\centering
		\includegraphics[scale=0.600]{res/images/admin/modEnte.png}
		\caption{Form di modifica di un ente}
	\end{figure}

		In questa sezione è possibile cambiarne il nome e il luogo. Per salvare le modifiche effettuate cliccare poi su Salva modifiche.

	\subsubsection{Eliminazione ente}	

		\begin{figure}[H]
		\centering
		\includegraphics[scale=0.600]{res/images/admin/selModEnte.png}
		\caption{Eliminazione di un ente}
	\end{figure}


		Per eliminare un ente è necessario cliccare sul bottone Modifica all'interno della sezione di gestione degli enti.

		\begin{figure}[H]
		\centering
		\includegraphics[scale=0.600]{res/images/admin/elimEnte.png}
		\caption{Eliminazione di un ente}
	\end{figure}


		Infine cliccare sul bottone Elimina ente posto in alto a destra.

	

\newpage \subsection{13. Admin - Gestione dispositivi}

	\begin{itemize}
		\item \href{https://www.youtube.com/watch?v=PjySMOLCtMA&list=PLPKYjnuIh1FA3b3jn_bwY_ztYzaFn2mIT&index=16}{Visualizza il video tutorial su YouTube} 
	\end{itemize}

	\subsubsection{Entrare nella gestione dispositivi}

		\begin{figure}[H]
		\centering
		\includegraphics[scale=0.600]{res/images/admin/menuDisp.png}
		\caption{Selezione della sezione di gestione dei dispositivi}
	\end{figure}

		Per entrare entrare nella sezione di gestione dei dispositivi è sufficiente cliccare la voce Dispositivi e sensori del menù laterale.

	\subsubsection{Visualizzazione lista dispositivi}

		\begin{figure}[H]
		\centering
		\includegraphics[scale=0.600]{res/images/admin/listaDisp.png}
		\caption{Lista dei dispositivi presenti nel sistema}
	\end{figure}

	\begin{figure}[H]
		\centering
		\includegraphics[scale=0.600]{res/images/admin/listaDispOpen.png}
		\caption{Lista dei dispositivi aperta}
	\end{figure}

		All'interno della sezione di gestione di dispositivi è possibile vedere per ogni gateway la lista dei dispositivi associati cliccando sul nome del gateway prescelto. 

	\subsubsection{Visualizzazione informazioni dispositivo}

		\begin{figure}[H]
		\centering
		\includegraphics[scale=0.600]{res/images/admin/selDetDisp.png}
		\caption{Selezione di un dispositivo}
	\end{figure}


		\begin{figure}[H]
		\centering
		\includegraphics[scale=0.600]{res/images/admin/dettDisp.png}
		\caption{Dettagli di un dispositivo}
	\end{figure}


		Per visualizzare le informazioni di un dispositivo, cliccare sul nome o l'id del dispositivo prescelto all'interno della lista di dispositivi visualizzabile nella sezione di gestione dei dispositivi.

	\subsubsection{Visualizzazione dettaglio sensore}

		\begin{figure}[H]
		\centering
		\includegraphics[scale=0.600]{res/images/admin/selDettSens.png}
		\caption{Selezione di un sensore}
	\end{figure}


		Per visualizzare le informazioni di un sensore è necessario cliccare sul bottone Dettagli presente all'interno della lista dei sensori di un determinato dispositivo (Visualizzabile seguendo la sezione \textbf{Visualizzazione informazioni dispositivo}).

		\begin{figure}[H]
		\centering
		\includegraphics[scale=0.600]{res/images/admin/dettSensore.png}
		\caption{Dettagli di un sensore}
	\end{figure}


		All'interno della pagina è possibile visualizzare le informazioni di un dispositivo, il grafico con i suoi valori di output e la lista degli enti a cui è stato assegnato.

	\subsubsection{Creazione dispositivo}

		\begin{figure}[H]
		\centering
		\includegraphics[scale=0.600]{res/images/admin/selCreazDisp.png}
		\caption{Creazione di un dispositivo}
	\end{figure}


		All'interno della sezione di gestione dei dispositivi è possibile creare un nuovo dispositivo cliccando sul bottone Aggiungi dispositivo. 

		\begin{figure}[H]
		\centering
		\includegraphics[scale=0.600]{res/images/admin/creazDisp.png}
		\caption{Form di creazione di un dispositivo}
	\end{figure}

		A questo punto è possibile creare un dispositivo inserendo il suo id reale, il nome, il gateway di appartenenza e la frequenza di ricezione dei dati. 

		\begin{figure}[H]
		\centering
		\includegraphics[scale=0.600]{res/images/admin/aggRimSens.png}
		\caption{Form di aggiunta e rimozione di un sensore}
	\end{figure}

		Per inserire inserire uno o più sensori è necessario compilare il form sottostante inserendo l'id reale del sensore, la sua tipologia e scegliendo se il sensore può o meno ricevere dei comandi. Infine cliccare su Aggiungi sensore.
		Per salvare i dati inseriti nel nuovo dispositivo cliccare su Salva modifiche.

	\subsubsection{Modifica dispositivo}

		\begin{figure}[H]
		\centering
		\includegraphics[scale=0.600]{res/images/admin/selModDisp.png}
		\caption{Modifica di un dispositivo}
	\end{figure}

		Per modificare un dispositivo è necessario entrare nella sezione di gestione dei dispositivi e cliccare sul bottone Modifica associato al dispositivo prescelto.

		\begin{figure}[H]
		\centering
		\includegraphics[scale=0.600]{res/images/admin/modDispositivo.png}
		\caption{Form di modifica di un dispositivo}
	\end{figure}

		A questo punto è necessario modificare i dati ed infine cliccare su Salva modifiche. 

	\subsubsection{Eliminazione dispositivo}	

		\begin{figure}[H]
		\centering
		\includegraphics[scale=0.600]{res/images/admin/selModDisp.png}
		\caption{Eliminazione di un dispositivo}
	\end{figure}


		Per eliminare un dispositivo è necessario entrare nella sezione di gestione dei dispositivi e cliccare sul bottone Modifica associato al dispositivo prescelto. 

		\begin{figure}[H]
		\centering
		\includegraphics[scale=0.600]{res/images/admin/elimDisp.png}
		\caption{Eliminazione di un dispositivo}
	\end{figure}

		All'interno di questa pagina è poi sufficiente cliccare sul bottone Elimina dispositivo posto sotto al form di inserimento dei dati del dispositivo.

\newpage \subsection{14. Admin - Gestione gateway}
	
	\begin{itemize}
		\item \href{https://www.youtube.com/watch?v=PjySMOLCtMA&list=PLPKYjnuIh1FA3b3jn_bwY_ztYzaFn2mIT&index=17}{Visualizza il video tutorial su YouTube} 
	\end{itemize}

	\subsubsection{Entrare nella gestione gateway}

		\begin{figure}[H]
		\centering
		\includegraphics[scale=0.600]{res/images/admin/menuGateway.png}
		\caption{}
	\end{figure}

		Per entrare entrare nella sezione di gestione dei gateway è sufficiente cliccare la voce Gestione gateways del menù laterale.

	\subsubsection{Visualizzazione lista gateway}

		\begin{figure}[H]
		\centering
		\includegraphics[scale=0.600]{res/images/admin/listaGateway.png}
		\caption{Lista dei gateway presenti nel sistema}
	\end{figure}

		All'interno della sezione di gestione gateway è possibile visualizzare la lista dei gateway disponibili nel sistema e le relative informazioni.

	\subsubsection{Visualizzazione informazioni gateway}

		\begin{figure}[H]
		\centering
		\includegraphics[scale=0.600]{res/images/admin/selDettGateway.png}
		\caption{Selezione di un gateway}
	\end{figure}


		Le informazioni più dettagliate dei gateway sono disponibili a partire dalla sezione Gestione gateways in cui è presente la lista dei gateways presenti nel sistema.

		\begin{figure}[H]
		\centering
		\includegraphics[scale=0.600]{res/images/admin/dettGateway.png}
		\caption{Dettagli di un gateway}
	\end{figure}

		Da questa sezione è possibile cliccare sul nome o l'ID del gateway prescelto e quindi visualizzarne le informazioni e lista dei dispositivi associati.

	\subsubsection{Creazione gateway}

		\begin{figure}[H]
		\centering
		\includegraphics[scale=0.600]{res/images/admin/selCreazGateway.png}
		\caption{Creazione di un gateway}
	\end{figure}


		Per creare un nuovo gateway è necessario cliccare su Aggiungi gateway a partire dalla lista dei gateway disponibili.

		\begin{figure}[H]
		\centering
		\includegraphics[scale=0.600]{res/images/admin/creazGateway.png}
		\caption{Form di creazione di un gateway}
	\end{figure}


		Nella nuova pagina poi compilare il form presente e cliccare su Conferma aggiunta. 

	\subsubsection{Modifica gateway}

		\begin{figure}[H]
		\centering
		\includegraphics[scale=0.600]{res/images/admin/selModGateway.png}
		\caption{Modifica di un gateway}
	\end{figure}

		Per modificare le informazioni di un gateway preesistente è necessario cliccare sul bottone Modifica presente nella lista dei gateway presenti nel sistema e successivamente modificarne le informazioni.

		\begin{figure}[H]
		\centering
		\includegraphics[scale=0.600]{res/images/admin/modGateway.png}
		\caption{Form di modifica di un gateway}
	\end{figure}

		Infine è necessario cliccare su Conferma modifiche per renderle permanenti.

	\subsubsection{Eliminazione gateway}

		\begin{figure}[H]
		\centering
		\includegraphics[scale=0.600]{res/images/admin/selModGateway.png}
		\caption{Eliminazione di un gateway}
	\end{figure}

		\begin{figure}[H]
		\centering
		\includegraphics[scale=0.600]{res/images/admin/elimGateway.png}
		\caption{Eliminazione di un gateway}
	\end{figure}


		Per eliminare un gateway preesistente è necessario cliccare sul bottone Modifica presente nella lista dei gateway presenti nel sistema e successivamente premere sul bottone Elimina.

	\subsubsection{Invio configurazione}

		Per inviare una nuova configurazione di un gateway, entrare nella sezione di gestione dei gateways.

		\begin{figure}[H]
		\centering
		\includegraphics[scale=0.600]{res/images/admin/inviaConfig.png}
		\caption{Invio di una configurazione}
	\end{figure}

		Da qui è necessario cliccare sul bottone Invia config associato al gateway di cui si vuole inviare la configurazione.


\newpage \subsection{15. Admin - Logs}

	\begin{itemize}
		\item \href{https://www.youtube.com/watch?v=PjySMOLCtMA&list=PLPKYjnuIh1FA3b3jn_bwY_ztYzaFn2mIT&index=18}{Visualizza il video tutorial su YouTube} 
	\end{itemize}
	
	\subsubsection{Visualizzazione logs}

		\begin{figure}[H]
		\centering
		\includegraphics[scale=0.600]{res/images/admin/menuLogs.png}
		\caption{Selezione della voce logs dal menù}
	\end{figure}

		Per visualizzare i logs degli utenti del sistema è necassario entrare nella sezione Logs del menù di amministrazione.

		\begin{figure}[H]
		\centering
		\includegraphics[scale=0.600]{res/images/admin/logs.png}
		\caption{Logs del sistema}
	\end{figure}

		In questa sezione è possibile vedere le azioni compiute da tutti gli utenti, visualizzandone la data e l'ore, il nome e cognome, l'azione, il rango dell'utente e l'indirizzo IP.