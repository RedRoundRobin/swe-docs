\section{Analisi del prodotto}
	\subsection{Scopo del prodotto}
		Il capitolato C6 si pone come obiettivo quello di creare una /glock{web-application} che permette di analizzare grosse moli di dati ricevuti da sensori eterogenei tra loro. Tale applicazione mette a disposizione un'interfaccia che permette di visualizzare alcuni dati di interesse od eventuali correlazioni tra i dati stessi. Infine, per ogni tipologia di dato è possibile assegnarne il monitoraggio ad un particolare ente, ruolo o gruppo.
		
	\subsection{Analisi della struttura}
		La struttura del prodotto sarà divisa in:
		\begin{itemize}
			\item \textbf{Interfaccia di gestione}: L'interfaccia di gestione ed interrogazione permetterà agli utenti di seguire l'andamento di uno o più sensori, di controllarli, attivandoli o disattivandoli ed infine, agli utenti con maggiori privilegi, di aggiungere o rimuovere altri utenti o gruppi.
			Tramite questa interfaccia deve essere possibile inviare dei comandi al giusto gateway tramite un apposito /glock{topic} di Kafka per controllare gli /glock{edge data point}. 
			L'interfaccia di gestione dovrà dare la possibilità agli utenti di gestire i propri dati (quale ad esempio la password di accesso).
			\item \textbf{Database}: Si è pensato di utilizzare due tipi di database in base ai dati da memorizzare:
			\begin{itemize}
				\item Un database di tipo /glock{Time-Series} per memorizzare i dati temporali ricevuti dai sensori;
				\item Un database relazionale, per memorizzare i metadati relativi ai sensori (ad esempio il loro censimento) o relativi agli utenti che andranno ad utilizzare l'interfaccia di gestione.
			\end{itemize}

			\item \textbf{Gateway}: I dispositivi (o edge data point) che invieranno i dati dei loro sensori, non si interfaccerano direttamente con i database o con il cluster Kafka (descritto in seguito) bensì dialogheranno con un /glock{gateway} tramite degli appositi driver.
			Saranno presenti diversi gateway sudddivisi per locazione, in modo tale da facilitare l'invididuazione di uno specifico edge data point.
			Le modalità di invio dei dati (quale ad esempio la frequenza di invio) per ogni singolo dispositivo dovranno essere inviate dall'interfaccia ed essere salvate nell'apposito gateway, che poi si occuperà di filtrare i dati inviati dai dispositivi. 

			\item \textbf{Dispositivo (Edge Data Point)}: I dispositivi dopo essere stati censiti, invieranno dati al proprio gateway, senza essere consci della propria configurazione.s 

			\item \textbf{Cluster Apache Kafka}: Tramite Kafka e le sue API verranno ricevuti i dati dai sensori, passando da un gateway, verranno manipolati tramite degli stream ed infine salvati nel database e/o mostrati agli utenti che li richiedono. L'interfaccia di gestione ed il database contenente le misurazioni dei sensori si interfacceranno quindi con Kafka e non direttamente tra loro.

		\end{itemize}
	\subsection{Attori}
		Gli attori individuati dopo un'attenta analisi del capitolato sono i seguenti:
		\subsubsection{Attori principali}
		\begin{itemize}
			\item \textbf{Utente non autorizzato}: Questa tipologia di utente rappresenta un 
			vecchio amministratore che ora non ha più alcun privilegio o un utente appartenente ad un ente che poi è stato eliminato. Nello specifico non ha accesso al sito in quanto non avrebbe i permessi necessari per visualizzare alcuna sezione.

			\item \textbf{Utente autorizzato}: Un utente autorizzato ha la possibilità di autenticarsi e quindi di accedere alle varie sezioni del sito. Questo tipo di utente deve appartenere ad uno ed un solo ente, ovvero un gruppo di utenti che hanno il permesso di visualizzare (in forma di tabella o di grafico) le misurazioni dei sensori assegnati a quell'ente.
			Infine un utente autorizzato può gestire un dispositivo remoto (da notare che un dispositivo può essere assegnato ad uno o più enti).

			\item \textbf{Moderatore ente}: Un moderatore ente ha tutti i permessi di un utente autorizzato e può gestire (visualizzare, rimuovere o aggiungere) i membri del proprio ente. Di questi ultimi può inoltre visualizzare le relative attività (logs).
			Infine questo attore può impostare dei valori soglia, che quando superati provocano l'invio di avvisi a tutti i membri dell'ente tramite un'apposita sezione nell'interfaccia di gestione. 
			Possono essere presenti uno o più moderatori ente per ogni singolo ente.

			\item \textbf{Amministratore generale}: L'amministratore generale rappresenta un utente con il più alto livello di privilegi. Questo attore può infatti gestire (modificare, creare e rimuovere) gli enti, i loro membri e i dispositivi a loro assegnati.
			L'amministratore può inoltre vedere tutte le attività di ogni singolo utente.
			Possono essere presenti uno o più amministratori generali.
		\end{itemize}
		\subsubsection{Attori secondari}
			Al momento non sono stati trovati degli attori secondari.

	