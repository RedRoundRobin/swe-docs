\subsubsection{UC 11 - Amministrazione - Gestione utenti}

		\begin{itemize}
			\item \textbf{Attori Primari}: Amministratore.
			\item \textbf{Descrizione}: L'amministratore, che sta visualizzando il menù, seleziona la voce "Gestione utenti" che permette la gestione degli utenti all'interno del sistema.
			\item \textbf{Precondizione}: L'amministratore visualizza il menù di navigazione.
			\item \textbf{Postcondizione}: L'amministratore ha visualizzato/gestito gli utenti all'interno del sistema. 
			\item \textbf{Scenario Principale}:
			\begin{enumerate}
				\item{L'amministratore seleziona la voce "Gestione utenti amministratore"}
				\item{L'amministratore può svolgere diverse azioni allo scopo di gestire gli utenti all'interno del sistema}
			\end{enumerate}	
		\end{itemize}

			\subsubsection{UC 11.1 - Visualizzazione lista utenti completa}
			\begin{itemize}
				\item \textbf{Attori Primari}: Amministratore.
				\item \textbf{Descrizione}: L'amministratore, che sta visualizzando la schermata di gestione utenti ente, visualizza la lista degli utenti nel sistema.
				\item \textbf{Precondizione}: L'amministratore sta visualizzando la schermata per la gestione degli utenti nel sistema.
				\item \textbf{Postcondizione}: L'amministratore visualizza la lista degli utenti registrati al sistema.
				\item \textbf{Scenario Principale}:
				\begin{enumerate}
					\item{L'amministratore clicca sulla sezione per la visualizzazione della lista utenti}
					\item{L'amministratore visualizza la lista degli utenti registrati al sistema}
				\end{enumerate}	
			\end{itemize}
			
			\subsubsection{UC 11.2 - Creazione nuovo account}
			\begin{itemize}
				\item \textbf{Attori Primari}: Amministratore.
				\item \textbf{Descrizione}: L'amministratore crea un nuovo account che viene inserito nel sistema e assegnato a un ente.
				\item \textbf{Precondizione}: L'amministratore visualizza la schermata per la gestione utenti.
				\item \textbf{Postcondizione}: L'amministratore ha creato un nuovo account.
				\item \textbf{Scenario Principale}:
				\begin{enumerate}
					\item{L'amministratore visualizza la schermata per la gestione degli utenti}
					\item{L'amministratore seleziona la voce per la creazione di un nuovo account}
					\item{L'amministratore compila il form utente con i dati dell'utente da aggiungere}
					\item{L'amministratore preme sul bottone di salvataggio}
					\item{L'amministratore ha aggiunto un nuovo utente al sistema}
				\end{enumerate}	
				\item \textbf{Inclusioni}:
					\item L'amministratore compila il form utente con i dati dell'utente da aggiungere (UC 11.2.1)
				\item \textbf{Estensioni}:
				\begin{itemize}
					\item L'amministratore inserisce un email non valida (UC 11.7)
					\item L'amministratore inserisce un nome e/o cognome non validi (UC 11.8)
				\end{itemize}
			\end{itemize}
			
			\subsubsection{UC 11.2.1 - Compilazione form nuovo account}
			\begin{itemize}
				\item \textbf{Attori Primari}: Amministratore.
				\item \textbf{Descrizione}: L'amministratore compila i campi per l'aggiunta dell'utente: il campo "email", corrispondente all'email dell'utente, e i campi "nome" e "cognome", corrispondenti al nominativo dell'utente. Oltre a ciò, viene richiesto anche di selezionare la tipologia di utente (membro o moderatore) e l'ente a cui assegnarlo.
				\item \textbf{Precondizione}: L'amministratore ha selezionato la voce "crea nuovo membro".
				\item \textbf{Postcondizione}: L'amministratore ha compilato il form per la creazione di un nuovo membro.
				\item \textbf{Scenario Principale}:
				\begin{enumerate}
					\item{L'amministratore seleziona la voce per la creazione di un nuovo membro}
					\item{L'amministratore compila il campo "email"}
					\item{L'amministratore compila il campo "nome"}
					\item{L'amministratore compila il campo "cognome"}
					\item{L'amministratore seleziona la tipologia di utente (membro o moderatore)}
					\item{L'amministratore seleziona l'ente a cui assegnare il nuovo account tra quelli disponibili}
					\item{L'amministratore ha compilato il form utente}
				\end{enumerate}	
			\end{itemize}

			\subsubsection{UC 11.3 - Visualizzazione profilo utente}
			\begin{itemize}
				\item \textbf{Attori Primari}: Amministratore.
				\item \textbf{Descrizione}: L'amministratore vuole visualizzare il profilo di un utente presente nel sistema.
				\item \textbf{Precondizione}: L'amministratore seleziona un utente dalla lista degli utenti.
				\item \textbf{Postcondizione}: L'amministratore ha visualizzato i dati dell'utente selezionato.
				\item \textbf{Scenario Principale}:
				\begin{enumerate}
					\item{L'amministratore seleziona un utente dalla lista degli utenti}
					\item{L'amministratore seleziona la voce per visualizzare i dati dell'utente}
					\item{L'amministratore visualizza i dati dell'utente selezionato}
				\end{enumerate}
			\end{itemize}


			\subsubsection{UC 11.4 - Modifica profilo utente}
			\begin{itemize}
				\item \textbf{Attori Primari}: Amministratore.
				\item \textbf{Descrizione}: L'amministratore modifica il profilo dell'utente selezionato dalla lista degli utenti.
				\item \textbf{Precondizione}: L'amministratore seleziona un utente dalla lista degli utenti.
				\item \textbf{Postcondizione}: L'amministratore ha modificato l'utente selezionato appartenente al proprio ente.
				\item \textbf{Scenario Principale}:
				\begin{enumerate}
					\item{L'amministratore seleziona un utente dalla lista degli utenti}
					\item{L'amministratore seleziona la voce per la modifica di un utente}
					\item{L'amministratore compila il form utente contenente i dati da modificare dell'utente}
					\item{L'amministratore preme sul bottone di salvataggio per modificare l'utente selezionato}
					\item{L'amministratore ha modificato l'utente selezionato}
				\end{enumerate}	
				\item \textbf{Inclusioni}:
					\item L'amministratore compila il form utente con i dati dell'utente da modificare (UC 11.4.1)
				\item \textbf{Estensioni}:
				\begin{itemize}
					\item L'amministratore inserisce un email non valida (UC 11.7)
					\item L'amministratore inserisce un nome e/o cognome non validi (UC 11.8)
				\end{itemize}
			\end{itemize}

			\subsubsection{UC 11.4.1 - Compilazione form nuovo utente}
			\begin{itemize}
				\item \textbf{Attori Primari}: Amministratore.
				\item \textbf{Descrizione}: L'amministratore compila i campi obbligatori da modificare a un utente: il campo "email", corrispondente all'email dell'utente, e i campi "nome" e "cognome", corrispondenti al nominativo dell'utente. Oltre a ciò, può modificare l'ente di appartenenza e la tipologia dell'account (membro o moderatore).
				\item \textbf{Precondizione}: L'amministratore ha selezionato la voce per modificare un membro dell'ente.
				\item \textbf{Postcondizione}: L'amministratore ha compilato il form per la modifica profilo di un membro dell'ente.
				\item \textbf{Scenario Principale}:
				\begin{enumerate}
					\item{L'amministratore seleziona la voce per la modifica di un nuovo membro e compare un form}
					\item{L'amministratore modifica il campo "email"}
					\item{L'amministratore modifica il campo "nome"}
					\item{L'amministratore modifica il campo "cognome"}
					\item{L'amministratore seleziona la tipologia di utente (membro o moderatore)}
					\item{L'amministratore seleziona l'ente a cui assegnare il nuovo account tra quelli disponibili}
					\item{L'amministratore ha completato il form utente e lo invia}
				\end{enumerate}	
			\end{itemize}


			\subsubsection{UC 11.5 - Reset password account}
			\begin{itemize}
				\item \textbf{Attori Primari}: Amministratore.
				\item \textbf{Descrizione}: L'amministratore vuole resettare la password di un account censito nel sistema.
				\item \textbf{Precondizione}: L'amministratore seleziona un utente dalla lista utenti.
				\item \textbf{Postcondizione}: L'amministratore ha resettato la password dell'account selezionato.
				\item \textbf{Scenario Principale}:
				\begin{enumerate}
					\item{L'amministratore seleziona un utente cui vuole resettare la password}
					\item{L'amministratore seleziona la voce per resettargli la password}
					\item{L'amministratore ha resettato la password all'utente di sistema}
				\end{enumerate}		
			\end{itemize}

			
			\subsubsection{UC 11.6 - Disattivazione utente}
			\begin{itemize}
				\item \textbf{Attori Primari}: Amministratore.
				\item \textbf{Descrizione}: L'amministratore vuole disattivare l'account di un utente.
				\item \textbf{Precondizione}: L'amministratore seleziona un utente dalla lista utenti.
				\item \textbf{Postcondizione}: L'amministratore ha disattivato l'account dell'utente selezionato.
				\item \textbf{Scenario Principale}:
				\begin{enumerate}
					\item{L'amministratore seleziona un utente da disattivare}
					\item{L'amministratore seleziona la voce per disattivare l'utente}
					\item{L'amministratore ha disattivato l'utente dal sistema.}
				\end{enumerate}		
			\end{itemize}


			\subsubsection{UC 11.7 - Errore: email non valida}
			\begin{itemize}
				\item \textbf{Attori Primari}: Amministratore.
				\item \textbf{Descrizione}: Dopo aver premuto il bottone per salvare i dati inseriti nel form utente viene visualizzato il messaggio di errore "Email non valida" perchè l'email inserita non è valida. 
				\item \textbf{Precondizione}: L'amministratore ha premuto sul bottone di salvataggio del form utente.
				\item \textbf{Postcondizione}: Visualizzazione messaggio di errore "Email non valida".
				\item \textbf{Scenario Principale}:
				\begin{enumerate}
					\item{L'amministratore ha premuto sul bottone di salvataggio del form utente}
					\item{La email inserita nel campo "email" non è valida}
					\item{Viene visualizzato il messaggio di errore "Email non valida"}
				\end{enumerate}	
			\end{itemize}
			
			\subsubsection{UC 11.8 - Errore: nome e/o cognome non validi}
			\begin{itemize}
				\item \textbf{Attori Primari}: Amministratore.
				\item \textbf{Descrizione}: Dopo aver premuto il bottone per salvare i dati inseriti nel form utente viene visualizzato il messaggio di errore "Nome e/o cognome non validi" perchè il nome e/o il cognome inseriti non sono validi. 
				\item \textbf{Precondizione}: L'amministratore ha premuto sul bottone di salvataggio del form utente.
				\item \textbf{Postcondizione}: Visualizzazione messaggio di errore "Nome e/o cognome non validi".
				\item \textbf{Scenario Principale}:
				\begin{enumerate}
					\item{L'amministratore ha premuto sul bottone di salvataggio del form utente}
					\item{Il nome nel campo "nome" non è valido e/o il cognome nel campo "cognome" non è valido}
					\item{Viene visualizzato il messaggio di errore "Nome e/o cognome non validi"}
				\end{enumerate}	
			\end{itemize}
			

			
			
			