\subsubsection{UC 10 - Gestione enti}
		\begin{itemize}
			\item \textbf{Attori Primari}: [AM]
			\item \textbf{Descrizione}: L'utente, che sta visualizzando il menù, seleziona la voce "Gestione enti" che permette la gestione degli enti all'interno del sistema.
			\item \textbf{Precondizione}: L'utente visualizza il menu
			\item \textbf{Postcondizione}: L'utente ha visualizzato/gestito gli enti all'interno del sistema. 
			\item \textbf{Scenario Principale}:
			\begin{enumerate}
				\item{L'utente seleziona la voce "Gestione enti"}
				\item{L'utente può svolgere diverse azioni allo scopo di gestire gli enti all'interno del sistema}
			\end{enumerate}	
		\end{itemize}

			\paragraph{UC 10.1 - Visualizzazione lista enti }
			\begin{itemize}
				\item \textbf{Attori Primari}: [AM]
				\item \textbf{Descrizione}: L'utente, che sta visualizzando la schermata per la gestione enti, visualizza la lista degli enti presenti nel sistema.
				\item \textbf{Precondizione}: L'utente visualizza la schermata per la gestione degli enti
				\item \textbf{Postcondizione}: L'utente visualizza la lista degli enti presenti nel sistema
				\item \textbf{Scenario Principale}:
				\begin{enumerate}
					\item{L'utente seleziona la voce "visualizza enti"}
					\item{L'utente visualizza la lista enti}
				\end{enumerate}	
			\end{itemize}

			\paragraph{UC 10.2 - Visualizzazione informazioni ente}
			\begin{itemize}
				\item \textbf{Attori Primari}: [AM]
				\item \textbf{Descrizione}: L'amministratore, che sta visualizzando la lista degli enti, seleziona un ente e ne visualizza le informazioni riguardanti.
				\item \textbf{Precondizione}: L'utente visualizza la la schermata gestione enti e la lista degli enti
				\item \textbf{Postcondizione}: L'utente visualizza le informazioni di un ente selezionato
				\item \textbf{Scenario Principale}:
				\begin{enumerate}
					\item{L'utente seleziona dalla lista un ente}
					\item{L'utente visualizza le informazioni riguardanti l'utente selezionato}
				\end{enumerate}	
			\end{itemize}

			\paragraph{UC 10.3 - Aggiunta ente}
			\begin{itemize}
				\item \textbf{Attori Primari}: [AM]
				\item \textbf{Descrizione}: L'amministratore, che sta visualizzando la schermata per la gestione degli enti, aggiunge un nuovo ente al sistema.
				\item \textbf{Precondizione}: L'utente visualizza la schermata per la gestione degli enti
				\item \textbf{Postcondizione}: L'utente ha creato un nuovo ente
				\item \textbf{Scenario Principale}:
				\begin{enumerate}
					\item{L'utente inserisce i dati nei campi}
					\item{L'ente viene creato dall'utente con le informazioni fornite}
				\end{enumerate}	
				\item \textbf{Estensioni}:
					\begin{itemize}
						\item Il nome dell'ente viene inserito in un formato non valido
					\end{itemize}
			\end{itemize}		

			\paragraph{UC 10.4 - Modifica ente}
			\begin{itemize}
				\item \textbf{Attori Primari}: [AM]
				\item \textbf{Descrizione}: L'amministratore, che sta visualizzando la lista con tutti gli enti creati fino a quel momento, seleziona l'ente di cui desidera modificare le informazioni. 
				\item \textbf{Precondizione}: L'utente visualizza la lista degli enti
				\item \textbf{Postcondizione}: L'utente ha modificato l'ente selezionato
				\item \textbf{Scenario Principale}:
				\begin{enumerate}
					\item{L'utente seleziona l'ente da modificare}
					\item{L'utente modifica i dati dell'ente selezionato}
				\end{enumerate}
			\end{itemize}	

			\paragraph{UC 10.5 - Rimozione ente}
			\begin{itemize}
				\item \textbf{Attori Primari}: [AM]
				\item \textbf{Descrizione}: L'amministratore, che sta visualizzando la lista degli enti, seleziona un ente e lo rimuove dal sistema.
				\item \textbf{Precondizione}: L'utente visualizza la lista degli enti appartenenti al sistema
				\item \textbf{Postcondizione}: L'utente he rimosso un ente
				\item \textbf{Scenario Principale}:
				\begin{enumerate}
					\item{L'utente seleziona l'ente da rimuovere}
					\item{L'ente selezionato viene rimosso dal sistema}
				\end{enumerate}	
			\end{itemize}		