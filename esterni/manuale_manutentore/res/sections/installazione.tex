\section{Installazione}
	In questa sezione viene spiegato come effettuare l'installazione delle componenti del progetto ThiReMa.
	Per effettuare l'installazione del prodotto è sufficiente eseguire il file docker-compose.yml nel proprio terminale tramite il comando:
	\begin{verbatim}
	docker-compose up -d
	\end{verbatim}

	\subsection{Gateway}
		Per installare localmente la componente gateway, è necessario eseguire nel terminale il maven build lifecycle all'interno della cartella della componente tramite il comando:
		\begin{verbatim}
		mvn install
		\end{verbatim}
		ed in seguito avviare il client.

	\subsubsection{Data Collector}
		Per installare localmente la componente data collector è necessario eseguire nel terminale il maven build lifecycle all'interno della cartella della componente tramite il comando:
		\begin{verbatim}
		mvn install
		\end{verbatim}
		ed in seguito avviare il client.

	\subsubsection{Api}
		Per installare localmente la componente api è necessario eseguire nel terminale il maven build lifecycle all'interno della cartella della componente tramite il comando:
		\begin{verbatim}
		mvn install
		\end{verbatim}
		ed in seguito avviare l'eseguibile ApirestApplication.java.

	\subsubsection{Web Application}
		Per installare localmente la componente api è necessario eseguire nel terminale i seguenti comandi all'interno della cartella della componente:
		\begin{verbatim}
		$npm install
		$npm run dev
		$composer install
		$php artisan serve
		\end{verbatim}
		ed in seguito collegarsi all'indirizzo specificato nel terminale.
	\subsubsection{Bot Telegram}
		Per installare ed eseguire localmente il bot Telegram è sufficiente eseguire i seguenti comandi mentra ci si trova all'interno della cartella della componente:
		\begin{verbatim}
		$npm update
		$node main
		\end{verbatim}
		
