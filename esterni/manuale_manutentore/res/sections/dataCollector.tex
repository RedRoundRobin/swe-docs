\subsection{Data collector}
	La componente \textit{data collector} permette di trasferire i dati prodotti dai singoli gateway ed inviati nei vari topic di Kafka, all'interno di un database di tipo timeseries.
	\newline
	Allo stesso tempo, permette di filtrare i dati collezionati ed inviare dei messaggi di avviso, all'interno di un apposito topic, nel caso in cui vengano rilevati dei valori anomali.
	\newline
	Il filtraggio dei dati viene fatto a partire dagli alert impostati all'interno della web app e salvati in un database relazionale (PostgreSQL).
	\begin{itemize}
		\item La componente è stata sviluppata in Java 11.
	\end{itemize}
	
		\subsubsection{Diagramma dei package}%%%%%%%%%OK
		\begin{figure}[H]
				\centering
				\includegraphics[scale=0.600]{res/images/DATACOLLECTOR/Packagekafkadatacollector.png}
				\caption{Diagramma dei package della componente data collector}
				\label{Diagramma 5}
			\end{figure}
		\subsubsection{Dipendenze esterne}
			La componente ha due dipendenze esterne:
			\begin{itemize}
				\item \textbf{KafkaProducer<K, V>}, classe concreta che implementa l'interfaccia Producer<K,V>. Svolge il compito di client per il cluster kafka, pubblicando dati all'interno di un topic. La classe DataFilter ne possiede un riferimento.
				\item \textbf{KafkaConsumer<K,V>}, classe concreta che implementa l'interfaccia Consumer<K,V>. Svolge il compito di client per il cluster Kafka, consumando i messaggi resenti all'interno di uno o più topic. La classi DataInserter e DataFilter ne possiedono un riferimento.	
			\end{itemize}	
		\begin{landscape}
		\subsubsection{Diagramma delle classi}%%%%%%%%%%%%%%%%%%%%%%%OK
			\begin{figure}[H]
				\centering
				\includegraphics[scale=0.550]{res/images/DATACOLLECTOR/ClassikafkaDataCollector.png}
				\caption{Diagramma delle classi della componente data collector}
				\label{Diagramma 6}
			\end{figure}
		Come si vede nel diagramma la classe DataInserter ha un riferimento alla classe Database (che rappresenta Timescale in questo caso) ed alla classe Consumer. La classe DataFilter invece ha, oltre che un riferimento a Database (che rappresenta Postgre) e a Consumer, anche un riferimento a Producer e AlertTimeTable; questo perché DataFilter dopo aver filtrato i dati grazie ai suoi metodi (e grazie alla classe AlertTimeTable), inserisce i messaggi derivati dai dati consumati direttamente in un topic Kafka tramite un producer.  
		\end{landscape}
		\begin{landscape}
		\subsubsection{Diagrammi di sequenza}%%%%%%%%%OK
			\begin{figure}[H]
				\centering
				\includegraphics[scale=0.550]{res/images/DATACOLLECTOR/DataFilter.ThreadsKafkaDataCollector.png}
				\caption{Diagramma di sequenza in cui viene mostrato il funzionamento del filtraggio dati nella componente data collector}
				\label{Diagramma 7}
			\end{figure}
			Come si evince dal diagramma in alto, funzionamento di DataFilter è il seguente:
			\begin{itemize}
				\item DataFilter apre la connessione con il database relazionale Postgre;
				\item vengono quindi estratti i dati dai topic Kafka collegati ai gateway;
				\item i valori vengono filtrati garantendo che il dispositivo ed il sensore siano presenti nel database e che i valori rilevati superino la soglia impostata.
				\item i valori vengono filtrati nuovamente per verificare che gli utenti a cui dovrebbero arrivare abbiano configurato il bot Telegram correttamente;
				\item infine vengono prodotti un insieme di messaggi di alert che, tramite un produttore vengono inseriti nel topic kafka "alerts". 
			\end{itemize}
			\begin{figure}[H]
				\centering
				\includegraphics[scale=0.550]{res/images/DATACOLLECTOR/DataInserter.ThreadsKafkaDataCollector.png}
				\caption{Diagramma di sequenza in cui viene mostrato il funzionamento dell'inserimento dati nella componente data collector}
				\label{Diagramma 8}
			\end{figure}
			Il comportamento di DataInserter è il seguente:
			\begin{itemize}
				\item DataInserter apre una connessione con il database Timescale;
				\item Ogni qualvolta il suo Consumer trova dei nuovi dati nel topic collegati ai gateway, vengono inviati i messaggi al DataInserter;
				\item DataInserter infine inserisce i dati all'interno del database.
			\end{itemize}
	\end{landscape}