\section{Tecnologie interessate}
	\subsection{Linguaggi}
		I linguaggi consigliati per una futura estensione e/o manutenzione di ThiReMa sono:
		\subsubsection{Java}
			Il linguaggio è stato scelto perchè permette di interfacciarsi con facilità con la piattaforma Kafka richiesta dal capitolato. Java permette inoltre di gestire le dipendenze in maniera automatica tramite Maven e di scrivere ed eseguire suite di test in maniera facile e veloce tramite JUnit. Il linguaggio è inoltre ben conosciuto e facile da apprendere per uno sviluppatore con una qualche esperienza nella programmazione OO. 
		\subsubsection{PHP}
			Questo linguaggio è stato scelto per lo sviluppo dell'applicazione web perchè fornito di framework che permettono di realizzare componenti web in maniera veloce ed intuitiva, utilizzando sempre una sintassi elegante.
		\subsubsection{Javascript}
			Javascript è stato scelto perchè permette una realizzazione della componente bot telegram in poche righe di codice usando dei moduli presenti per il suddetto linguaggio. Un altro punto a suo favore è che il suo tempo di apprendimento è piuttosto breve.
	\subsection{Strumenti}
		\subsubsection{Docker}
			Questo strumento è stato utilizzato sia perché consigliato fortemente dal capitolato che perché permette il rilascio delle varie componenti in ambienti separati tra loro detti \glock{Container}. Questi ultimi vengono configurati tramite dei Dockerfile dove vengono specificate le operazioni da eseguire all'avvio del container. Infine tramite dei docker-compose vengono assemblate le componenti, permettendo in pochi comandi di costruire l'intera architettura. per maggiori informazioni si consiglia di visitare il seguente link: \url{https://www.docker.com}
	\subsection{Framework e librerie}
		Le librerie esterne utilizzate sono state gestite tramite Maven per le componenti sviluppate in Java (gateway, data connector ed api); per i moduli del bot Telegram (sviluppato in Javascript) è richiesto invece il gestore di pacchetti di Node.js \textit{npm}. Infine per la componente web è necessario, oltre al già menzionato \textit{npm} anche il gestore di pacchetti PHP \textit{Composer}.
		\subsubsection{Spring}
			Il framework Spring è stato utilizzato per lo sviluppo della componente api. Più precisamente dell'intera suite sono stati utilizzati Spring Boot, Spring Security, Spring Kafka e Spring Jpa. Per maggiori informazioni visitare il seguente link:\url{https://spring.io/projects/spring-framework}
		\subsubsection{Jwt}
			Questa libreria permette di utilizzare lo standard JWT per trasmettere informazioni tra due componenti in formato JSON. Per maggiori informazioni visitare il seguente link:\url{https://jwt.io}
		\subsubsection{Laravel}
			Questo framework permette di realizzare applicazioni web utilizzando una sintassi espressiva ed elegante. Nel progetto è stato utizzato per sviluppare la componente web app. Per maggiori informazioni visitare il seguente link: \url{https://laravel.com}
		\subsubsection{Vue.js}
			Questo framework permette di scrivere pagine web reattive e dalla sintassi elegante. Nel progetto è stato utilizzato per sviluppare la componente web app. per maggiori informazioni visitare il seguente link: \url{https://vuejs.org}
		\subsubsection{Axios}
			Questo modulo Javascript viene utilizzato all'interno del bot Telegram per effettuare e ricevere richieste HTTP. per maggiorni informazioni visitare il link seguente: \url{https://axios.nuxtjs.org}
		\subsubsection{Telegraf}
			Questo modulo Javascript viene usato all'interno della componente bot Telegram per creare ed interfacciarsi con le \href{https://core.telegram.org/bots/api}{API ufficiali di Telegram}. Per maggiori informazioni visistare il link seguente: \url{https://telegraf.js.org} 
	