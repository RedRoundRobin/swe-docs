\subsection{Bot Telegram}
	La componente \textit{bot Telegram} permette di ricevere codici di autenticazione a due fattori, notifiche di alert ed inviare direttamente dei comandi ai singoli dispositivi, per alterarne lo stato.
	\newline
	La componente è stata sviluppata usando JavaScript ed i moduli Axios, HTTP e Telegraf.
	 
\subsubsection{Diagramma delle classi}%%%%%%%%%%%%%%%%%OK
	\begin{figure}[H]
		\centering
		\includegraphics[scale=0.600]{res/images/BOTTELEGRAM/ClassiTelegram.png}
		\caption{Diagramma delle classi della componente bot Telegram}
		\label{Diagramma 19}
	\end{figure}
\subsubsection{Dipendenze esterne}	
	La componente ha tre dipendenze esterne:
	\begin{itemize}
		\item \textbf{Telegraf}, modulo che permette di collegarsi con le API ufficiali di Telegram. Ogni comando ne ha un riferimento. 
		\item \textbf{Axios}, modulo che permette di effettuare richieste POST e GET e di ritornare una risposta. Viene utilizzato da Server, Login e Status per comunicare con le \textit{API} e per inviare uno o più messaggi agli utilizzatori del bot;
		\item \textbf{HTTP}, modulo che permette di creare un server HTTP per restare in ascolto di eventuali richieste. Viene utilizzato da Server per ascoltare eventali richieste delle \textit{API}.    
	\end{itemize}
\subsubsection{Diagramma di sequenza}%%%%%%%%%OK
	\begin{figure}[H]
		\centering
		\includegraphics[scale=0.500]{res/images/BOTTELEGRAM/TelegramRichiestaPOST.png}
		\caption{Diagramma di sequenza che riporta la ricezione di una richiesta POST delle \textit{API} all'interno della componente bot Telegram}
		\label{Diagramma 20}
	\end{figure}

	Nel diagramma di sequenza in alto viene mostrata la ricezione di una richiesta POST dalle \textit{API}. La richiesta può essere di due tipologie: 
	\begin{itemize}
		\item \textbf{authentication:} in cui vengono inviati dalle \textit{API} un chat Id ed un codice di autenticazione; quest'ultimo dovrà poi essere inviato al chat Id specificato per permettere all'utente l'autenticazione sulla web app;
		\item \textbf{alert:} in cui vengono mandati dalle \textit{API} una lista di chat id ed un insieme di dati che poi andranno composti in un messaggio ed inviati a tutti i chat Id specificati.
	\end{itemize}
\subsubsection{Estensione}
	\paragraph{Inserimento di un nuovo comando}
		Per inserire un nuovo comando all'interno del bot è necessario creare un file .js all'interno della cartella commands.
		\newline
		Poiché sarà poi necessario esportare questo comando, per poi inserirlo all'interno di main.js, l'intestazione del nuovo comando dovrà essere del tipo:
		\begin{verbatim}const botNomeComando = (bot) => {
								bot.command( "nomeComando", (param) => {
							 		...
							 	});
						}; 
		\end{verbatim}	