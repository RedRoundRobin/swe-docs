\subsection{Data Collector}
La componente \textit{data collector} permette di reperire i dati prodotti dai gateway che sono presenti all'interno dei vari topic di Kafka all'interno di un database di tipo timeseries. Allo stesso tempo permette di filtrare i messaggi che sono stati collezionati e inviare di conseguenza dei messaggi di avviso all'interno di un apposito topic.
Il filtraggio dei dati viene fatto degli alert impostati all'interno della web-app e salvati all'interno di un database relazionale (PostgreSql);
\begin{landscape}
	\subsubsection{Diagramma delle classi}
		\begin{figure}[H]
			\centering
			\includegraphics[scale=0.500]{res/images/DATACOLLECTOR/ClassikafkaDataCollector.png}
			\caption{Diagramma delle classi della componente Data Collector}
		\end{figure}
\end{landscape}
\subsubsection{Diagrammi di sequenza}
\subsubsection{Diagrammi di attività}