\subsection{Database}
All'interno dell'architettura sono presenti:
\begin{itemize}
	\item un database relazionale (PostgreSql) che viene utilizzato per il salvataggio di configurazioni dei gateway, informazioni di utenti ed enti e per salvare le impostazioni dei grafici creati dagli utenti;
	\item un database non relazionale (Timescale) che viene invece utilizzato per salvare i dati inviati dai dispositivi, le logs degli eventi e gli alert con i valori anomali. 
\end{itemize}
Entrambi i database sono rilasciati tramite dockerfile in cui viene effettuata una prima configurazione.
\subsubsection{Timescale}
La struttura del database non relazionale è la seguente:
\begin{itemize}
	\item Sensors
	\begin{itemize}
		\item time:timestamptz
		\item real\_sensor\_id:integer
		\item real\_device\_id:integer
		\item gateway\_name:text
		\item value:double
	\end{itemize}
	\item Alerts
	\begin{itemize}
		\item time:timestamptz
		\item real\_sensor\_id:integer
		\item real\_device\_id:integer
		\item gateway\_name:text
		\item value:double
	\end{itemize}
	\item Logs
	\begin{itemize}
		\item time:timestamptz
		\item user\_id:integer
		\item ip\_address:varchar
		\item operation:text
		\item data:text
	\end{itemize}
\end{itemize}
\subsubsection{PostgreSql}
La struttura del database SQL è rappresentata nel diagramma logico sottostante:
	
		\begin{landscape}
		\begin{figure}[H]
			\centering
			\includegraphics[scale=0.600]{res/images/DATABASE/ER_Modificato.png}
			\caption{Diagramma logico del database relazionale}
			\label{Diagramma 9}
		\end{figure}
		\end{landscape}		
 