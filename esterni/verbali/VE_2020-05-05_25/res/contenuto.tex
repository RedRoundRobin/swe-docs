\section*{Introduzione}

\subsection*{Luogo e data dell'incontro}
	\begin{itemize}
		\item \textbf{luogo:}\glock{Slack};
		\item \textbf{data:} 2020-05-05;
		\item \textbf{ora di inizio:} 18:00;
		\item \textbf{ora di fine:} 18:45.
	\end{itemize}

\subsection*{Ordine del giorno}
	\begin{enumerate}
		\item chiarimenti riguardanti il ciclo \glock{PDCA};
		\item chiarimenti riguardanti il manuale utente;
		\item varie ed eventuali.
	\end{enumerate}

\subsection*{Presenze}
	\begin{itemize}
		\item \textbf{totale presenti:} 7 su 7
		\item \textbf{presenti: }
			\begin{itemize}
				\item Giuseppe Vito Bitetti (segretario);
				\item Lorenzo Dei Negri;
				\item Nicolò Frison;
				\item Fouad Mouad;
				\item Mariano Sciacco;
				\item Alessandro Tommasin;
				\item Giovanni Vidotto;
			\end{itemize}
		\item \textbf{assenti: }
			\begin{itemize}
				\item nessuno;
			\end{itemize}
		\item \textbf{partecipanti esterni:}
			\begin{itemize}
				\item Professor Tullio Vardanega.
			\end{itemize}
	\end{itemize}

\newpage
\section*{Svolgimento}
	\subsection*{1. Chiarimenti riguardanti il ciclo PDCA}
		Si sono chiesti chiarimenti rigardanti il ciclo PDCA, in particolare è stata chiarita la differenza tra miglioramento proattivo e correzione reattiva.Inoltre, è stato fatto notare come le appendici B e C del \dext{PdQ} siano molto reattive ma non apportano nulla di quantitativo.


	\subsection*{2. Chiarimenti riguardanti il manuale utente}

		Si è discusso sulla qualitá del manuale utente fornito in formato video e si sono riscontrate alcune lacune nelle informazioni fornite da quest'ultimi, inoltre sono state fatte notare come un video sia difficilmente indicizzabile e questo può portare ad una perdita di tempo da parte dell'utente per la ricerca delle informazioni.

	\subsection*{3. Varie ed eventuali}

		Sono stati chiesti chiarimenti sulla consegna dell'RA, in particolare sulla data della consegna e sulla sua modalitá.
