\section{Qualità di Processo}

\subsection{Introduzione}

Nello svolgimento del progetto si fa uso di processi attraverso cui si opera per l'intero sviluppo del prodotto. Tali processi devono sostenere degli standard che perseguono la qualità, le cui metriche di analisi sono definite in maniera chiara e precisa. Analizzando lo standard ISO/IEC 12207:1995 sono state reperite le principali tecniche da adottare per il raggiungimento di un livello di qualità soddisfacente, sia per il gruppo che per il cliente.


% Copiare da qui in Norme di Progetto

\subsection{Classificazione dei Processi}

I processi vengono definiti con la seguente nomenclatura per essere facilmente tracciati:

\[
		\text{QP}-[\lambda]
\]

Dove: 

\begin{itemize}
	\item QP : indica letteralmente \textit{Quality Process};
	\item \(\lambda\) : numero intero che indica il processo e parte da 1.
\end{itemize}

\subsection{Classificazione delle Metriche}

Le metriche sono gli indici coi quali vengono misurate nei processi e nel prodotto i gradi di qualità raggiunti. A ciascuna metrica si associa il seguente identificatore:

\[
		\text{QM}-[\delta]-[\lambda]
\]

Dove: 

\begin{itemize}
	\item QM : indica letteralmente \textit{Quality Metric}
	\item \(\delta\) : riguarda la tipologia della metrica e può assumere i valori riportati di seguito:
		\begin{itemize}
			\item \textbf{PROCESS}: indica che la metrica si associa per un processo;
			\item \textbf{PRODUCT}: indica che la metrica si associa per il prodotto;
			\item \textbf{TEST}: indica che la metrica si associa per i test;
		\end{itemize}
	\item \(\lambda\) : numero intero che indica la metrica e parte da 1.
\end{itemize}

\subsection{Monitoraggio dei Processi}

L'intento di questa sezione è quello di riportare tutte le metriche che sono utilizzate col fine di monitorare i processi lungo tutto il ciclo di sviluppo del software. I processi presi correntemente in esame sono i seguenti:

\begin{itemize}
	\item Gestione dei Costi
	\item Gestione dei Rischi
	\item Analisi dei Requisiti
	\item Progettazione 
	\item Verifica del Software
\end{itemize}

\subsubsection{Gestione dei Rischi}

\begin{itemize}
	\item \textbf{Obiettivi:}
	\item \textbf{Scopo:}
\end{itemize}

