\section{Qualità di Processo}

\subsection{Introduzione}

Nello svolgimento del progetto si fa uso di processi attraverso cui si opera per l'intero sviluppo del prodotto. Tali processi devono sostenere degli standard che perseguono la qualità, le cui metriche di analisi sono definite in maniera chiara e precisa. Analizzando lo standard ISO/IEC 12207:1995 sono state reperite le principali tecniche da adottare per il raggiungimento di un livello di qualità soddisfacente, sia per il gruppo che per il cliente.


% Copiare da qui in Norme di Progetto

\subsection{Classificazione dei Processi}

I processi vengono definiti con la seguente nomenclatura per essere facilmente tracciati:

\[
		\text{QP}-[\lambda]
\]

Dove: 

\begin{itemize}
	\item QP : indica letteralmente \textit{Quality Process};
	\item \(\lambda\) : numero intero che indica il processo e parte da 1.
\end{itemize}

\subsection{Classificazione delle Metriche}

Le metriche sono gli indici coi quali vengono misurate nei processi e nel prodotto i gradi di qualità raggiunti. A ciascuna metrica si associa il seguente identificatore:

\[
		\text{QM}-[\delta]-[\lambda]
\]

Dove: 

\begin{itemize}
	\item QM : indica letteralmente \textit{Quality Metric}
	\item \(\delta\) : riguarda la tipologia della metrica e può assumere i valori riportati di seguito:
		\begin{itemize}
			\item \textbf{PROCESS}: indica che la metrica si associa per un processo;
			\item \textbf{PRODUCT}: indica che la metrica si associa per il prodotto;
			\item \textbf{TEST}: indica che la metrica si associa per i test;
		\end{itemize}
	\item \(\lambda\) : numero intero che indica la metrica e parte da 1.
\end{itemize}

\subsection{Monitoraggio dei Processi}

L'intento di questa sezione è quello di riportare tutte le metriche che sono utilizzate col fine di monitorare i processi lungo tutto il ciclo di sviluppo del software. I processi presi correntemente in esame sono i seguenti:

\begin{itemize}
	\item Gestione delle Risorse
	\item Gestione dei Rischi
	\item Analisi dei Requisiti
	\item Progettazione 
	\item Verifica del Software
\end{itemize}

	\subsubsection{QP-1 Gestione delle Risorse}

		\paragraph{Scopo}

		Si vuole gestire la copertura di risorse disponibili per la realizzazione del progetto, monitorando i costi aggiuntivi e le tempestiche non rispettate dallo \glock{scheduling}. Questo può essere utile al cliente per capire in fase di sviluppo l'andamento del progetto a livello di gestione delle risorse.

		\paragraph{Introduzione alle metriche}

		Per la gestione delle risorse si farà uso delle seguenti metriche:

		\begin{itemize}
			\item QM-PROCESS-1. Budgeted Cost of Work Scheduled (BCWS);
			\item QM-PROCESS-2. Actual Cost of Work Performed (ACWP);
			\item QM-PROCESS-3. Budgeted Cost of Work Performed (BCWP);
			\item QM-PROCESS-4. Schedule Variance (SV);
			\item QM-PROCESS-5. Cost Variance (CV);
		\end{itemize}

		\paragraph{QM-PROCESS-1. Budgeted Cost of Work Scheduled (BCWS)}

			\subparagraph{Descrizione}
			La metrica BCWS definisce il costo pianificato per realizzare le attività di progetto alla data corrente. 

			\subparagraph{Unità di Misura}
			Il costo pianificato è misurato in EURO.

		\paragraph{QM-PROCESS-2. Actual Cost of Work Performed (ACWP)}

			\subparagraph{Descrizione}
			La metrica ACWP definisce il costo effettivamente sostenuto per realizzare le attività di progetto alla data corrente. 

			\subparagraph{Unità di Misura}
			Il costo sostenuto è misurato in EURO.

		\paragraph{QM-PROCESS-3. Budgeted Cost of Work Performed (BCWP)}

			\subparagraph{Descrizione}
			La metrica BCWP definisce il valore delle attività realizzate alla data corrente. In altre parole, misura il valore del prodotto fino ad ora realizzato.

			\subparagraph{Unità di Misura}
			Il valore del prodotto è misurato in EURO.

		\paragraph{QM-PROCESS-4. Schedule Variance (SV)}

			\subparagraph{Descrizione}
			La metrica SV indica se si è in anticipo, in ritardo o in linea rispetto alle schedulazioni pianificate per il progetto. Questo può essere utile per il cliente per valutare l'efficacia del gruppo nei confronti della realizzazione del progetto.

			\subparagraph{Unità di Misura}
			La metrica viene espressa in percentuale.

			\subparagraph{Formula}
			La formula per il calcolo della metrica è la seguente:

			\[
				\text{SV} = \frac{\text{BCWP} - \text{BCWS}}{\text{BCWS}} \times 100
			\]

			\subparagraph{Risultato}
			\begin{itemize}
				\item Un risultato \textbf{positivo} (\(> 0\)) indica che il progetto è avanti rispetto alla schedulazione.
				\item Un risultato \textbf{negativo} (\(< 0\)) indica che il progetto è indietro rispetto alla schedulazione.
				\item Un risultato \textbf{pari a zero} indica che il progetto è in linea rispetto alla schedulazione.
			\end{itemize}

		\paragraph{QM-PROCESS-5. Cost Variance (CV)}

			\subparagraph{Descrizione}
			La metrica CV indica se il valore del costo realmente maturato è maggiore, minore o uguale rispetto al costo effettivo. In altre parole, permette di comprendere con che livello di efficienza il gruppo sta sviluppando il progetto, rispetto a quanto pianificato.

			\subparagraph{Unità di Misura}
			La metrica viene espressa in percentuale.

			\subparagraph{Formula}
			La formula per il calcolo della metrica è la seguente:

			\[
				\text{CV} = \frac{\text{BCWP} - \text{ACWP}}{\text{BCWP}} \times 100
			\]

			\subparagraph{Risultato}
			\begin{itemize}
				\item Un risultato \textbf{positivo} (\(> 0\)) indica che il progetto sta sviluppando con un costo minore rispetto a quanto pianificato (maggiore efficienza).
				\item Un risultato \textbf{negativo} (\(< 0\)) indica che il progetto sta sviluppando con un costo maggiore rispetto a quanto pianificato (minore efficienza).
				\item Un risultato \textbf{pari a zero} indica che il progetto sta sviluppando con un costo in linea rispetto a quello pianificato.
			\end{itemize}