\section{Qualità di prodotto}

\subsection{Introduzione}
	Per garantire e valutare la qualità del prodotto il gruppo ha deciso di fare riferimento allo standard ISO/IEC 9126, il quale definisce i parametri per produrre un prodotto di buona qualità, questi parametri quantificano il grado di raggiungimento di tale caratteristica.
	\newline
	Oltre alle qualità presenti nello standard sopra citato, il gruppo ha deciso di utilizzare altri parametri per quantificare la qualità della documentazione fornita con il prodotto, oltre al software stesso.
	\newline
	Di seguito sono riportate le qualità che il gruppo ha ritenuto appropriate per quanto riguarda lo stato attuale del progetto.
	
%\subsection{Qualità del software}
	%\subsubsection{QP-5 Funzionabilità}
	La Funzionabilità definisce la capacità del prodotto di fornire le funzioni che soddisfano con le esigenze stabilite nell'Analisi dei Requisiti.
	\paragraph{Obiettivi}
		\begin{itemize}
			\item \textbf{appropriatezza:} viene richiesto che il prodotto metta a disposizione tutte le funzionalità richieste dall'utente;
			\item \textbf{accuratezza:} il prodotto deve riuscire a produrre risultati che rispettano l'aspettativa ed il grado di precisione richiesti;
			\item \textbf{interoperabilità:} il prodotto deve essere in grado di interagire ed operare con tutti i sistemi e vincoli specificati;
			\item \textbf{conformità:} il prodotto deve aderire a standard e regolamenti noti;
			\item \textbf{sicurezza:} i dati sensibili utilizzati e generati dal prodotto devono essere disponibili esclusivamente agli utenti e/o coloro che risultano autorizzati all'uso di tali dati.
		\end{itemize}
	\paragraph{Metriche}
	La Funzionabilità del prodotto viene valutata dal seguente criterio:
	\begin{itemize}
		\item implementazione: misura in percentuale le funzionalità (sia richieste che opzionali) implementate a fronte delle funzionalità proposte.
	\end{itemize}
	\begin{center}
		\begin{tabular}{|c|c|c|c|}
			\hline
			ID & Nome & Valore ottimale & Valore accettabile \\
			\hline
			QM-PROD-1 & Implementazione (IMP)  & 100\% & 100\% \\
			\hline
		\end{tabular}
	\end{center}

	
\subsection{Monitoraggio dei documenti}
	Le qualità dei documenti, monitorate con delle metriche precise di qualità, sono le seguenti:
	
	\begin{itemize}
		\item QC-1 Comprensione.
	\end{itemize}

	\subsubsection{QC-1 Comprensione}
Tutti i documenti devono essere leggibili e comprensibili, queste qualitá derivano dalla correttezza lessicografica, grammaticale, e semantica.
	\paragraph{Obiettivi}
		\begin{itemize}
			\item \textbf{leggibilità:} per garantire la leggibilità dei documenti si è deciso di utilizzare l'indice di Gulpease come indicatore per questa caratteristica;
			\item \textbf{correttezza:} i documenti presentati non devono contenere errori ortografici di alcun genere.
		\end{itemize}
	\paragraph{Metriche}
	La comprensione dei documenti viene valutata dai seguenti criteri:
	\begin{itemize}
		\item QM-PROD-1 Indice di Gulpease;
    \item QM-PROD-2 Correttezza ortografica.
	\end{itemize}
	\begin{center}
		\rowcolors{2}{lightest-grayest}{white}
		\begin{longtable}{|c|c|c|c|}
			\rowcolor{lighter-grayer}
			\hline
			ID & Nome & Valore ottimale & Valore accettabile \\
			\hline
			QM-PROD-1 & Indice di Gulpease (GULP) & 100 & 40 \\
      \hline
			QM-PROD-2 & Correttezza ortografica (CORT) & 0 & 0 \\
			\hline
			\caption{Indici di qualità per le metriche di comprensione del prodotto}
		\end{longtable}
		
	\end{center}

	
\subsection{Monitoraggio del software}
	Le qualità del software, monitorate con delle metriche precise di qualità, sono le seguenti:
	
	\begin{itemize}
		\item QP-2 Sicurezza;
		\item QP-3 Affidabilità;
		\item QP-4 Efficienza;
		\item QP-5 Usabilità;
		\item QC-6 Manutenibilità.
	\end{itemize}

	\subsubsection{QC-2 Sicurezza}
Con il termine sicurezza si intende la proprietà del prodotto di essere privo di vulnerabilità all'interno del codice che lo implementa.

	\paragraph{Metriche utilizzate}
	\begin{itemize}
		\item QM-PROD-4 Numero di vulnerabilità rilevate (NVUL);
		\item QM-PROD-5 Tempo di risoluzione vulnerabilità (TVUL).
	\end{itemize}

	\paragraph{Indici di qualità}
	\begin{center}
		\rowcolors{2}{lightest-grayest}{white}
		\begin{tabular}{|c|c|c|}
			\rowcolor{lighter-grayer}
			\hline
			\textbf{ID metrica} & \textbf{Valore preferibile} & \textbf{Valore accettabile} \\
			\hline
			QM-PROD-4 (NVUL) & \(= 0\) & \(\le 5\) \\
			\hline
			QM-PROD-5 (TVUL) & \(= 00:00:00\) &  \(\le 01:00:00\) \\
			\hline
		\end{tabular}
	\end{center}

	\subsubsection{QC-3 Affidabilità}
Con il termine affidabilità si intende la capacità del prodotto di mantenere un livello minimo di funzionamento, deciso a priori, in qualsiasi situazione di utilizzo. Quindi, oltre all'affidabilità in senso stretto, vengono considerate anche la correttezza e la tolleranza agli errori.

	\paragraph{Metriche utilizzate}
	\begin{itemize}
		\item QM-PROD-6 Numero di bug rilevati (BUGR);
		\item QM-PROD-7 Tempo stimato risoluzione bug (TBUG).
	\end{itemize}

	\paragraph{Indici di qualità}
	\begin{center}
		\rowcolors{2}{lightest-grayest}{white}
		\begin{tabular}{|c|c|c|}
			\rowcolor{lighter-grayer}
			\hline
			\textbf{ID metrica} & \textbf{Valore preferibile} & \textbf{Valore accettabile} \\
			\hline
			QM-PROD-6 (BUGR) & \(= 0\) &\(\le 3\) \\
			\hline
			QM-PROD-7 (TBUG) & \(= 00:00:00\) & \(\le 01:00:00\) \\
			\hline
		\end{tabular}
	\end{center}

	\subsubsection{QC-4 Efficienza}
Con efficienza si intende la capacità del prodotto di mantenere un livello adeguato di prestazioni in determinate situazioni.
	
	\paragraph{Metriche}
	L'efficenza del prodotto viene valutata dal seguente criterio:
	\begin{itemize}
		\item risposta media (RM): è una misurazione in ms che indica il tempo medio di risposta per ogni richiesta.
	\end{itemize}
	\begin{center}
		\rowcolors{2}{lightest-grayest}{white}
		\begin{tabular}{|c|c|c|c|}
			\rowcolor{lighter-grayer}
			\hline
			ID & Nome & Valore ottimale & Valore accettabile \\
			\hline
			QM-PROD-8 & Risposta media (RM) & <00:00:05.000 & <00:00:07.500 \\
			\hline
		\end{tabular}
	\end{center}

	\subsection{Usabilità}
L'usabilità definisce la capacità del prodotto di essere appreso ed usato dall'utente in determinate situazioni
	\subsubsection{Obiettivi}
		\begin{itemize}
			\item \textbf{comprensibilità:} determina la facilità di utilizzo e di comprensione del prodotto e delle sue funzionalità da parte dell'utente;
			\item \textbf{apprendibilità:} definisce il livello di impegno richiesto, da parte dell'utilizzatore, per imparare ad usare il prodotto;
			\item \textbf{operabilità:} stabilisce il grado con cui il software riesce a mettere il suo utilizzatore in condizione di sfruttare il prodotto per i suoi fini;
			\item \textbf{attrattiva:} la proprietà del software di produrre un'esperienza d'uso gradevole per l'utente.
		\end{itemize}
	\subsection{Metriche}

	\subsubsection{Manutenibilità}
Fornisce un indicatore sul livello di semplicità per quanto riguarda la modifica, correzzione ed estendibilità del prodotto software.
	\paragraph{Obiettivi}
		\begin{itemize}
			\item \textbf{analizzabilità:} determina la facilità con cui é possibile analizzare e localizzare un errore all'interno del codice;
			\item \textbf{modificabilità:} definisce la capacitá del prodotto di apportare una modifica o una estensione;
			\item \textbf{stabilità:} il software deve essere un grado di essere usato anche in caso le modifiche apportate siano errate;
			\item \textbf{testabilità:} determina la capacità del software di essere testato facilmente per fornire una validazione delle modifica apportate.
		\end{itemize}
	\paragraph{Metriche}
	La manutenibilità del prodotto viene valutata dai seguenti criteri:
	\begin{itemize}
		\item complessità del codice: consiste nel rapporto tra il numero di linee di commento ed il numero di linee di codice;
		\item complessità della classe: si contano il numero dei metodi di una classe per avere una misura della sua complessità;
		\item complessità del metodo: si valuta la lunghezza del metodo e il numero di chiamate (dirette) ad altri metodoi da parte di quest'ultimo.
	\end{itemize}
	\begin{center}
		\begin{tabular}{|c|c|c|c|c|}
			\hline
			ID & Nome & Misurazione & Valore ottimale & Valore accettabile \\
			\hline
			QM-PROD-7 & Complessità del codice (CCOD) & \(CCOD = \frac{\# linee commento}{\# linee codice}\) & NAN & NAN \\
			\hline
			QM-PROD-8 & Complessità della classe (CCLA) & \textit{CCLA = \# numero metodi} & NAN & NAN \\
			\hline
			QM-PROD-9 & Complessità del metodo (CMET) & \(CMET = \frac{\# linee codice}{\# chiamate interne ad altri metodi+1}\) & NAN & NAN \\
			\hline
		\end{tabular}
	\end{center}
