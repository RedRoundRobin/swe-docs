\subsubsection{QC-1 Comprensione}
Tutti i documenti devono essere leggibili e comprensibili, queste qualità derivano dalla correttezza lessicografica, grammaticale, e semantica.
\newline
Inoltre, i video sostitutivi del classico manuale utente cartaceo devono essere ragionevolmente corti e sufficientemente chiari nell'illustrare le funzionalità del prodotto, in modo da evitare ambiguità ed esporre tutte e solo le azioni necessarie.
	
	\paragraph{Metriche}
	La comprensione dei documenti e dei video viene valutata dai seguenti criteri:
	\begin{itemize}
		\item QM-PROD-1 \glock{Indice di Gulpease};
    	\item QM-PROD-2 Correttezza ortografica;
    	\item QM-PROD-3 Limite alla lunghezza dei video tutorial.
	\end{itemize}
	\begin{center}
		\rowcolors{2}{white}{lightest-grayest}
		\begin{longtable}{|c|c|c|c|}
			\hline
			\rowcolor{lighter-grayer}
			ID & Nome & Valore ottimale & Valore accettabile \\
			\hline 
			\endhead
			QM-PROD-1 & \glock{Indice di Gulpease} (GULP) & \(\ge 80\) & \(\ge 60\) \\
 		  	\hline
			QM-PROD-2 & Correttezza ortografica (CORT) & 0 & 0 \\
			\hline
			QM-PROD-3 & Limite Lunghezza dei video tutorial (LLVT) & \(\le 00:01:00\)  & \(\le 00:02:00\) \\
			\hline
			\caption{Indici di qualità per le metriche di comprensione del prodotto}
		\end{longtable}
	\end{center}
