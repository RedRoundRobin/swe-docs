\appendix
\addcontentsline{toc}{section}{Appendice}

\section{Resoconto attività di verifica}

\subsection{Verifica processi}

\subsubsection{Calcolo metriche gestione delle risorse}
Per verificare che i costi stabiliti rientrino in quanto preventivato si trovano a seguire dei grafici contenenti i risultati ottenuti:

	\paragraph{Budgeted cost of work scheduled}
		\begin{figure}[H]
			\centering
			\includegraphics[width=0.8\linewidth]{./res/images/BCWS_1.png}
			\caption{Grafico contenente il costo preventivato in Euro per fase.}
			\label{fig:Grafico contenente il costo preventivato in Euro per fase.}
		\end{figure}

	\paragraph{Actual cost of work performed}
		\begin{figure}[H]
			\centering
			\includegraphics[width=0.8\linewidth]{./res/images/ACWP_1.png}
			\caption{Grafico contenente il costo effettivo in Euro per fase.}
			\label{fig:Grafico contenente il costo effettivo in Euro per fase.}
		\end{figure}

	\paragraph{Budgeted cost of work performed}
		\begin{figure}[H]
			\centering
			\includegraphics[width=0.8\linewidth]{./res/images/BCWP_1.png}
			\caption{Grafico contenente il valore del prodotto in Euro per fase.}
			\label{fig:Grafico contenente il valore del prodotto in Euro per fase.}
		\end{figure}

	\paragraph{Schedule variance}
		\begin{figure}[H]
			\centering
			\includegraphics[width=0.8\linewidth]{./res/images/SV_1.png}
			\caption{Grafico contenente la schedule variance per fase.}
			\label{fig:Grafico contenente la schedule variance per fase.}
		\end{figure}

	\paragraph{Cost Variance}
		\begin{figure}[H]
			\centering
			\includegraphics[width=0.8\linewidth]{./res/images/CV_1.png}
			\caption{Grafico contenente la cost variance per fase.}
			\label{fig:Grafico contenente la cost variance per fase.}
		\end{figure}

\subsubsection{Calcolo metriche gestione dei rischi}
Per monitorare i rischi non preventivati riscontrati durante l'avanzamento del progetto si sono raccolte le misurazioni nel seguente grafico:

\paragraph{Unbudgeted Risks}

	\begin{figure}[H]
		\centering
		\includegraphics[width=0.8\linewidth]{./res/images/RischiNonPrevent_1.png}
		\caption{Grafico periodo/rischio della fase di analisi dei requisiti.}
		\label{fig:Grafico periodo/rischio nel periodo di analisi dei requisiti.}
	\end{figure}

	\begin{figure}[H]
			\centering
			\includegraphics[width=0.8\linewidth]{./res/images/RischiNonPrevent_2.png}
			\caption{Grafico Grafico periodo/rischio per la fase di consolidamento dei requisiti.}
			\label{fig:Grafico periodo/rischio per la fase di consolidamento dei requisiti.}
	\end{figure}

	\begin{figure}[H]
			\centering
			\includegraphics[width=0.8\linewidth]{./res/images/RischiNonPrevent_3.png}
			\caption{Grafico periodo/rischio per le fasi che vanno dalla progettazione della technology baseline all'incremento 5.}
			\label{fig:Grafico periodo/rischio per le fasi che vanno dalla progettazione della technology baseline all'incremento 5.}
	\end{figure}
	
%\subsection{Verifica prodotto}

\subsection{Verifica documentazione}

\subsubsection{Calcolo metriche comprensione}

\paragraph{Indice di Gulpease}
Per verificare quanto sono leggibili i documenti redatti si utilizza \glock{l'indice di Gulpease}, di seguito il grafico contenente i risultati ottenuti:

\begin{figure}[H]
	\centering
	\includegraphics[width=0.8\linewidth]{./res/images/gulpease_1.png}
	\caption{Grafico periodo/indice di gulpease nel periodo di analisi dei requisiti.}
	\label{fig:Grafico indice di gulpease periodo di analisi dei requisiti.}
\end{figure}

\begin{figure}[H]
	\centering
	\includegraphics[width=0.8\linewidth]{./res/images/gulpease_2.png}
	\caption{Grafico periodo/indice di gulpease nel periodo di progettazione della technology baseline.}
	\label{fig:Grafico indice di gulpease periodo di progettazione della technology baseline.}
\end{figure}

\paragraph{Correttezza ortografica}
È stato effettuato un controllo di ortografia e di seguito vengono illustrati i risultati ottenuti:

\begin{center}
	\rowcolors{2}{lightest-grayest}{white}
	\begin{longtable}{|c|c|c|}
	\hline
	\rowcolor{lighter-grayer}
	\textbf{Documento} & \textbf{Errori ortografici} & \textbf{Accettabile} \\
	\hline
	\endfirsthead

	\hline
	Piano di Progetto & 0 & Sì \\
	\hline
	\hline
	Norme di Progetto &  0 & Sì \\
	\hline
	\hline
	Studio di fattibilità & 0 & Sì \\
	\hline
	\hline
	Glossario & 0 & Sì \\
	\hline
	\hline
	Piano di qualifica & 0 & Sì \\
	\hline
	\hline
	Media verbali & 0 & Sì \\
	\hline
	\hline
	Analisi dei requisiti & 0 & Sì \\
	\hline
	\caption{Tabella delle medie degli errori di ortografia durante il periodo di analisi dei requisiti}
	\end{longtable}
\end{center}

\begin{center}
	\rowcolors{2}{lightest-grayest}{white}
	\begin{longtable}{|c|c|c|}
	\hline
	\rowcolor{lighter-grayer}
	\textbf{Documento} & \textbf{Errori ortografici} & \textbf{Accettabile} \\
	\hline
	\endfirsthead

	\hline
	Piano di Progetto & 0 & Sì \\
	\hline
	\hline
	Norme di Progetto &  0 & Sì \\
	\hline
	\hline
	Glossario & 0 & Sì \\
	\hline
	\hline
	Piano di qualifica & 0 & Sì \\
	\hline
	\hline
	Media verbali & 0 & Sì \\
	\hline
	\hline
	Analisi dei requisiti & 0 & Sì \\
	\hline
	\caption{Tabella delle medie degli errori di ortografia durante il periodo di progettazione della technology baseline}
	\end{longtable}
\end{center}

Di seguito il grafico riguardante gli errori rilevati:

\begin{figure}[H]
	\centering
	\includegraphics[width=0.8\linewidth]{./res/images/ortografia_1.png}
	\caption{Grafico periodo/errori ortografici nel periodo di analisi dei requisiti.}
	\label{fig:Grafico errori ortografici durante il periodo di analisi dei requisiti.}
\end{figure}

\begin{figure}[H]
	\centering
	\includegraphics[width=0.8\linewidth]{./res/images/ortografia_2.png}
	\caption{Grafico periodo/errori ortografici nel periodo di progettazione della technology baseline.}
	\label{fig:Grafico errori ortografici durante il periodo di progettazione della technology baseline.}
\end{figure}
