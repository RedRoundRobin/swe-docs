% ---------------------------
%	  [ NOME DOCUMENTO ]
%	----------------------
%	   Red Round Robin
%  Progetto di SWE (2019-20)
% ---------------------------


% ---------------------------
% Packages e configurazioni
% ---------------------------

\documentclass[11pt,a4paper,fleqn]{article}
    
    \usepackage{geometry}
    \geometry{
        margin=1.05in,
        top=20mm,
        bottom=25mm,
        left=20mm,
        right=20mm,
    }
    \usepackage{graphicx} 
    \usepackage[utf8]{inputenc}
    \usepackage{eurosym}
    \usepackage[italian]{babel}
    \usepackage{float}
    \usepackage{subcaption}
    \usepackage{wrapfig}
    \usepackage{fancyhdr} 
    \usepackage{lastpage}
    \usepackage{amsfonts}
    \usepackage{fancyvrb}
    \usepackage[pages=some]{background}


    \pagestyle{fancy}
    \fancyhead{}
    \fancyhead[R]{\rightmark}
    %\fancyhf{}
    \fancyfoot{}
    \fancyfoot[L]{ NOME DEL DOCUMENTO - VERSIONE (vX.X.X)}
    \renewcommand{\headrulewidth}{0pt}
    \renewcommand{\footrulewidth}{0.1pt}
    \rfoot{ Pagina \thepage \hspace{1pt} di \pageref{LastPage}}
    
    \usepackage{xcolor}

    \setlength{\parindent}{1.8em}
	\setlength{\parskip}{1.2em}
	\renewcommand{\baselinestretch}{1.1}

    \usepackage{hyperref}         
    \hypersetup{
        colorlinks,
        linkcolor={blue!80!black},
        citecolor={blue!50!black},
        urlcolor={red!50!black}
    }

    \PassOptionsToPackage{hyphens}{url}\usepackage{hyperref}


% Equivalente a <hr>

\newcommand{\hr}{\par\vspace{-.1\ht\strutbox}\noindent\hrulefill\par}


% Tabelle e tabulazione

\setlength{\tabcolsep}{10pt}
\renewcommand{\arraystretch}{1.4}

% Unicode per simbolo euro

\DeclareUnicodeCharacter{20AC}{\euro}


\backgroundsetup{
scale=1.0,
color=black,
opacity=0.4,
angle=0,
contents={%
  \includegraphics[height=297mm]{res/images/background.png}
  }%
}


    % Codice

    \usepackage{xcolor}
    \usepackage{listings}

    \renewcommand{\lstlistingname}{Snippet}
    \renewcommand{\lstlistlistingname}{Lista di \lstlistingname s}    
     
    \definecolor{codegreen}{rgb}{0,0.4,0.2}
    \definecolor{codegray}{rgb}{0.5,0.5,0.5}
    \definecolor{codepurple}{rgb}{0.58,0,0.82}
    \definecolor{backcolour}{rgb}{0.95,0.95,0.96}
   
    \lstdefinestyle{chungusHighlight}{
        frame=tb,
        backgroundcolor=\color{backcolour},   
        commentstyle=\color{codegreen},
        keywordstyle=\color{magenta}\textbf,
        numberstyle=\color{codegray},
        stringstyle=\color{codepurple},
        basicstyle={\ttfamily},
        breakatwhitespace=false,         
        breaklines=true,                 
        captionpos=b,                    
        keepspaces=true,                 
        numbers=left,                    
        numbersep=5pt,                  
        showspaces=false,                
        showstringspaces=false,
        showtabs=false,
        numbers=none,                  
        tabsize=2
    }

\lstset{style=chungusHighlight}


% ---------------------------
% Dati frontespizio
% ---------------------------

\title{\hr \huge NOME DEL DOCUMENTO \\ \vspace{11pt} \large \textsc{NOME DEL PROGETTO}\hr}
\author{}
\date{}

% ---------------------------
% Composizione del documento
% ---------------------------

\begin{document} 

% Frontespizio

\pagenumbering{gobble}

% FRONTESPIZIO

% Logo aziendale

\begin{figure}[t!]
    \centering
    \includegraphics[height=8.5em]{res/images/logo.png}
\end{figure}

\vspace{-7.5em}

% Titolo principale

\maketitle 
\thispagestyle{empty}


% Riferimenti email e sito web

\vspace{-7em}

\begin{center}
    \href{https://www.redroundrobin.site}{www.redroundrobin.site} --- \href{mailto:redroundrobin.site@gmail.com}{redroundrobin.site@gmail.com}
\end{center}

\vspace{1em}

% Informazioni documento

\begin{table}[ht]
  \begin{center}
    \label{tab:Informazioni_Documento}
    \begin{tabular}{r|l}
        \multicolumn{2}{c}{ \textsc{Informazioni sul documento} } \\
        \hline
    	\textbf{Versione} &  \docVersione \\
		\textbf{Uso} &  \docUso \\
        \textbf{Stato} & \docStatus \\
		\textbf{Destinatari} & \docDestinatari \\
		\textbf{Redattori} & \docRedattori \\
		\textbf{Verificatori} & \docVerificatori \\
		\textbf{Approvazione} &  \docApprovazione \\
    \end{tabular}
  \end{center}
\end{table}


% Descrizione del documento

%\vspace{0em}

%\begin{center}
%   \textbf{Descrizione}\\
%  \docDescrizione
%\end{center}



% Registro delle modifiche

\newpage
\pagenumbering{arabic}
\BgThispage
\section*{Registro delle modifiche}

\begin{center}
	\rowcolors{2}{lightest-grayest}{white}
	\begin{longtable}{|c|p{3.5cm}|c|p{3cm}|p{3cm}|}
	\hline
	\rowcolor{lighter-grayer}
	\textbf{Versione} & \textbf{Descrizione} & \textbf{Data} & \textbf{Autore} & \textbf{Ruolo} \\
	\hline
	\endfirsthead

	% ----- Modificare da qui -----
	1.0.0+b0.18 & Approvazione del documento & 2020-05-14 & Lorenzo Dei Negri & Responsabile \\
	\hline
	0.0.2+b0.18 & Stesura e revisione del documento & 2020-05-14 & Giuseppe Vito Bitetti e Alessandro Tommasin & Amministratore e verificatore \\
	\hline
	0.0.1+b0.18 & Creazione iniziale del documento & 2020-05-14 & Giuseppe Vito Bitetti & Redattore \\
	\hline

	\end{longtable}
\end{center}


% Tabella dei contenuti

\newpage
\BgThispage
\tableofcontents

% Sezioni

\newpage
% -----------------------
% Sezioni da inserire
% -----------------------
% Pro tip: usare il comando \yetAnotherSectionNamed{nome_file}

% NOTA BENE: AGGIUNGI LE SEZIONI IN ORDINE ALFABETICO

\yetAnotherSectionNamed{a}
\yetAnotherSectionNamed{b}


\end{document}


% EOF