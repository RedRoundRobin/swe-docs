\subsection{Sviluppo}
		\subsubsection{Scopo}
			Il processo di sviluppo definisce i compiti e le attività da intraprendere volte al raggiungimento del prodotto finale richiesto dal proponente.
		\subsubsection{Aspettative}
			Per una corretta implementazione di questo processo è necessario fissare:
				\begin{itemize}
					\item Obbiettivi di sviluppo;
					\item Vincoli tecnologici e di design.
				\end{itemize}	
			Il prodotto finale deve rispettare i requisiti e le aspettative del proponente, superando i test definiti dalle norme di qualità.
		\subsubsection{Descrizione}
			Il processo di sviluppo, secondo lo standard ISO/IEC 12207:1995, si articola nelle seguenti attività:
				\begin{itemize}
					\item Analisi dei Requisiti;
					\item Progettazione;
					\item Codifica.
				\end{itemize}
		\subsubsection{Attività}
			Di seguito verranno analizzate dettagliatamente le attività menzionate nella sezione precedente.
			\paragraph{Analisi dei requisiti}
				Gli Analisti si occupano di stilare il documento di Analisi dei Requisiti, il cui scopo è appunto quello di definire ed elencare tutti i requisiti del capitolato. Il documento finale conterrà:
				\begin{itemize}
					\item Descrizione generale del prodotto;
					\item Argomentazioni precise ed affidabili per i Progettisti;
					\item Casi d'uso rappresentati tramite diagrammi UML;
					\item Fissare funzionalità e requisiti concordi con le richieste del cliente;
					\item Stima dei costi.
				\end{itemize}
				I requisiti verranno classificati per facilitarne la comprensione e vengono identificati, in maniera univoca, secondo il seguente schema identificativo:
				
				\centering\textbf R[]
			
			\paragraph{Progettazione}

			\paragraph*{Codifica}

		\subsubsection{Strumenti}