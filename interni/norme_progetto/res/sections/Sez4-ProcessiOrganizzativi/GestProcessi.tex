\subsection{Gestione dei Processi}

	\subsubsection{Scopo}
	
		Il processo di gestione dei processi ha lo scopo di:
		
		\begin{itemize}
			\item identificare i possibili rischi e definirne la gestione;
			\item definire un modello di sviluppo;
			\item pianificare i task da svolgere in base alle scadenze temporali;
			\item calcolare un preventivo in termini di ore e costi suddiviso per ruoli; 
			\item calcolare un preventivo a finire delle risorse richieste, visto il consuntivo di periodo.
		\end{itemize}
		
	\subsubsection{Aspettative}
		
		Le principali aspettative del processo di gestione dei processi sono:
	
		\begin{itemize}
			\item agire preventivamente per evitare i rischi identificati e, qualora si verificassero, limitare le loro ripercussioni sull'efficacia e l'efficienza del lavoro svolto dal gruppo;
			\item effettuare una pianificazione ragionevole dei task da svolgere ed assegnarli in modo equilibrato ai diversi ruoli, tenendo conto dei tempi e delle risorse disponibili;
			\item gestire i componenti del gruppo e i loro task in modo da facilitare la collaborazione e la comunicazione interna tra di loro;
			\item mantenere sotto controllo l'andamento del progetto, monitorando il lavoro svolto dal gruppo in modo tale da non comprometterne l'efficienza.
		\end{itemize}
	
	\subsubsection{Descrizione}
	
		Le attività previste dal processo di gestione dei processi sono raccolte nel \textit{Piano di Progetto}, la cui redazione è in carico al responsabile, con la collaborazione dell'amministratore.\newline
		Nello specifico, sono trattati:
		
		\begin{itemize}
			\item introduzione;
			\item analisi dei rischi, classificazione degli stessi;
			\item istanziazione dei processi che realizzano il modello di sviluppo adottato;
			\item pianificazione dei task e assegnazione degli stessi ai ruoli di progetto;
			\item stima dei costi in termini di tempo e risorse;
			\item calcolo delle risorse necessarie per terminare il progetto, visto il bilancio del lavoro svolto nel periodo;
			\item revisione delle attività sulla base dei riscontri dei rischi.
		\end{itemize}
	
	\subsubsection{Ruoli di Progetto}
	
		Ogni membro del gruppo deve ricoprire più ruoli, i quali sono cambiati a rotazione con una frequenza che permetta ad ogni componente di assumere almeno una volta ogni ruolo previsto per il progetto e, allo stesso tempo, di garantire la continuità delle attività in corso.\newline
		Le attività assegnate ad ogni ruolo sono programmate ed organizzate nel \textit{Piano di Progetto}.\newline
		Di seguito vengono descritti tutti i ruoli richiesti dal progetto.
	
		\paragraph{Responsabile}
		
			Il responsabile accentra tutte le responsabilità di pianificazione, controllo, gestione e coordinamento di attività e risorse all'interno del progetto. Inoltre svolge la funzione di intermediario verso le persone esterne al gruppo, quali committente e proponente del capitolato, ed è il responsabile ultimo dei risultati del progetto.\newline
			In particolare si occupa di:
			
			\begin{itemize}
				\item elaborare ed emanare piani e scadenze;
				\item approvare l'emissione dei documenti;
				\item coordinare le attività, le risorse e i componenti del gruppo;
				\item gestire le criticità incontrate dal gruppo;
				\item redigere l'\textit{Organigramma} e il \textit{Piano di Progetto};
				\item approvare l'\textit{Offerta} sottoposta al committente.
			\end{itemize}
		
		\paragraph{Amministratore}
		
			L'amministratore è incaricato della gestione dell'ambiente di lavoro.\newline
			All'interno del gruppo egli:
			
			\begin{itemize}
				\item è responsabile dell'efficacia e dell'efficienza dell'ambiente di sviluppo e di tutte le installazioni di supporto;
				\item è responsabile della redazione ed attuazione dei piani e delle procedure per la gestione della qualità;
				\item controlla le versioni e le configurazioni del prodotto;
				\item gestisce la documentazione del progetto;
				\item collabora alla redazione del \textit{Piano di Progetto};
				\item redige le \textit{Norme di Progetto}.
			\end{itemize}
		
		\paragraph{Analista}
		
			L'analista è il responsabile di tutte le attività di analisi svolte durante l'\textit{Analisi dei Requisiti}, al cui termine hanno fine anche tutti i sui incarichi all'interno del gruppo. Egli infatti è una figura che non è presente all'interno del gruppo per tutta la durata del progetto.\newline
			L'analista ha il compito di:
			
			\begin{itemize}
				\item studiare il dominio applicativo del progetto;
				\item definire i requisiti del progetto;
				\item redigere lo \textit{Studio di Fattibilità} e l'\textit{Analisi dei Requisiti}.
			\end{itemize}
		
		\paragraph{Progettista}
		
			Il progettista è il responsabile di tutte le attività di progettazione svolte durante la \textit{Progettazione dell'Architettura} e la \textit{Progettazione di Dettaglio}.\newline
			Il progettista deve:
			
			\begin{itemize}
				\item prendere decisioni riguardanti gli aspetti tecnici del progetto, favorendo l'efficacia e l'efficienza;
				\item definire l'architettura del prodotto da sviluppare, perseguendo la sua efficienza, efficacia e manutenibilità, tramite l'utilizzo di apposite tecnologie individuate a partire dai requisiti definiti dall'analista;
				\item redigere la \textit{Specifica Tecnica}, la \textit{Definizione di Prodotto} e la parte pragmatica del \textit{Piano di Qualifica}.
			\end{itemize}
		
		\paragraph{Programmatore}
		
			Il programmatore è il responsabile di tutte le attività di codifica effettuate per lo sviluppo del progetto.\newline
			In particolare, il programmatore è responsabile:
			
			\begin{itemize}
				\item dell'implementazione della \textit{Specifica Tecnica} redatta dal progettista;
				\item della codifica mirata alla realizzazione del prodotto;
				\item della codifica di componenti di ausilio necessarie per l'esecuzione delle prove di verifica e validazione.
			\end{itemize}
		
		\paragraph{Verificatore}
		
			Il verificatore è il responsabile di tutte le attività di verifica dei documenti e del codice scritti dagli altri componenti del gruppo. Il suo compito è quello di trovare errori, di qualunque tipo, nei prodotti che controlla e di segnalare tali errori a chi ha la responsabilità diretta sul quel prodotto, in modo che possa apportare le dovute correzioni.\newline
			Il verificatore non ha il compito di correggere gli errori rilevati, deve quindi:
			
			\begin{itemize}
				\item esaminare i prodotti in fase di revisione, utilizzando le tecniche e gli strumenti definiti nelle \textit{Norme di Progetto};
				\item indicare eventuali errori riscontrati nel prodotto in esame;
				\item segnalare eventuali errori rilevati al responsabile dell'oggetto in fase di verifica, in modo che possa correggerli.
			\end{itemize}	
	
	\subsubsection{Procedure}
	
		\paragraph{Gestione delle Comunicazioni}
		
		\paragraph{Gestione degli Incontri}
		
		\paragraph{Gestione degli Strumenti di Coordinamento}
		
		\paragraph{Gestione dei Rischi}
	
	\subsubsection{Strumenti}