\section{Introduzione}
	\subsection{Scopo del documento}
		Il documento ha lo scopo di definire quelle che sono le regole su cui si basa il way of working\ped{G} del gruppo Red Round Robin per lo svolgimento del progetto. Le attività che possono essere trovate all'interno di questo documento sono state prese da processi appartenenti allo standard ISO 12207. Tutti i membri del gruppo sono quindi tenuti a prendere visione di questo documento così da garantire uniformità e coesione all'interno del progetto.   
	\subsection{Scopo del prodotto}
		DA FARE
	\subsection{Glossario}
		Viene fornito un \textit{Glossario v0.0.1} per evitare possibili ambiguità relative alle terminologie utilizzate nei vari documenti. Nel documento sarà presente a pedice di quelle che riteniamo delle parole una '\textbf{G}'.
	\subsection{Riferimenti}

		\paragraph{Riferimenti normativi}
			\begin{itemize}
				\item \textbf{Standard ISO/IEC 12207:1995: } 
				\url{https://www.math.unipd.it/~tullio/IS-1/2009/Approfondimenti/ISO_12207-1995.pdf}
				\item \textbf{Capitolato d'appalto Cx -} 
				\url{}
			\end{itemize}	
		\paragraph{Riferimenti informativi}
			\begin{itemize}
				\item Da aggiungere man mano che si fa riferimento alle slide del prof
				\item Guardare bene gli approfondimenti sul sito:
				\url{https://www.math.unipd.it/~tullio/IS-1/2019/} 
			\end{itemize}
		

