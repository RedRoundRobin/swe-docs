\subsection{Documentazione}

		\subsubsection{Scopo}
			Lo scopo principale di questo capitolo è fornire una guida esaustiva di tutti gli standard e regole per quanto riguarda la stesura ed approvazione dei documenti.
		\subsubsection{Aspettative}
			Si vuole fornire uno strumenti per la stesura dei documenti che sia unico per tutto il guppo in modo da avere una documentazione uniforme e adempiente agli standard e regole sotto riportate.
		\subsubsection{Descrizione}
			Questo capitolo fornisce i dettagli su come deve essere redata e verificata la documentazione. Tutte le norme descritte devono essere adempite in pieno da tutti i documenti, sia interno che esterni, rilasciati durante il ciclo di vita del software.

		\subsubsection{Ciclo di vita}
			Il ciclo di vita dei documenti è suddiviso in vari processi, eventualmente ripetibili:
			\begin{itemize}
				\item \textbf{Stesura:} è il processo di scrittura del documento in sé, questa attivitá viene assegnata ad un redatore che, una volta terminata, fará riferimento al responsabile, il quale fará avanzare il documento nella fase successiva;
				\item \textbf{Verifica:} è il processo eseguito dai verificatori, i quali hanno il compito di controllare che la stesura del documento sia avvenuta in modo corretto, sintatticamente e semanticamente, seguendo le norme di progetto. Ogni problema viene riferita al responsabile che provvederá a notificare il redatore e riporterá il documento in fase di Stesura. Quando quasta fase ha successo il responsabile fará avanzare il documento nell'ultima fase del ciclo di vita;
				\item \textbf{Approvazione:} è l'ultima attivitá del ciclo di vita del documento, in questa fase il verificatore ha terminato il suo compito ed ha comunicato l'esito positivo al responsabile. Il responsabile procederá a confermare il documento ed eseguire il rilascio.
			\end{itemize}

		\subsubsection{Template LaTeX}
			Si è deciso di utilizzare una struttura template \LaTeX{} per facilitare il versionamento e la stesura dei documenti. Inoltre, l'utilizzo di tale struttura fornisce uniformitá al layout di tutti i documenti.

		\subsubsection{Struttura dei documenti}
			Tutti i documenti hanno una struttura predefinita e determinata:
			\paragraph{Frontespizio}
				Il frontespizio ha la funzione di fornire i dati principali del documento. Esso presenterá il logo e relativo nome del team, il titolo del documento e la sua appartenenza ad un determinato progetto, le informazioni sul documento quali:
				\begin{itemize}
					\item \textbf{versione:} versione attuale del documento;
					\item \textbf{uso:} destinazione d'uso del documento, che potrá essere ``interno'' o ``esterno'';
					\item \textbf{stato:} attuale stato del documento, che potrá essere ``in redazione'' o ``approvato'';
					\item \textbf{destinatari:} destinatari del documento;
					\item \textbf{redatori:} lista delle persone che si sono accupate della stesura dello specifico documento;
					\item \textbf{verificatori:} lista delle persone che si sono occupati della fase di verifica dello specifico documento;
					\item \textbf{approvazione:} nominativo della persona che ha approvato il documento per il rilascio.
				\end{itemize}
				Come ultimo elemento nella pagina verrá fornita una breve descrizione del documento. Tutti gli elementi di quasta pagina sono centrati ed incolonnati.
			\paragraph{Registro modifiche}
				Il registro delle modifiche è la seconda pagina del documento e consiste in una tabella contenente le informazioni riguardanti il ciclo di vita del documento. La tabella contiene:
				\begin{itemize}
					\item \textbf{versione:} versione del documento relativa alla modifica effettuata;
					\item \textbf{descrizione:} breve descrizione della modifica effettuata;
					\item \textbf{data:} data in cui la modifica è stata effettuata;
					\item \textbf{autore:} nominativo della persona che ha effetuato la modifica;
					\item \textbf{ruolo:} ruolo della persona che ha effettuato la modifica;
				\end{itemize}
			\paragraph{Indice}
				L'indice ha una struttura standard: numero e titolo del capitolo, con eventuali sottosezioni, e il numero della pagina del contenuto; Inoltre, ogni titolo è un link alla pagina del contenuto.
			\paragraph{Contenuto principale}
				?
			\paragraph{note a piè di pagina}
				Il pié di pagina contiene il titolo e la versione del documento nella parte sinistra, mentre nella parte destra sono presenti il numero della pagina attuale, in numeri romani nel caso la pagina non faccia parte del contenuto preincipale del documento, es. indice, ed il numero totale delle pagine di cui è composto il documento.

		\subsubsection{Classificazione dei documenti}
			\paragraph{Documenti ufficiosi}
				I documenti ufficiosi sono utilizzati all'interno dell'ambiente di lavoro ed sono divisi in due categorie:
				\begin{itemize}
					\item \textbf{informativi:} hanno il mero scopo di passare informazioni meno rilevanni tra i membri del gruppo (es. appunti, richieste, riflessioni);
					\item \textbf{proto-ufficiali:} sono tutti i documenti che in futuro diventeranno ``ufficiali'' ma sono in attesa di revisione e verifica.
				\end{itemize}
			\paragraph{Documenti ufficiali}
				I documenti ufficiali sono tutti quei documenti che:
				\begin{itemize}
					\item sono stati revisionati, verificati ed approvati dal responsabile di progetto;
					\item sono gli unici documenti che possono essere rilasciati all'esterno del gruppo di progetto.
				\end{itemize}
			\paragraph{Verbali}
				I verbali hanno lo scopo di riassumere, in modo concreto e preciso, tutti gli argomenti che sono stati discussi in una riunione, sia interna che esterna. È prevista un'unica stesura del verbale per ogni riunione.

				I verbali interni...
				I verbali esterni...
			\paragraph{Glossario}
				Il glossario ha la funzione di disambiguare alcune parole all'interno di determinati contesti. Al suo interno saranno presenti tutte parole con le seguenti caratteristiche:
				\begin{itemize}
					\item sono presenti in almeno un documento;
					\item trattano argomenti di natura tecnica;
					\item trattano argomenti ambigui e\/o poco conosciuti;
					\item rappresentano delle sigle e\/o degli acronimi.
				\end{itemize}
				Inoltre, il glossario è strutturato in maniera precisa seguendo due regole:
				\begin{itemize}
					\item i termini seguono l'ordine lessicografico;
					\item ogni termine è spiegato in maniera chiara e in alcun modo ambigua.
				\end{itemize}
				La stesura del glossario deve avvenire in parallelo alla stesura dei documenti al fine di evitare confusione tra i termini. Inoltre, ogni parola nei documenti, presente nel glossario, deve essere caratterizzata dallo stile ``maiuscoletto'' con il pedice ``g''.
			\paragraph{Lettere}

		\subsubsection{Norme tipografiche}
			\paragraph{Convenzioni sui nomi dei file}
			\paragraph{Glossario}
				I termini appartenenti al glossario si possono identificare dallo stile della parola, in particolare si è deciso di utilizzare lo stile ``maiuscoletto'' con una ``g'' come pedice, per esempio ????.
			\paragraph{Stile del testo}
				I vari stili del testo hanno una specifica funzione semantica.
				\begin{itemize}
					\item \textbf{corsivo:} viene utilizzato per denotare termini tecnici appartenenti ad una particolare tecnologia, esempio \textit{branch};
					\item \textbf{grassetto:} viene utilizzato per evidenziare le parole con la definizione della stessa in seguito, per esempio in un elenco puntato, queste includeranno i due punti in grassetto, per esempio ``\textbf{def.:} abbreviazione per la parola definizione''; oppure per denotare le sezioni e sotto-sezioni dei documenti;
					\item \textbf{maiuscoletto:} viene utilizzato per denotare parole che sono:
						\begin{itemize}
							\item riferimenti a documenti esterni, con pedice una ``d'';
							\item appartenenti al glossario, con pedice una ``g''.
						\end{itemize}
				\end{itemize}
			\paragraph{Elenchi puntati}
				Ogni elemento dell'elenco deve essere seguito da un punto e virgola, fatta eccezione per l'ultimo elemento che sará seguito da un punto; di conseguenza la prima lettera di ogni sentenza deve essere minuscola. Gli elenchi avaranno punto elenco differente a seconda della loro tipologia:
				\begin{itemize}
					\item per gli elenchi non ordinati si è scelto di usare come punto elenco un cerchietto pieno e come sub-punto elenco il trattino;
					\item per gli elenchi ordinati si è optato per un punto elenco ``flessibile'', ossia possono essere usati sia i numeri che i letterali, purché quest'ultimi siano in minuscolo, seguiti da un punto, esempio ``1.'' o ``a.''.
				\end{itemize}
			\paragraph{Formati comuni}
				\begin{itemize}
					\item \textbf{date:} viene utilizzato lo standard italiano UNI EN 28601:1993, esempio gg-mm-aaaa;
					\item \textbf{versione:} viene utilizzato il formato vXX.XX.XX.
				\end{itemize}
			\paragraph{Sigle}

		\subsubsection{Elementi grafici}
			\paragraph{Tabelle}
			\paragraph{Immagini}
			\paragraph{Diagrammi UML}

		\subsubsection{Strumenti}
			\paragraph{LaTeX}
			\paragraph{TexStudio, TexMaker e TexLive con IDE}
