\subsection{Garanzia della Qualità}

	\subsubsection{Scopo}

	Si occupa di stabilire una metrica precisa per tutti i servizi nell'ambito della verifica e della validazione, mantenendo un certo grado di qualità che rimanga uniforme e misurabile durante tutto il ciclo di vita del software.

	\subsubsection{Aspettative}

	Il sistema qualità deve fornire delle metriche di giudizio uniformi volte a quantificare in maniera comprensibile la correttezza dei documenti e del software. Ciò va unito anche all'affidabilità nello svolgimento dei processi di verifica, che vanno monitorati e guidati nell'intera procedura, senza lasciare a interpretazioni. Pertanto, ci si aspetta:
	\begin{itemize}
		\item un prodotto software di qualità;
		\item una documentazione completa e facilmente comprensibile per tutti;
		\item dei processi che seguono dei punti ben specificati per l'analisi di qualità;
		\item una comunicazione chiara e semplice delle problematiche relative alla qualità tra i membri del team;
		\item una registrazione dei risultati ottenuti.
	\end{itemize}

	\subsubsection{Descrizione}

	La garanzia della qualità si compone di diversi controlli che devono essere effettuati per:
	\begin{itemize}
		\item il software;
		\item la documentazione;
		\item tutti i processi che portano alla realizzazione della documentazione e del software.
	\end{itemize}

	Per ogni processo mirato alla qualità si definiscono delle metriche che vengono riportate in ciascuna sezione del presente documento.
	La registrazione dei risultati ottenuti dall'analisi della qualità sono salvati con degli appositi report.

		\paragraph{Obbiettivi Qualità di Prodotto}

		La qualità del prodotto viene garantita attraverso l'attuazione dei processi di verifica e validazione basati su fondamenti normativi. In particolare, definiamo quanto segue:
		\begin{itemize}
			\item \textbf{Verifica:} processo di analisi continua che garantisce qualità dei processi di fornitura del prodotto;
			\item \textbf{Validazione:} processo di controllo del prodotto volto a confermare le aspettative, i requisiti e le funzionalità concordate.
		\end{itemize}

		L'insieme di questi processi deve portare a un miglioramento continuo del prodotto, che viene sottoposto agli standard di qualità riportati nel \glock{way of working}.

		\paragraph{Obbiettivi Qualità di Processo}

		La qualità di processo deve essere perseguita nel corso del ciclo di vita del software attraverso i principi di efficacia ed efficienza mirati al prodotto.
		Nello specifico definiamo quanto segue:
		\begin{itemize}
			\item \textbf{Efficacia:} si richiede un prodotto valido in relazione alle aspettative;
			\item \textbf{Efficienza:} i processi devono convergere con costi ridotti in termini di risorse a pari qualità di prodotto.
		\end{itemize}

		Ciascun processo va migliorato durante la sua esecuzione facendo uso di monitoraggi mirati che permettano di acquisire, attraverso l'esperienza, una risposta critica alla qualità stessa del processo. \\


	% \subsubsection{Standard utilizzati}

	% Da aggiungere - modificare questa parte

	% Il miglioramento continuo, inoltre, viene standardizzato da PDCA, SPY e SPICE.  % Quale usare?


	\subsubsection{Classificazioni metriche} % Da mettere nelle singole sezioni delle attività

	Per ciascuna attività, sia che riguardi la documentazione, il software o il monitoraggio di processo, si riporta nella relativa sezione una classificazione delle metriche di qualità.

	\begin{center}
		\begin{longtable}{|p{2.5cm}|p{2cm}|p{3cm}|p{7.5cm}|}
			\hline
			Classe di appartenenza & ID & Nome & Misurazione \\
			\hline
			Funzionalità & QM-PROD-1 & Implementazione (IMP) & \(IMP = \frac{\# funzionalità implementate}{\# funzionalità proposte}\times100\) \\
			\hline
			Affidabilità & QM-PROD-2 & Densità errori (DE) & \(DE = \frac{\# test passati}{\# test condotti}\times100\) \\
			\hline
			Affidabilità & QM-PROD-3 & Complessità dei test di classe (CTCLA) & \textit{CTCLA = \# dei test che coinvolgono la classe} \\
			\hline
			Efficienza & QM-PROD-4 & Risposta media (RM) & \textit{RM = tempo di risposta in ms} \\
			\hline
			Usabilità & QM-PROD-5 & Profondità dell'albero delle azioni (PAA) & \textit{PAA = \# delle azioni} \\
			\hline
			Usabilità & QM-PROD-6 & Profondità dell'albero delle pagine (PAP) & \textit{PAP = \# delle pagine visitate dall'utente} \\
			\hline
			Manutenibilità & QM-PROD-7 & Complessità del codice (CCOD) & \(CCOD = \frac{\# linee commento}{\# linee codice}\) \\
			\hline
			Manutenibilità & QM-PROD-8 & Complessità della classe (CCLA) & \textit{CCLA = \# numero metodi} \\
			\hline
			Manutenibilità & QM-PROD-9 & Complessità del metodo (CMET) & \(CMET = \frac{\# linee codice}{\# chiamate interne ad altri metodi+1}\) \\
			\hline
			Comprensione & QM-PROD-10 & Indice di Gulpease (GULP) & \(GULP = 89+\frac{300\times\# frasi-10\times\#lettere}{\#parole}\) \\
			\hline
			Comprensione & QM-PROD-11 & Correttezza ortografica (CORT) & \textit{CORT = \# numero di errori ortografici} \\
			\hline
			Gestione delle risorse & QM-PROC-1 & Budgeted Cost of Work Scheduled (BCWS) & \textit{costo pianificato in \euro{}} \\
			\hline
			Gestione delle risorse & QM-PROC-2 & Actual Cost of Work Performed (ACWP) & \textit{costo affrontato sino alla data/attività corrente in \euro{}} \\
			\hline
			Gestione delle risorse & QM-PROC-3 & Budgeted Cost of Work Performed (BCWS) & \textit{costo pianificato sino alla data/attività corrente in \euro{}} \\
			\hline
			Gestione delle risorse & QM-PROC-4 & Schedule Variance (SV) & \(\text{SV} = \frac{\text{BCWP} - \text{BCWS}}{\text{BCWS}} \times 100\) \\
			\hline
			Gestione delle risorse & QM-PROC-5 & Cost Variance (CV) & \(\text{CV} = \frac{\text{BCWP} - \text{ACWP}}{\text{BCWP}} \times 100\) \\
			\hline
			Gestione dei rischi & QM-PROC-6 & Unbudgeted Risks (UR) & \textit{\# numero di nuovi rischi non preventivati}  \\
			\hline
			Analisi dei requisiti & QM-PROC-7 & Satisfied Mandatory Requirements (SMR) & \(\text{SMR} = \frac{\text{requisiti obbligatori soddisfatti}}{\text{requisiti obbligatori totali}} \times 100\) \\
			\hline
			Analisi dei requisiti & QM-PROC-8 & Satisfied Desirable Requirements (SDR) & \(\text{SDR} = \frac{\text{requisiti desiderabili soddisfatti}}{\text{requisiti desiderabili totali}} \times 100\) \\
			\hline
			Analisi dei requisiti & QM-PROC-9 & Satisfied Optional Requirements (SOR) & \(\text{SOR} = \frac{\text{requisiti opzionali soddisfatti}}{\text{requisiti opzionali totali}} \times 100\) \\
			\hline
			Verifica del software & QM-PROC-10 & Branch Coverage (BCOV) & \(\text{BCOV} = \frac{\text{Numero di rami eseguiti}}{\text{Numero totale di rami}} \times 100\) \\
			\hline
			Verifica del software & QM-PROC-11 & Condition Coverage (COCOV) & \(\text{COCOV} = \frac{\text{Numero di operandi eseguiti}}{\text{Numero totale di operandi}} \times 100\) \\
			\hline
			Verifica del software & QM-PROC-12 & Statement Coverage (SCOV) & \(\text{SCOV} = \frac{\text{Numero di statement eseguiti}}{\text{Numero totale di statement}} \times 100\) \\
			\hline
			Verifica del software & QM-PROC-13 & Passed Test Cases Percentage (PTCP) & \(\text{PTCP} = \frac{\text{Numero di test passati}}{\text{Numero totale di test eseguiti}} \times 100\) \\
			\hline
			Verifica del software & QM-PROC-14 & ailed Test Cases Percentage (FTCP) & \(\text{FTCP} = \frac{\text{Numero di test falliti}}{\text{Numero totale di test eseguiti}} \times 100\) \\
			\hline
			Verifica del software & QM-PROC-15 & Bug-Fixing Percentage (BFP) & \(\text{BFP} = \frac{\text{Numero di difetti corretti}}{\text{Numero di difetti trovati}} \times 100\) \\
			\hline
			Verifica del software & QM-PROC-16 & Test Effectiveness (TE) & \(\text{TE} = \frac{\text{Difetti trovati con i test}}{\text{Numero totale di difetti trovati}} \times 100\) \\
			\hline
		\end{longtable}
	\end{center}

	% \subsubsection{Strumenti} % Ce ne sono?
