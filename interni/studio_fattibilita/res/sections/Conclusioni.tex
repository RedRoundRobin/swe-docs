\section{Conclusioni finali e motivazioni del capitolato scelto}

Dopo una attenta analisi di tutti i seminari di approfondimento effettuati dai relativi proponenti, si è deciso di optare per un argomento che racchiudesse il maggiore grado di interesse del gruppo. Basandosi sulle proprie conoscenze, si è cercato inoltre di determinare un capitolato che potesse produrre una applicazione concreta della materia di interesse del proponente, cercando di raggiungere una qualità adeguata per il prodotto finale. \ 
Per questo motivo si è voluto optare per il capitolato C6 (\textit{ThiReMa}) che racchiude lo sviluppo di una web application in diretto contatto con il mondo IoT e che attraversa l'impiego delle più moderne tecnologie (quali Docker e i database non relazionali) per giungere alla gestione remota di dispositivi fisici su larga scala. 

L'idea che ha menzionato l'azienda proponente per questo capitolato, inoltre, ci ha particolarmente colpito: \textit{nascondere la complessità e ricavare l'informazione dai dati}. Dal punto di vista pratico, è esattamente quello che un utente finale si aspetta e su cui si è stimolato il nostro interesse per poter approfondire queste pratiche di sviluppo per la gestione e manipolazione del dato. \
Chiaramente la nostra sfida si baserà sull'apprendere il più possibile queste nuove tecnologie così da realizzare un prodotto valido che soddisfi le aspettative e i requisiti richiesti dal capitolato, nonchè presentare tutta la documentazione aggiuntiva richiesta dal proponente prima e dopo l'inizio dello sviluppo del software. 

