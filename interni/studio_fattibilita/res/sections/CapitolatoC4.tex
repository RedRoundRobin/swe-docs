
\subsection{Capitolato 4 - Predire in Grafana}

		\subsubsection{Informazioni generali}
			\begin{itemize} %enumerate per elenco puntato numerato
			  \item \textbf{Nome:} Predire in Grafana - Monitoraggio predittivo per DevOps;
			  \item \textbf{Proponente:} Zucchetti SPA;
				\item \textbf{Committente: }Prof. Tullio Vardanega e Prof. Riccardo Cardin.
			\end{itemize}

		\subsubsection{Descrizione capitolato}
			La società Zucchetti, per eseguire il monitoraggio dei propri sistemi, ha scelto \glock{Grafana}, un prodotto \glock{open source} estendibile tramite \glock{plug-in}.
			Il capitolato chiede la realizzazione di alcuni \glock{plug-in}, da integrare in \glock{Grafana}, che effettuino delle previsioni sul flusso dei dati raccolti al fine di monitorare il corretto funzionamento del sistema e consigliare eventuali interventi alla linea di produzione del software tramite allarmi o segnalazioni.

		\subsubsection{Finalità del progetto}
			É richiesto lo sviluppo di due \glock{plug-in} per Grafana che siano in grado di predire un possibile stato dell'imminente futuro della macchina. La predizioni devo essere effettuate tramite \glock{Regressione Lineare} (RL) e/o \glock{Support Vector Machine} (SVM).
			La Regressione Lineare consente di analizzare dati uniformi ed é molto utile se i dati hanno un dominio continuo e una dispersione uniforme (\glock{omoschedastici}) con andamento lineare. Per avere una predizione valida si deve guardare la dispersione del grafo dei residui, nel quale una dispersione uniforme é indice di una RL corretta ed attendibile.
			Gli SVM sono algoritmi di classificazione per dati con dominio discreto e sono scalabili su spazi di dati molto grandi. Essi consistono nell'effettuare trasformazioni dello spazio dati per trovare un predizione che sia in grado di separare i dati in varie classi.
			L'addestramento degli algoritmi di \glock{machine learning} viene effettuato in un ambiente separato ed isolato da \glock{Grafana} tramite \glock{suit} di generazione dati come \glock{JMeter} che consente di creare dati compatibili. A fine addestramento i predittori necessitano di essere salvati in file \glock{JSON} per poi essere associati al flusso dati di \glock{Grafana}. Una volta ottenute le previsioni sui dati essi dovranno essere resi disponibili al sistema di creazione di grafici di Grafana ed essere visualizzati sulla \glock{dashboard}.
			\newline
			Come obbiettivi opzionali viene richiesto di dare la possibilità di definire ``alert'' in base a soglie raggiunte dalle previsioni; di dare la possibilità di effettuare trasformazioni sulle misure in modo da ottenere regressioni non lineare (es. logaritmiche o esponenziali); di consentire l'apprendimento constante tramite i dati raccolti dall'agente JMX. Viene richiesto, inoltre, di fornire anche i dati di bontà dei predittori. Per gli SVM sono ``Precision'', che consiste nel rapporto tra i dati veri predetti veri e i dati falsi predetti veri(veri trovati\/falsi positivi); e ``Recall'', che è il rapporto tra i veri valutati veri ed i veri non valutati tali; invece per RL si fa riferimento al coefficiente ``R\textsuperscript{2}'' che rappresenta il rapporto tra gli errori rispetto alla retta e gli errori rispetto alla media del codominio dei dati.


		\subsubsection{Tecnologie interessate}
			Il progetto sfrutterà le seguenti tecnologie:
			\begin{itemize}
			  \item \textbf{\glock{Support Vector Machines} (SVM):} modello di classificazione basata su trasformazioni dello spazio dei dati;
			  \item \textbf{\glock{Regressione Lineare} (RL):} modello di predizione basata sulla differenza dei quadrati che genera una predizione lineare del tipo \(y=ax+b\);
				\item \textbf{\glock{JMX}:} agente installato sulle macchine che raccoglie i dati necessari alle predizioni;
				\item \textbf{\glock{JMeter}:} \glock{suit} di generazione di dati virtuali per l'addestramento;
			  \item \textbf{\glock{JavaScript}:} linguaggio non tipizzato orientato ad oggetti sia \glock{client side} (\glock{JavaScript}, \glock{JQuery}) che \glock{server side} (\glock{AJAX}, \glock{Node.JS}).
			\end{itemize}

		\subsection{Vincoli del progetto}
			Non sono stati riscontrati vincoli per quanto concerne lo sviluppo del progetto.

		\subsubsection{Aspetti positivi}
			\begin{itemize}
			  \item L’azienda, consapevole del fatto che gli algoritmi di \glock{machine learning} non fanno parte del corso di studi della laurea triennale, è disponibile alla formazione ed alla fornitura di questi tipi di algoritmi;
			  \item i requisiti obbligatori del capitolato sono basilari e danno la possibilità di ampliare notevolmente il progetto con i requisiti opzionali;
			  \item uso di un linguaggio flessibile, all'avanguardia e richiesto sul mercato.
			\end{itemize}

		\subsubsection{Criticità}
			\begin{itemize}
				\item Integrare un sistema già esistente con conseguente innalzamento delle ore di documentazione sul sistema, oltre che sulle tecnologie;
				\item le tecnologie di \glock{machine learning} non sono ancora ben chiare al gruppo;
				\item necessita uno studio approfondito della documentazione di varie \glock{suit} e programmi per poter avere un'idea più concreta del progetto.
			\end{itemize}

		\subsubsection{Conclusioni}
			Il capitolato ha stimolato un discreto interesse nel gruppo dal momento che si usano tecnologie importanti e consente di acquisire conoscenze spendibili nel mondo reale. L'idea di integrare un sistema già esistente però ha convinto il gruppo ad orientarsi su un altro capitolato.
