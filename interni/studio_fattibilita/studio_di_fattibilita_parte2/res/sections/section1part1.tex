\section{Valutazione Capitolati}

\subsection{Capitolato 4 - Predire in Grafana}

		\subsubsection{Informazioni generali}
			\begin{itemize} %enumerate per elenco puntato numerato
			  \item \textbf{Nome:} Predire in Grafana - Monitoraggio predittivo per DevOps
			  \item \textbf{Proponente:} Zucchetti SPA
			  \item \textbf{Link:} https://www.math.unipd.it/~tullio/IS-1/2019/Progetto/C4.pdf
			\end{itemize}
		
		\subsubsection{Descrizione capitolato}
			La società Zucchetti, per eseguire il monitoraggio dei propri sistemi, ha scelto Grafana, un prodotto open source estendibile tramite plug-in.
			Il capitolato chiede la realizzazione di un plug-in, da integrare a Grafana stessa, che effettuino delle previsioni sul flusso dei dati raccolti al fine di monitorare il corretto funzionamento del sistema e consigliare eventuali interventi alla linea di produzione del software tramite allarmi o segnalazioni.
		
		\subsubsection{Finalità del progetto}
			È richiesto lo sviluppo di un plug-in Grafana che, dopo aver letto da un file json la definizione di calcoli da applicare ed essere e stati associati ad alcuni nodi della rete, producano dei valori che possano essere aggiunti al flusso del monitoraggio come se fossero dati rilevati sul campo.
			La definizione dei calcori dovrà essere effetuata tramite Support Vector Machines o Regressione Lineare. L’addestramento delle SVM e della RL dovranno essere effetuati in una applicazione apposita a cui verranno forniti i dati di test o alternativamente direttamente in Grafana quando non sono necessari dati aggiuntivi per l’addestramento.
			Una volta ottenuti i dati ed effetuata la previsione su di essi dovranno essere resi disponibili al sistema di creazione di grafici di Grafana ed essere visualizzati sulla dashboard.
		
		\subsubsection{Tecnologie interessate}
			\begin{itemize}
			  \item \textbf{Support Vector Machines (SVM):} Modello di apprendimento supervisionato associati ad algoritmi di apprendimento per la regressione e la classificazione.
			  \item \textbf{Regressione Lineare (RL):} Modello di apprendimento supervisionato associati ad algoritmi di apprendimento per la regressione.
			  \item \textbf{JavaScript:} Linguaggio di scripting orientato agli oggetti ed agli eventi.
			\end{itemize}
		
		\subsubsection{Aspetti positivi:}
			\begin{itemize}
			  \item L'azienda, consapevole del fatto che gli algoritmi di Machine Learning non fanno parte del corso di studi della laurea triennale, è disponibile alla formazione su questo tipo di algoritmi.
			  \item I requisiti obbligatori del capitolato sono tuttosommato non troppo esigenti dando la possibilità di ampliare notevolmente il progetto con i requisiti opzionali.
			\end{itemize}
			
		\subsubsection{Criticità:}
			\begin{itemize}
			  \item Integrare un sistema già esistente.
			  \item Le tecnologie di Machine Learning non hanno suscitato interesse nel gruppo.
			\end{itemize}
		
		\subsubsection{Conclusioni:}
			Il capitolato non è riuscito a stimolare l'interesse del gruppo dal momento che si tratta di ampliare un sistema già esistente con un plug-in e per le tecnologie che verranno apprese in corso d'opera.
			Il gruppo ha espresso quindi un gudizione negativo nel confronto di questo capitolato.